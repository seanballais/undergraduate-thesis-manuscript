\chapter{Review of Related Literature} \label{sec:back}




\section{Background Information (optional)}

A brief section giving background information may be necessary, especially
if your work spans two or more traditional fields. That means that your
readers may not have any experience with some of the material needed to
follow your thesis, so you need to give it to them. A different title than
that given above is usually better; e.g., ``A Brief Review of Frammis
Algebra" \cite{Vapnik}.

\section{Review of the State of the Art} \label{sec:StateoftheArt}

Here you review the state of the art relevant to your thesis. Again, a
different title is probably appropriate; e.g., ``State of the Art in Zylon
Algorithms." The idea is to present (critical analysis comes a little bit
later) the major ideas in the state of the art right up to, but \textbf{not
including}, your own personal brilliant ideas.

You organize this section by idea, and not by author or by publication.
For example if there have been three important main approaches to Zylon
Algorithms to date, you might organize subsections around these three
approaches, if necessary:

\begin{itemize}
\item Iterative Approximation of Zylons
\item Statistical Weighting of Zylons
\item Graph-Theoretic Approaches to Zylon Manipulation
\end{itemize}



