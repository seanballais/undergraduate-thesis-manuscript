\chapter{Review of Related Literature} \label{sec:rrl-title}
Facility layout problems is a well-known problem with many decades of research behind it. The benefits it provides in many situations, such as in factories and office spaces, while being a rather challenging problem to solve motivates the ongoing research for it. As mentioned in the previous chapter, facility layout problems are known to be NP-Hard. This makes the use of metaheuristics popular when it comes to solving FLPs. Based on our review, genetic algorithms are the most popular form of metaheuristics that have been used to solve various forms of facility layout problems  \cite{Hosseini-Nasab2018}. Newer forms of metaheuristics, such as variable neighbourhood search, are being applied to FLPs more and more. Despite the popularity of the use of metaheuristics, exact methods have also been adapted to facility layout problems but not as widely used as metaheuristics. In this chapter, we will be reviewing many of the previous works available within the literature of facility layout problems. This will provide us with a good enough understanding about the current state of research around facility layout problems and allow us to find where this work may fit in the ocean of previous works. Due to the vastness of the field, we will only be focusing particularly on unequal area facility layout problems in this chapter. We will also be mentioning works that have been used in other types of facility layout problems. Additionally, most of the works mentioned here will be from the past 10 years.

\section{Works Using Metaheuristics}
Exact methods have been used to solve many different problems, particularly those problems with known optimal solutions. Unfortunately, not all problems have known best solutions, and looking for them will take a reasonably long time to find \cite{Glover2015}. Facility layout problems are under these types of problems. As such, metaheuristics are popular when it comes to solving FLPs \cite{Drira2007}. This is further supported by the survey of Hosseini-Nasab et al. (2018) \cite{Hosseini-Nasab2018}, where it is found that most papers they have surveyed used a metaheuristic to solve FLPs.

\subsection{Genetic Algorithms}

There are various forms of metaheuristics. Common of which is the genetic algorithm \cite{Hosseini-Nasab2018}. Genetic algorithm is a form of evolutionary-based metaheuristic. It works by breeding a generation of individuals from pairs of parents (through a crossover operation). The children produced from the breedings may undergo mutation to improve the diversity of the population and help find better solutions. This process is repeated until the algorithm reaches a certain number of generations, or a stopping condition has been met \cite{Luke2013Metaheuristics}.

\subsubsection{Pure Genetic Algorithms}
In literature, to our knowledge, there is no term called pure genetic algorithms. However, for the sake of ease of differentiation, we will be referring to the genetic algorithms in prior related works without any combination with other optimization algorithms as "pure". Genetic algorithms that have been combined with other algorithms will be called "hybridized" genetic algorithms. These algorithms are discussed in the next subsection.

One work that uses pure genetic algorithms is that of Hasda et al. (2016). In their work, they attempted to solve the static unequal-area facility layout problem using a modification of the genetic algorithm. They have also used elitism in their modification. Their variation of the genetic algorithm still includes the traditional operators (despite being named differently in their paper), but with the inclusion of a rotation operator. The rotation operator is simply an operator that rotates a facility of a solution. It is similar to that of the mutation operator in that it only runs when a certain rotation probability is reached, and this probability is user-defined and is recommended to be of a small value. Their method has proven to be slightly better than the works they compared it to \cite{Hasda2017}. Another paper, proposed by Besbes et al. (2020) \cite{Besbes2020}, also modifies the genetic algorithm for use with the facility layout problem. In most papers dealing with facility layout problems, the distance between the geometric centers of facilities considered in the objective function are computed using Euclidean or rectilinear distance. Besbes et al. changed this by using the A* algorithm to compute the distance more realistically and consider obstacles. This use of A* search has produced better solutions than when using the other two distance computation functions. Fernando, J., and Resende, M. (2015) modified the genetic algorithm to change the parent selection behaviour. Their method has the population partitioned into the elite individuals (those with the best fitness, and they are a small number) and non-elite individuals. During breeding, one parent will be from the elite partition and the other from the non-elite partition. The facilities are also arranged using maximal spaces and placing facilities in those spaces in such a way that it is as close to the rest of the facilities as possible. Their scheme created the better solutions for many of the datasets they applied it to compared to previous studies \cite{Fernando2015}. Placing facilities within a site layout, especially when considering multiple time periods, is another problem that may be considered to be under facility layout problems. Farmakis, P, and Chassiakos, A. (2018) developed a genetic algorithm to minimize the resource transportation costs between facilities or between facilities and work fields, and facility construction and relocation costs in a construction site considering changing requirements over time (an instance of the dynamic facility layout problem). According to the authors, their method produces "rational solutions", and the consideration for the changing demands over time produced a more effective layout than a static layout \cite{Farmakis2018}. Similar to Farmakis, P, and Chassiakos, A. (2018), Peng et al. (2018) are also dealing with an instance of the dynamic facility layout problem. In their problem instance, they are also considering transport devices, such as conveyers and tow trains. A Monte Carlo simulation method has been used to generate scenarios, due to demand uncertainty. The crossover and mutation probability of an offspring in their genetic algorithm implementation is determined by its fitness relative to the fitness of the other individuals. The authors compared their genetic algorithm to particle swarm optimization and found that it produces the better results in all but two experiment data sets \cite{Peng2018}. A genetic algorithm for facility layout problems can produce subjectively more desirable results when interactively given feedback from a decision maker. This idea is being utilized in the work of Garcia-Hernandez et al. (2013). % Talk more about "Facility layout design using a multi-objective interactive genetic algorithm to support the DM".

Genetic algorithms may also be applied to non-traditional configurations of FLPs. Barriga et al. (2014) used genetic algorithms to produce the best layout of buildings in a Protoss base in classic StarCraft. The fitness of a base's configuration is based on the health of its army, workers, and pylons \cite{Barriga2014}. % Talk more about slight modifications of GAs for FLPs.

\subsubsection{Hybridized Genetic Algorithms}
There are many other papers that modified genetic algorithms to solve facility layout problems. However, many of them did not only slightly modify the genetic algorithm. Rather, they also combined it with another algorithm, usually a local search algorithm. This resulted in \textbf{hybridized algorithms} that better exploited the search space of the solution produced by the genetic algorithm.

Asl et al. (2015) \cite{Asl2015} and Asl, A. and Wong, K. (2015) \cite{Asl2015a} produced works that hybridized genetic algorithms with local search algorithms. The local search algorithms they used moved buildings in such a way that the solution generated is better than the original solution. The local search method used in the work of Asl, A. and Wong, K. (2015) moves a building in different directions. This movement was performed for each building. The best new layout produced will replace the original solution if it is better than the original solution. The paper of Asl et al. (2015) also uses this local search method, and it is referred to as Local Search 1. The same paper also uses another local search method called Local Search 2. It works the same as Local Search 1. However, it moves two buildings at the same time. Both papers also utilize a swapping method in their genetic algorithms, which swaps facility positions to find a better arrangement.

Genetic algorithms has also been hybridized with variable neighbourhood search. Variable neighbourhood search (VNS) is a relatively recent local search algorithm introduced in 1997 by Mladenovic, N. and Hansen, P.. The algorithm utilizes and moves through a set of neighbourhood structures to find the local optimum \cite{Hansen2018}, performed within the three phases of its main step \cite{Hansen2017}. In the paper of Uddin, M. (2015), genetic algorithm was used in conjunction with VNS. The author used the combined algorithm of GA-VNS to solve a problem instance of the dynamic facility layout problem. In each iteration, a percentage of the current population is subjected to breeding using a genetic algorithm, while the rest are optimized using VNS. This hybridized algorithm produced the same results with half of the datasets it was tested to compared to some of the previous works, while performing the best in two of the datasets, the worst in one, and the second best in the last \cite{Uddin2015}.

Variable neighbourhood search is not the only local search algorithm that has been hybridized with genetic algorithms for solving facility layout problems. Simulated annealing has also been combined wth genetic algorithms. Simulated annealing (SA) is a local search algorithm inspired by annealing, which is a process that finds the low energy state of a metal by melting and then cooling it slowly \cite{Lai1997}. The main idea behind SA is to slightly modify a solution to form a new solution, and that solution is only accepted when it is better than the older solution or with a certain probability when it is worse \cite{Dueck1993}. A hybrid of genetic algorithms and simulated annealing was used in the work of Pourvaziri, B., and Naderi, B. (2014) in order to solve another instance of the dynamic facility layout problem. Contrary to traditional genetic algorithms, their work utilizes multiple populations to find the solutions. Each population is involved independently of the other populations. These populations are then coalesced into a main population, which is now composed of the best individuals of the initial populations, after a pre-determined number of generations. The main population is then evolved, and the most fit solution from the population is further optimized using simulated annealing. This evolution and local search optimization is repeated until a stopping condition is met \cite{Pourvaziri2014}. 

\subsection{Non-Genetic Algorithms}

% Talk more hybridization wtth GAs.

% Add part where hybridized algorithms have been used.
