\chapter{Review of Related Literature} \label{sec:rrl-title}
The polygonal bounding area building placement problem (PBABPP) deals with the arrangement of buildings within a polygonally-shaped area. As far as the authors know, no previous research has been conducted for the problem. More so with the fact that this research also takes into account areas prone to natural hazards, namely flooding and landslides. Nevertheless, PBABPP is still an extension of the facility layout problems (FLP), which delves with determining the placement of various assets in a facility. Many techniques utilized in solving FLP instances can be adapted in PBABPP. As such, this literature review will mostly consist of prior works that attempt to solve facility layout problems. Majority of the FLP works included here utilize approximation methods since this work uses one. FLP researches are also easy to find thanks to the fact that it is NP-Hard. Being NP-Hard resulted in numerous researches being done for the field \cite{Drira2007}.

All research works in the literature produce layouts with varying degrees of fitness and performance. The process of layout generation differ from proposal to proposal based on the specific FLP instance they are working on. Various techniques are used to solve the FLP. Exact methods have been used, but stochastic-based methods like local search and population-based evolutionary algorithms are popular.

The instance of the facility layout problem that is most related to this work is the unequal area static facility layout problem (UA-SFLP). The works dealing with the UA-SFLP (and even the unequal-area dynamic facility layout problem) use a rectangle to mark the bounds of the area where assets can be placed. This is unlike the problem we are solving here where a polygonal area is used instead.
