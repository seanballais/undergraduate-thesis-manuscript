\chapter{Review of Related Literature} \label{sec:rrl-title}
Facility layout problems is a well-known problem with many decades of research behind it. The benefits it provides in many situations, such as in factories and office spaces, while being a rather challenging problem to solve motivates the ongoing research for it. As mentioned in the previous chapter, facility layout problems are known to be NP-Hard. This makes the use of metaheuristics popular when it comes to solving FLPs. Based on our review, genetic algorithms are the most popular form of metaheuristics that have been used to solve various forms of facility layout problems  \cite{Hosseini-Nasab2018}. Newer forms of metaheuristics, such as variable neighbourhood search, are being applied to FLPs more and more. Despite the popularity of the use of metaheuristics, exact methods have also been adapted to facility layout problems but not as widely used as metaheuristics. In this chapter, we will be reviewing many of the previous works available within the literature of facility layout problems. This will provide us with a good enough understanding about the current state of research around facility layout problems and allow us to find where this work may fit in the ocean of previous works. Due to the vastness of the field, we will only be focusing particularly on unequal area facility layout problems in this chapter. We will also be mentioning works that have been used in other types of facility layout problems. Additionally, most of the works mentioned here will be from the past 10 years.



% The polygonal bounding area building placement problem (PBABPP) deals with the arrangement of buildings within a polygonally-shaped area. As far as the authors know, no previous research has been conducted for the problem. More so with the fact that this research also takes into account areas prone to natural hazards, namely flooding and landslides. Nevertheless, PBABPP is still an extension of the facility layout problems (FLP), which delves with determining the placement of various assets in a facility. Many techniques utilized in solving FLP instances can be adapted in PBABPP. As such, this literature review will mostly consist of prior works that attempt to solve facility layout problems. Majority of the FLP works included here utilize approximation methods since this work uses one. FLP researches are also easy to find thanks to the fact that it is NP-Hard. Being NP-Hard resulted in numerous researches being done for the field \cite{Drira2007}.

% All research works in the literature produce layouts with varying degrees of fitness and performance. The process of layout generation differ from proposal to proposal based on the specific FLP instance they are working on. Various techniques are used to solve the FLP. Exact methods have been used, but stochastic-based methods like local search and population-based evolutionary algorithms are popular.

% The instance of the facility layout problem that is most related to this work is the unequal area static facility layout problem (UA-SFLP). The works dealing with the UA-SFLP (and even the unequal-area dynamic facility layout problem) use a rectangle to mark the bounds of the area where assets can be placed. This is unlike the problem we are solving here where a polygonal area is used instead.
