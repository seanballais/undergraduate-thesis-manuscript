\chapter{Review of Related Literature} \label{sec:rrl-title}
Facility layout problems is a well-known problem with many decades of research behind it. The benefits it provides in many situations, such as in factories and office spaces, while being a rather challenging problem to solve motivates the ongoing research for it. As mentioned in the previous chapter, facility layout problems are known to be NP-Hard. This makes the use of metaheuristics popular when it comes to solving FLPs. Based on our review, genetic algorithms are the most popular form of metaheuristics that have been used to solve various forms of facility layout problems  \cite{Hosseini-Nasab2018}. Newer forms of metaheuristics, such as variable neighbourhood search, are being applied to FLPs more and more. Despite the popularity of the use of metaheuristics, exact methods have also been adapted to facility layout problems but not as widely used as metaheuristics. In this chapter, we will be reviewing many of the previous works available within the literature of facility layout problems. This will provide us with a good enough understanding about the current state of research around facility layout problems and allow us to find where this work may fit in the ocean of previous works. Due to the vastness of the field, we will only be focusing particularly on unequal area facility layout problems in this chapter. We will also be mentioning works that have been used in other types of facility layout problems. Additionally, most of the works mentioned here will be from the past 10 years.

\section{Exact Methods}
The survey of Hosseini-Nasab et al. (2017) showed that metaheuristics are popular approaches in solving facility layout problems. However, exact methods have also been used to solve FLPs \cite{Hosseini-Nasab2018}. Exact methods are algorithms that are able to find the optimal solution for an optimization problem \cite{Dumitrescu2003}. However, they are not well-suited for large NP-Hard problems, such as the facility layout problem, due to the amount of time they will require in solving them. This does not mean that they are never used to solve NP-Hard problems. Small instances of those problems and multi-objective combinatorial optimization problems may still be solved by exact methods \cite{Jourdan2009}\cite{Ehrgott2016}. Numerous techniques may be used to improve the speed of these methods \cite{Woeginger2003}.

In 2006, Amaral, A. (2006) proposed a new mixed-integer linear programming model for the single-row facility layout problem (SRFLP). The author's model provided fewer continuous variables compared to the model he was comparing the new model against. Both models were solved with CPLEX 8.0 using a branch-and-bound method. It was found that the new model performed better than the previous one \cite{Amaral2006}. The branch-and-bound method solves optimization problems by exploring the entire search space \cite{Datta2020}. It produces a search tree of subproblems and solves a subproblem on every iteration. This is repeated until no subproblems remain \cite{Morrison2016}. Solimanpur, M., and Jafari, A. (2008) developed a branch-and-bound algorithm to solve an instance of the facility layout problem. Their method managed to find good solutions for small and medium problem instances. However, in line with the expectations of the performance of exact methods, they found that it is inefficient for large-sized problem instances \cite{Solimanpur2008}.

\section{Works Using Metaheuristics}
Exact methods have been used to solve many different problems, particularly those problems with known optimal solutions. Unfortunately, not all problems have known best solutions, and looking for them will take a reasonably long time to find \cite{Glover2015}. Facility layout problems are under these types of problems. As such, metaheuristics are popular when it comes to solving FLPs \cite{Drira2007}. This is further supported by the survey of Hosseini-Nasab et al. (2018) \cite{Hosseini-Nasab2018}, where it is found that most papers they have surveyed used a metaheuristic to solve FLPs.

\subsection{Genetic Algorithms}
There are various forms of metaheuristics. Common of which is the genetic algorithm \cite{Hosseini-Nasab2018}. Genetic algorithm is a form of evolutionary-based metaheuristic. It works by breeding a generation of individuals from pairs of parents (through a crossover operation). The children produced from the breedings may undergo mutation to improve the diversity of the population and help find better solutions. This process is repeated until the algorithm reaches a certain number of generations, or a stopping condition has been met \cite{Luke2013Metaheuristics}. We allocated a section for discussing works that utilize genetic algorithms for facility layout problems due to its popularity in terms of use within the field \cite{Hosseini-Nasab2018}.

\subsubsection{Pure Genetic Algorithms}
In literature, to our knowledge, there is no term called pure genetic algorithms. However, for the sake of ease of differentiation, we will be referring to the genetic algorithms in prior related works without any combination with other optimization algorithms as "pure". Genetic algorithms that have been combined with other algorithms will be called "hybridized" genetic algorithms. These algorithms are discussed in the next subsection.

One work that uses pure genetic algorithms is that of Hasda et al. (2016). In their work, they attempted to solve the static unequal-area facility layout problem using a modification of the genetic algorithm. They have also used elitism in their modification. Their variation of the genetic algorithm still includes the traditional operators (despite being named differently in their paper), but with the inclusion of a rotation operator. The rotation operator is simply an operator that rotates a facility of a solution. It is similar to that of the mutation operator in that it only runs when a certain rotation probability is reached, and this probability is user-defined and is recommended to be of a small value. Their method has proven to be slightly better than the works they compared it to \cite{Hasda2017}. Another paper, proposed by Besbes et al. (2020) \cite{Besbes2020}, also modifies the genetic algorithm for use with the facility layout problem. In most papers dealing with facility layout problems, the distance between the geometric centers of facilities considered in the objective function are computed using Euclidean or rectilinear distance. Besbes et al. changed this by using the A* algorithm to compute the distance more realistically and consider obstacles. This use of A* search has produced better solutions than when using the other two distance computation functions. Fernando, J., and Resende, M. (2015) modified the genetic algorithm to change the parent selection behaviour. Their method has the population partitioned into the elite individuals (those with the best fitness, and they are a small number) and non-elite individuals. During breeding, one parent will be from the elite partition and the other from the non-elite partition. The facilities are also arranged using maximal spaces and placing facilities in those spaces in such a way that it is as close to the rest of the facilities as possible. Their scheme created the better solutions for many of the datasets they applied it to compared to previous studies \cite{Fernando2015}. Placing facilities within a site layout, especially when considering multiple time periods, is another problem that may be considered to be under facility layout problems. Farmakis, P, and Chassiakos, A. (2018) developed a genetic algorithm to minimize the resource transportation costs between facilities or between facilities and work fields, and facility construction and relocation costs in a construction site considering changing requirements over time (an instance of the dynamic facility layout problem). According to the authors, their method produces "rational solutions", and the consideration for the changing demands over time produced a more effective layout than a static layout \cite{Farmakis2018}. Similar to Farmakis, P, and Chassiakos, A. (2018), Peng et al. (2018) are also dealing with an instance of the dynamic facility layout problem. In their problem instance, they are also considering transport devices, such as conveyers and tow trains. A Monte Carlo simulation method has been used to generate scenarios, due to demand uncertainty. The crossover and mutation probability of an offspring in their genetic algorithm implementation is determined by its fitness relative to the fitness of the other individuals. The authors compared their genetic algorithm to particle swarm optimization and found that it produces the better results in all but two experiment data sets \cite{Peng2018}. A genetic algorithm for facility layout problems can produce subjectively more desirable results when interactively given feedback from a decision maker. This idea is being utilized in the work of Garcia-Hernandez et al. (2013). In their work, they used two genetic algorithms to find an suboptimal layout. The first genetic algorithm is non-interactive and traditional, and only optimizes for material flow. The second genetic algorithm now takes into account the subjective evaluation by the decision maker, along with the material flow cost. This second algorithm is also partly based on NSGA-II, and only stops when the decision maker is satisfied with the results. The authors applied their genetic algorithm to two real-world cases, and found that their approach managed to capture the preferences of the decision maker and good solutions were generated in a reasonable number of iterations \cite{Garcia-Hernandez2013}.

Genetic algorithms may also be applied to non-traditional configurations of FLPs. Barriga et al. (2014) used genetic algorithms to produce the best layout of buildings in a Protoss base in classic StarCraft. The fitness of a base's configuration is based on the health of its army, workers, and pylons \cite{Barriga2014}. % Talk more about slight modifications of GAs for FLPs.

\subsubsection{Hybridized Genetic Algorithms}
There are many other papers that modified genetic algorithms to solve facility layout problems. However, many of them did not only slightly modify the genetic algorithm. Rather, they also combined it with another algorithm, usually a local search algorithm. This resulted in \textbf{hybridized algorithms} that better exploited the search space of the solution produced by the genetic algorithm.

Asl et al. (2015) \cite{Asl2015} and Asl, A. and Wong, K. (2015) \cite{Asl2015a} produced works that hybridized genetic algorithms with local search algorithms. The local search algorithms they used moved buildings in such a way that the solution generated is better than the original solution. The local search method used in the work of Asl, A. and Wong, K. (2015) moves a building in different directions. This movement was performed for each building. The best new layout produced will replace the original solution if it is better than the original solution. The paper of Asl et al. (2015) also uses this local search method, and it is referred to as Local Search 1. The same paper also uses another local search method called Local Search 2. It works the same as Local Search 1. However, it moves two buildings at the same time. Both papers also utilize a swapping method in their genetic algorithms, which swaps facility positions to find a better arrangement.

Genetic algorithms has also been hybridized with variable neighbourhood search. Variable neighbourhood search (VNS) is a relatively recent local search algorithm introduced in 1997 by Mladenovic, N. and Hansen, P.. The algorithm utilizes and moves through a set of neighbourhood structures to find the local optimum \cite{Hansen2018}, performed within the three phases of its main step \cite{Hansen2017}. In the paper of Uddin, M. (2015), genetic algorithm was used in conjunction with VNS. The author used the combined algorithm of GA-VNS to solve a problem instance of the dynamic facility layout problem. In each iteration, a percentage of the current population is subjected to breeding using a genetic algorithm, while the rest are optimized using VNS. This hybridized algorithm produced the same results with half of the datasets it was tested to compared to some of the previous works, while performing the best in two of the datasets, the worst in one, and the second best in the last \cite{Uddin2015}.

Variable neighbourhood search is not the only local search algorithm that has been hybridized with genetic algorithms for solving facility layout problems. Simulated annealing has also been combined wth genetic algorithms. Simulated annealing (SA) is a local search algorithm inspired by annealing, which is a process that finds the low energy state of a metal by melting and then cooling it slowly \cite{Lai1997}. The main idea behind SA is to slightly modify a solution to form a new solution, and that solution is only accepted when it is better than the older solution or with a certain probability when it is worse \cite{Dueck1993}. A hybrid of genetic algorithms and simulated annealing was used in the work of Pourvaziri, B., and Naderi, B. (2014) in order to solve another instance of the dynamic facility layout problem. Contrary to traditional genetic algorithms, their work utilizes multiple populations to find the solutions. Each population is involved independently of the other populations. These populations are then coalesced into a main population, which is now composed of the best individuals of the initial populations, after a pre-determined number of generations. The main population is then evolved, and the most fit solution from the population is further optimized using simulated annealing. This evolution and local search optimization is repeated until a stopping condition is met \cite{Pourvaziri2014}. 

\subsection{Non-Genetic Algorithms}
Genetic algorithms are not the only metaheuristics that have been used to solve facility layout problems. Metaheuristics, such as particle swarm optimization, simulated annealing, and even relatively recent algorithms such as fireworks algorithms, have found application in facility layout problems.

Simulated annealing without hybridization with genetic algorithm have been used in FLPs. The work of Turgay, S. (2018) is one such example. Turgay, S. sought to solve an instance of the unequal-area facility layout problem with consideration for multiple objectives. Each objective is given a weight, determining its impact, in the mathematical model of his work. The values of the weights of each objective are obtained using Shannon's entropy rule. Based on experiments, the SA implementation is capable of producing usable layouts. However, its performance was not compared against other metaheuristics \cite{Turgay2018}. McKendall et al. (2006) also developed a simulated annealing implementation that they used for the dynamic facility layout problem. They modified the simulating annealing algorithm to integrate a look-ahead/look-back strategy into the algorithm from the work of McKendall, A. and Shang, J. (2006) \cite{McKendall2006Ant}. They compared their modified SA with the traditional SA and a number of other algorithms, including a genetic algorithm implementation and a dynamic programming approach, through a set of experimental data. They discovered that their modified simulated annealing is effective in solving the dynamic facility layout problem, producing the best results in most of the problems in that large experimental dataset \cite{McKendall2006}. Another paper that used simulated annealing is that of Hosseini-Nasab, H., and Mobasheri, F. (2013). Their simulated annealing implementation utilized two mutation operators in generating neighbourhood solution. They added this modification to allow the algorithm to escape from local optimum, and allow for distinctions between solutions. They compared their work against GAMS, a modelling and optimization software \cite{GAMSSoftware}. Based on experimental results, their method can produce results significantly faster than GAMS, and can produce the best opttmum solution is mostly better or equal to the best optimum solution produced by GAMS \cite{Nasab2013}. It should be noted, however, that it may be better for them to have performed more runs for each method, compared to the five runs for their simulated annealing and one run for GAMS, to ensure that the results are statistically significant. Nevertheless, their work is still useful. Sahin, R. (2011) also developed a simulated annealing implementation for the facility layout problem. No modification to the simulated annealing algorithm was introduced. However, the mathematical model it is optimizing for considers the total material handling cost and the total closeness rating score. The author compared his work to two previous works, and found that the proposed SA approach produced same or better results than the previous works \cite{Sahin2011}.

Genetic algorithms are a population-based optimization algorithm that have seen wide use in solving facility layout problems. But, it is not the only population-based optimization algorithm that has been used in facility layout problems. Particle swarm optimization (PSO) is an optimization algorithm that has seen use in FLPs as well. Particle swarm optimization is an optimization algorithms inspired by the social behaviour of birds in finding safe locations in which to land on. This optimization algorithm utilizes particles that perform search in a search space but keep note of the best global solution and personal best solution found so far, to which they will tend to move towards to, with parameter settings determining the movement behaviour \cite{SeixasGomesdeAlmeida2019}. Derakhshan Asl, A. and Wong, K. Y. (2017) are two researchers that have utilized particle swarm optimization in their work. In their work, they developed a modified particle swarm optimization algorithm that solves the static and dynamic versions of an instance of the unequal-area facility layout problem. They applied local search and swapping methods into PSO to improve the quality of solutions, and prevent local optima for both version of UA-FLP. They compared this algorithm to a number of previous works to which they have determined that it produces better results than the previous works \cite{DerakhshanAsl2017}. Liu et al. (2018) developed a particle swarm optimization algorithm that optimizes a multi-objective function. Their algorithm also utilized objective space division method and a mutation operation and local search method to prevent facility overlaps. The algorithm was compared to previous works and was found to produce the best results in most of the experimental data set \cite{Liu2018}.

The metaheuristics simulated annealing, particle swarm optimization, and genetic algorithms first appeared decades ago. Simulated annealing was first proposed in 1983 \cite{Kirkpatrick1983}. while the genetic algorithm and particle swarm optimization were proposed in the 1990s \cite{Katoch2021}\cite{Kennedy1995}. Between the time the aforementioned algorithms were proposed and the time of writing of this paper, new optimization algorithms were proposed. Among these optimization algorithms is the coral reef optimization algorithm. Coral reef optimization (CRO) is based off of the formation and reproduction processes of coral reefs. In CRO, solutions are located in a grid initially partially populated by corals. A coral represents a solution, and the health of a coral represents its fitness. Corals in the grid sexually reproduce to produce larvae that are released into the water. Larvae settle in a grid depending on its health and the state of the grid cell they are attempting to settle in. Some corals are then made to asexually reproduce and occupy different parts of the grid with the same mechanism as larvae settling mentioned in the previous sentence. Some corals are also made to die to open up space for the next generation. These steps are performed until a stopping condition is met \cite{Salcedo-Sanz2014}. Garcia-Hernandez et al. (2019) utilized CRO in solving an instance of the facility layout problem and with the use flexible bay structures. No major modifications to CRO were used in their work. In their experimentations, they compared their CRO implementation with previous works, including those that do not use flexible bay structures as their layout representations. When comparing only against implementations with a flexible bay structure representation, their work produces the best results for most of the 17 cases. However, when considering a slicing tree structure layout as well, it only improves results for 7 of the cases \cite{Garcia-Hernandez2019}. The next year of the publication of their work, another paper combined coral reefs optimization with variable neighbourhood search. In this paper by Garcia-Hernandez (2020), the CRO algorithm remained as the original algorithm, but the larvae settling phase of the algorithm has been combined with VNS to further improve the larva/solution that is settling. Note that VNS is only ran when the larva is assured to occupy the grid cell it is settling towards. Their work also uses a relaxed flexible bay structure. The addition of VNS as well as the utilization of a relaxed flexible bay structure for layout representation has proven to be effective as it produced the better results than those generated in most of the previous related works \cite{Garcia-Hernandez2020}.

% Continue briefly discussing about CRO and adding papers here. Do not forget about the salp algorithm.
