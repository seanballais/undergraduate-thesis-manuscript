\prefacesection{Abstract}

The unequal area static facility layout problem (UA-SFLP) deals with arranging a set of buildings of varying sizes in a region for a long period of time based on certain objectives. This problem is well-researched, with most researches solving instances of the problem, and the general facility layout problem, using traditional algorithms such as genetic algorithms, simulated annealing, and particle swarm optimization. However, newer algorithms have been introduced and may produce better solutions than previous studies. In this study, we are using the grey wolf optimization algorithm to solve the UA-SFLP. We have modified the algorithm in order for it to produce feasible solutions to the problem. We conducted experiments that vary the value of the $c$ parameter, and discovered that its value impacts the performance of our approach, and the appropriate value for the parameter is correlated with the size of the bounding region. We also compared our GWO approach against a hybrid GA approach and a PSO approach. We have discovered through our results that the hybrid GA approach produces the best solutions on average but scales poorly when the number of buildings increase, with PSO producing the worst solutions on average but taking the least amount of time. Our GWO approach produced the second best solutions on average, and was found to scale better than the hybrid GA approach. Hence, our approach provides a balance between speed and solution quality. Future studies can be done to improve the performance GWO algorithm in solving the facility layout problem.