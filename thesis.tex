% ------------------------------------------------------------------------
% MS Thesis of Juan de la Cruz
% Date Defended: May 31, 2000
% ------------------------------------------------------------------------
\documentclass[letterpaper,12pt]{report}
\usepackage[centertags]{amsmath}
\usepackage{amsfonts}
\usepackage{amssymb}
\usepackage{amsthm}
\usepackage{amsmath}
\usepackage{newlfont}
\usepackage{epsfig}
\usepackage{UPCSTHESIS} %This loads the UP Computer Science Thesis Package
\usepackage{UPCSINC1}
\usepackage[active]{LTX}
\usepackage{subfig}
\usepackage{listings}
\usepackage{array}
\usepackage{graphicx}
\usepackage{float}
\usepackage{url}

\usepackage{fancyhdr}

\fancyhf{} % clear all header and footer fields
\fancyhead[R]{\thepage}

\renewcommand{\headrulewidth}{0pt}
\renewcommand{\footrulewidth}{0pt}

% Redefining plain style which is automatically applied to chapters (including Bibliography)

\fancypagestyle{plain}{%
\fancyhf{} % clear all header and footer fields
\fancyhead[R]{\thepage}
\renewcommand{\headrulewidth}{0pt}
\renewcommand{\footrulewidth}{0pt}}

\pagestyle{fancy} 

%% Define a new 'leo' style for the package that will use a smaller font.
\makeatletter
\def\url@leostyle{%
  \@ifundefined{selectfont}{\def\UrlFont{\sf}}{\def\UrlFont{\small\ttfamily}}}
\makeatother
%% Now actually use the newly defined style.
\urlstyle{leo}


\hfuzz2pt
\newlength{\defbaselineskip}
\setlength{\defbaselineskip}{\baselineskip}
\newcommand{\setlinespacing}[1]%
           {\setlength{\baselineskip}{#1 \defbaselineskip}}
\newcommand{\doublespacing}{\setlength{\baselineskip}%
                           {2.0 \defbaselineskip}}
\newcommand{\singlespacing}{\setlength{\baselineskip}{\defbaselineskip}}
% MATH -------------------------------------------------------------------
\newcommand{\A}{{\cal A}}
\newcommand{\h}{{\cal H}}
\newcommand{\s}{{\cal S}}
\newcommand{\W}{{\cal W}}
\newcommand{\BH}{\mathbf B(\cal H)}
\newcommand{\KH}{\cal  K(\cal H)}
\newcommand{\Real}{\mathbb R}
\newcommand{\Complex}{\mathbb C}
\newcommand{\Field}{\mathbb F}
\newcommand{\RPlus}{[0,\infty)}
\newcommand{\norm}[1]{\left\Vert#1\right\Vert}
\newcommand{\essnorm}[1]{\norm{#1}_{\text{\rm\normalshape ess}}}
\newcommand{\abs}[1]{\left\vert#1\right\vert}
\newcommand{\set}[1]{\left\{#1\right\}}
\newcommand{\seq}[1]{\left<#1\right>}
\newcommand{\eps}{\varepsilon}
\newcommand{\To}{\longrightarrow}
\newcommand{\RE}{\operatorname{Re}}
\newcommand{\IM}{\operatorname{Im}}
\newcommand{\Poly}{{\cal{P}}(E)}
\newcommand{\EssD}{{\cal{D}}}
% THEOREMS ---------------------------------------------------------------
\theoremstyle{plain}
\newtheorem{thm}{Theorem}[section]
\newtheorem{cor}[thm]{Corollary}
\newtheorem{lem}[thm]{Lemma}
\newtheorem{prop}[thm]{Proposition}
\theoremstyle{definition}
\newtheorem{defn}{Definition}[section]
\theoremstyle{remark}
\newtheorem{rem}{Remark}[section]
\numberwithin{equation}{section}
\renewcommand{\theequation}{\thesection.\arabic{equation}}
\setlength{\tclineskip}{1.05\baselineskip}

\begin{document}

%\nobib  			%toggles bibliography control
%\draft				%toggles draft printing
%\nofront			%toggles nofront printing

\permissionfalse

%\nolistoftables	%toggles table control
%\nolistoffigures	toggles figures

% ------------------------------------------------------------------------

\title{A Hybrid Approach to University Course Timetabling Using Reinforcement Learning and Genetic Algorithm}

\bs
\author{Sean Francis N. Ballais}
\degreeinitial{BS Computer Science}
\major{Computer Science}
\adviser{Prof. Victor M. Romero II}
\dean{Dr. Virgildo Sabalo}
\submityear{Month Year}
\studentid{2015-04562}

% ------------------------------------------------------------------------
{
%\WinEdt{?0000} % Don't bother with over/under-full boxes
%\beforepreface
%\WinEdt{?1111} % Process All Errors from Here on
\forgeterr{?0000} % Don't bother with over/under-full boxes
\beforepreface
\forgeterr{?1111} % Process All Errors from Here on
}

{
\typeout{Acknowledgements}
\prefacesection{Acknowledgements}

In 1623, an English poet by the name of John Donne has written an essay containing the widely quoted excerpt, "No man is an island.". These words have proven to be true throughout history and manifests itself in the human experience in general. This work of ours is obviously not an exception to this, and has benefited from the wisdom and inputs of different people from different walks of life.

Due gratitude and acknowledgments are dedicated to my thesis advisor, Prof. Victor M. Romero II, for his guidance, encouragements, and wisdom that enabled me to complete this thesis.

Special thanks to my friends in "Mga Loyal sa Komsai", Kenneth Lanante, Bea Santiago, Babes Ngoho, Aerol Nebril, and Cylwyn Creer for their comaraderie and general support in accomplishing this work, to Kate Young for semi-regularly asking me for the progress of my thesis, and to my friends in the Hangouts Int'l Discord server, notably, Shann Ripalda for providing inputs that helped me improve this work, and Denz Merin, Julian Yu, Julyanna Huang, Michael Omisol, and Elyzah Parcon for their \textit{encouragements} and support. Special thanks to Rina Falculan and Charles Webb for helping me with practicing for the defense.

Lastly, I would like to thank God, my family, my godparent, Ramil Perez, and my friends for supporting and aiding me in accomplishing this thesis study.

}
% ------------------------------------------------------------------------
{
\typeout{Abstract}
\prefacesection{Abstract}

The unequal area static facility layout problem (UA-SFLP) deals with arranging a set of buildings of varying sizes in a region for a long period of time based on certain objectives. This problem is well-researched, with most researches solving instances of the problem, and the general facility layout problem, using traditional algorithms such as genetic algorithms, simulated annealing, and particle swarm optimization. However, newer algorithms have been introduced and may produce better solutions than previous studies. In this study, we are using the grey wolf optimization algorithm to solve the UA-SFLP. We have modified the algorithm in order for it to produce feasible solutions to the problem. We conducted experiments that vary the value of the $c$ parameter, and discovered that its value impacts the performance of our approach, and the appropriate value for the parameter is correlated with the size of the bounding region. We also compared our GWO approach against a hybrid GA approach and a PSO approach. We have discovered through our results that the hybrid GA approach produces the best solutions on average but scales poorly when the number of buildings increase, with PSO producing the worst solutions on average but taking the least amount of time. Our GWO approach produced the second best solutions on average, and was found to scale better than the hybrid GA approach. Hence, our approach provides a balance between speed and solution quality. Future studies can be done to improve the performance GWO algorithm in solving the facility layout problem.
}
\tableofcontentspage
% ------------------------------------------------------------------------
%\setcounter{page}{1}
%\tableofcontents
% ------------------------------------------------------------------------

% ------------------------------------------------------------------------
\afterpreface
\def\baselinestretch{1}
\setlinespacing{1.66}
% ------------------------------------------------------------------------
{
%\typeout{Introduction}
%\include{introd}
}
% ------------------------------------------------------------------------
\setlinespacing{1.66}
% ------------------------------------------------------------------------

\chapter{Introduction} \label{sec:intro}

In computer graphics, most images are typically stored as a sequence of dots in a rectangular grid (see Fig. \ref{fig:raster-images-upclose}). Each dot is called a pixel, a small part of an image that holds one specific colour, Photographs, also called natural images\cite{hoshyari2018perceptiondriven}, are one of, if not the most, common images that are stored in this manner. Many digital forms of art or any graphics work, such as paintings, posters, and icons, are also stored the same. Digital images stored in this manner are called \textit{raster images}. These images are stored in various image formats. The most commonly used formats are JPEG, GIF, BMP, TIFF, and PNG. Each have their pros and cons, from quality of the resulting image to the file size. Nevertheless, they all still accomplish the task of holding raster image data. Everything you see in the displays of devices such as laptops and mobile devices is a raster image. Computer displays are collections of pixels, in the common definition of a dot on the screen, which computers map images to to be able display them. This is the reason why \textbf{all} \textit{displayed} images are raster images.

\begin{figure}[h]
	\centering
	\includegraphics[scale=1.0]{images/chap01-introduction/raster-images-upclose.png}
	\caption{When zoomed in enough, each individual pixel of a raster image is visible. Meme image obtained from \protect\url{http://thesismemes.tumblr.com/post/73483120281}}.
	\label{fig:raster-images-upclose}
\end{figure}

A positive aspect of raster images is their simplicity. As mentioned earlier, raster images consists of a grid of pixels (also called a pixel matrix in other literature \cite{realtimevectorizationgpu}). This pixel grid can simply be assigned a combination of colour values to create an image. As such, working with raster images can be analogous to painting in the real world \cite{rastervsvector}. Given the right combinations of colours, we can produce natural images, i.e. photographs \cite{hoshyari2018perceptiondriven}. Intuitively, this means that we can store fine details in a raster image \cite{optimizedgradientmeshes}. This is in contrast to \textit{vector images}, which use a series of points and mathematical calculations to form lines and shapes. Vector images are unable to display lush colour depth and keep granularity, as found in raster images, as they use solid colours or gradients  \cite{rastervsvector}\cite{rastervsvectorgraphics}. There are studies that have been conducted in improving and utilizing \textit{gradient meshes}, a vector graphics primitive that allows for intricate colour gradients in regular quadrilateral meshes first introduced by Adobe Illustrator, to produce photorealistic vector images. However, as noted in the paper by Jian, S, Liang, L., Wen, F., and Shum, H., simple gradient meshes are insufficient to keep the fine details of images \cite{barendrecht2018locally}\cite{optimizedgradientmeshes}. It is also important to mention that vector graphics, despite represented as mathematical calculations, are still converted to raster format in a process called \textit{rasterization} for it to be displayed on-screen, since many modern screens are raster displays \cite{howdovectorgraphicswork}.

\section{The Problems of Raster Graphics}
With all the pros raster graphics have, it does not mean raster graphics are not without their caveats. Raster graphics have their own disadvantages which could affect the image quality and their use.

Raster graphics are \textbf{resolution-dependent}. This simply means that raster images are in their highest quality in the resolution they are initially created in and attempting to scale it up will gradually degrade the quality of the image as the image size or resolution grows larger. Rasters only have a finite number of pixels. Increasing the size of an image (also called upsampling \cite{hoshyari2018perceptiondriven}) would entail moving the individual pixels into different locations depending on the scaling factor, the distance the individual pixels will be moved to horizontally and vertically. Upsampling will create empty pixels in between the shifted pixels when there is nothing done to substitute the empty pixels. This will create an unusable image \cite{resizingimages}. We can utilize interpolation to fill these empty pixels with colour. The approximate colour and intensity values of these empty pixels are calculated based on the values of surrounding pixels, typically the shifted pixels. However, the specifics are dependent on the interpolation algorithm used in upscaling the images. Commonly used interpolation methods for resizing images include Nearest-Neighbour, Linear Interpolation, and Cubic Interpolation \cite{interpolationtechniquessurvey}, which most, if not all, are readily available in popular image editing applications, such as Adobe Photoshop \cite{photoshopinterpolationmethods} and GIMP \cite{gimpinterpolationmethods}. But, the results produced by these interpolation algorithms typically suffer from blurring of sharp edges and ringing artifacts due to the fact that the algorithms do not assume anything about the data \cite{depixelizingpixelart}. Talk about the non-classical interpolation and AI Gigapixel.

\begin{figure}[h]
	\centering
	\includegraphics[scale=1.0]{images/chap01-introduction/raster-images-scaled.png}
	\caption{Different interpolation algorithms will produce different results. As seen in the image, the quality will also differ, from an image looking blocky to an image looking blurry. Cat meme image obtained from \protect\url{https://www.hercampus.com/school/uwindsor/school-thoughts-told-grumpy-cat-memes}}.
	\label{fig:raster-images-scaled}
\end{figure}

% Include the reason where you say that vector images are still rasterized before being displayed.

% Note: Maybe add here why raster images are simple in the context of image processing once we learn more about image processing.
\chapter{Review of Related Literature} \label{sec:rrl-title}
Facility layout problems is a well-known problem with many decades of research behind it. The benefits it provides in many situations, such as in factories and office spaces, while being a rather challenging problem to solve motivates the ongoing research for it. As mentioned in the previous chapter, facility layout problems are known to be NP-Hard. This makes the use of metaheuristics popular when it comes to solving FLPs. Based on our review, genetic algorithms are the most popular form of metaheuristics that have been used to solve various forms of facility layout problems  \cite{Hosseini-Nasab2018}. Newer forms of metaheuristics, such as variable neighbourhood search, are being applied to FLPs more and more. Despite the popularity of the use of metaheuristics, exact methods have also been adapted to facility layout problems but not as widely used as metaheuristics. In this chapter, we will be reviewing many of the previous works available within the literature of facility layout problems. This will provide us with a good enough understanding about the current state of research around facility layout problems and allow us to find where this work may fit in the ocean of previous works. Due to the vastness of the field, we will only be focusing particularly on unequal area facility layout problems in this chapter. We will also be mentioning works that have been used in other types of facility layout problems. Additionally, most of the works mentioned here will be from the past 10 years.

\section{Exact Methods}
The survey of Hosseini-Nasab et al. (2017) showed that metaheuristics are popular approaches in solving facility layout problems. However, exact methods have also been used to solve FLPs \cite{Hosseini-Nasab2018}. Exact methods are algorithms that are able to find the optimal solution for an optimization problem \cite{Dumitrescu2003}. However, they are not well-suited for large NP-Hard problems, such as the facility layout problem, due to the amount of time they will require in solving them. This does not mean that they are never used to solve NP-Hard problems. Small instances of those problems and multi-objective combinatorial optimization problems may still be solved by exact methods \cite{Jourdan2009}\cite{Ehrgott2016}. Numerous techniques may be used to improve the speed of these methods \cite{Woeginger2003}.

In 2006, Amaral, A. (2006) proposed a new mixed-integer linear programming model for the single-row facility layout problem (SRFLP). The author's model provided fewer continuous variables compared to the model he was comparing the new model against. Both models were solved with CPLEX 8.0 using a branch-and-bound method. It was found that the new model performed better than the previous one \cite{Amaral2006}. The branch-and-bound method solves optimization problems by exploring the entire search space \cite{Datta2020}. It produces a search tree of subproblems and solves a subproblem on every iteration. This is repeated until no subproblems remain \cite{Morrison2016}. Solimanpur, M., and Jafari, A. (2008) developed a branch-and-bound algorithm to solve an instance of the facility layout problem. Their method managed to find good solutions for small and medium problem instances. However, in line with the expectations of the performance of exact methods, they found that it is inefficient for large-sized problem instances \cite{Solimanpur2008}.

\section{Works Using Metaheuristics}
Exact methods have been used to solve many different problems, particularly those problems with known optimal solutions. Unfortunately, not all problems have known best solutions, and looking for them will take a reasonably long time to find \cite{Glover2015}. Facility layout problems are under these types of problems. As such, metaheuristics are popular when it comes to solving FLPs \cite{Drira2007}. This is further supported by the survey of Hosseini-Nasab et al. (2018) \cite{Hosseini-Nasab2018}, where it is found that most papers they have surveyed used a metaheuristic to solve FLPs.

\subsection{Genetic Algorithms}
There are various forms of metaheuristics. Common of which is the genetic algorithm \cite{Hosseini-Nasab2018}. Genetic algorithm is a form of evolutionary-based metaheuristic. It works by breeding a generation of individuals from pairs of parents (through a crossover operation). The children produced from the breedings may undergo mutation to improve the diversity of the population and help find better solutions. This process is repeated until the algorithm reaches a certain number of generations, or a stopping condition has been met \cite{Luke2013Metaheuristics}. We allocated a section for discussing works that utilize genetic algorithms for facility layout problems due to its popularity in terms of use within the field \cite{Hosseini-Nasab2018}.

\subsubsection{Pure Genetic Algorithms}
In literature, to our knowledge, there is no term called pure genetic algorithms. However, for the sake of ease of differentiation, we will be referring to the genetic algorithms in prior related works without any combination with other optimization algorithms as "pure". Genetic algorithms that have been combined with other algorithms will be called "hybridized" genetic algorithms. These algorithms are discussed in the next subsection.

One work that uses pure genetic algorithms is that of Hasda et al. (2016). In their work, they attempted to solve the static unequal-area facility layout problem using a modification of the genetic algorithm. They have also used elitism in their modification. Their variation of the genetic algorithm still includes the traditional operators (despite being named differently in their paper), but with the inclusion of a rotation operator. The rotation operator is simply an operator that rotates a facility of a solution. It is similar to that of the mutation operator in that it only runs when a certain rotation probability is reached, and this probability is user-defined and is recommended to be of a small value. Their method has proven to be slightly better than the works they compared it to \cite{Hasda2017}. Another paper, proposed by Besbes et al. (2020) \cite{Besbes2020}, also modifies the genetic algorithm for use with the facility layout problem. In most papers dealing with facility layout problems, the distance between the geometric centers of facilities considered in the objective function are computed using Euclidean or rectilinear distance. Besbes et al. changed this by using the A* algorithm to compute the distance more realistically and consider obstacles. This use of A* search has produced better solutions than when using the other two distance computation functions. Fernando, J., and Resende, M. (2015) modified the genetic algorithm to change the parent selection behaviour. Their method has the population partitioned into the elite individuals (those with the best fitness, and they are a small number) and non-elite individuals. During breeding, one parent will be from the elite partition and the other from the non-elite partition. The facilities are also arranged using maximal spaces and placing facilities in those spaces in such a way that it is as close to the rest of the facilities as possible. Their scheme created the better solutions for many of the datasets they applied it to compared to previous studies \cite{Fernando2015}. Placing facilities within a site layout, especially when considering multiple time periods, is another problem that may be considered to be under facility layout problems. Farmakis, P, and Chassiakos, A. (2018) developed a genetic algorithm to minimize the resource transportation costs between facilities or between facilities and work fields, and facility construction and relocation costs in a construction site considering changing requirements over time (an instance of the dynamic facility layout problem). According to the authors, their method produces "rational solutions", and the consideration for the changing demands over time produced a more effective layout than a static layout \cite{Farmakis2018}. Similar to Farmakis, P, and Chassiakos, A. (2018), Peng et al. (2018) are also dealing with an instance of the dynamic facility layout problem. In their problem instance, they are also considering transport devices, such as conveyers and tow trains. A Monte Carlo simulation method has been used to generate scenarios, due to demand uncertainty. The crossover and mutation probability of an offspring in their genetic algorithm implementation is determined by its fitness relative to the fitness of the other individuals. The authors compared their genetic algorithm to particle swarm optimization and found that it produces the better results in all but two experiment data sets \cite{Peng2018}. A genetic algorithm for facility layout problems can produce subjectively more desirable results when interactively given feedback from a decision maker. This idea is being utilized in the work of Garcia-Hernandez et al. (2013). In their work, they used two genetic algorithms to find an suboptimal layout. The first genetic algorithm is non-interactive and traditional, and only optimizes for material flow. The second genetic algorithm now takes into account the subjective evaluation by the decision maker, along with the material flow cost. This second algorithm is also partly based on NSGA-II, and only stops when the decision maker is satisfied with the results. The authors applied their genetic algorithm to two real-world cases, and found that their approach managed to capture the preferences of the decision maker and good solutions were generated in a reasonable number of iterations \cite{Garcia-Hernandez2013}.

Genetic algorithms may also be applied to non-traditional configurations of FLPs. Barriga et al. (2014) used genetic algorithms to produce the best layout of buildings in a Protoss base in classic StarCraft. The fitness of a base's configuration is based on the health of its army, workers, and pylons \cite{Barriga2014}. % Talk more about slight modifications of GAs for FLPs.

\subsubsection{Hybridized Genetic Algorithms}
There are many other papers that modified genetic algorithms to solve facility layout problems. However, many of them did not only slightly modify the genetic algorithm. Rather, they also combined it with another algorithm, usually a local search algorithm. This resulted in \textbf{hybridized algorithms} that better exploited the search space of the solution produced by the genetic algorithm.

Asl et al. (2015) \cite{Asl2015} and Asl, A. and Wong, K. (2015) \cite{Asl2015a} produced works that hybridized genetic algorithms with local search algorithms. The local search algorithms they used moved buildings in such a way that the solution generated is better than the original solution. The local search method used in the work of Asl, A. and Wong, K. (2015) moves a building in different directions. This movement was performed for each building. The best new layout produced will replace the original solution if it is better than the original solution. The paper of Asl et al. (2015) also uses this local search method, and it is referred to as Local Search 1. The same paper also uses another local search method called Local Search 2. It works the same as Local Search 1. However, it moves two buildings at the same time. Both papers also utilize a swapping method in their genetic algorithms, which swaps facility positions to find a better arrangement.

Genetic algorithms has also been hybridized with variable neighbourhood search. Variable neighbourhood search (VNS) is a relatively recent local search algorithm introduced in 1997 by Mladenovic, N. and Hansen, P.. The algorithm utilizes and moves through a set of neighbourhood structures to find the local optimum \cite{Hansen2018}, performed within the three phases of its main step \cite{Hansen2017}. In the paper of Uddin, M. (2015), genetic algorithm was used in conjunction with VNS. The author used the combined algorithm of GA-VNS to solve a problem instance of the dynamic facility layout problem. In each iteration, a percentage of the current population is subjected to breeding using a genetic algorithm, while the rest are optimized using VNS. This hybridized algorithm produced the same results with half of the datasets it was tested to compared to some of the previous works, while performing the best in two of the datasets, the worst in one, and the second best in the last \cite{Uddin2015}.

Variable neighbourhood search is not the only local search algorithm that has been hybridized with genetic algorithms for solving facility layout problems. Simulated annealing has also been combined wth genetic algorithms. Simulated annealing (SA) is a local search algorithm inspired by annealing, which is a process that finds the low energy state of a metal by melting and then cooling it slowly \cite{Lai1997}. The main idea behind SA is to slightly modify a solution to form a new solution, and that solution is only accepted when it is better than the older solution or with a certain probability when it is worse \cite{Dueck1993}. A hybrid of genetic algorithms and simulated annealing was used in the work of Pourvaziri, B., and Naderi, B. (2014) in order to solve another instance of the dynamic facility layout problem. Contrary to traditional genetic algorithms, their work utilizes multiple populations to find the solutions. Each population is involved independently of the other populations. These populations are then coalesced into a main population, which is now composed of the best individuals of the initial populations, after a pre-determined number of generations. The main population is then evolved, and the most fit solution from the population is further optimized using simulated annealing. This evolution and local search optimization is repeated until a stopping condition is met \cite{Pourvaziri2014}. 

\subsection{Non-Genetic Algorithms}
Genetic algorithms are not the only metaheuristics that have been used to solve facility layout problems. Metaheuristics, such as particle swarm optimization, simulated annealing, and even relatively recent algorithms such as fireworks algorithms, have found application in facility layout problems.

Simulated annealing without hybridization with genetic algorithm have been used in FLPs. The work of Turgay, S. (2018) is one such example. Turgay, S. sought to solve an instance of the unequal-area facility layout problem with consideration for multiple objectives. Each objective is given a weight, determining its impact, in the mathematical model of his work. The values of the weights of each objective are obtained using Shannon's entropy rule. Based on experiments, the SA implementation is capable of producing usable layouts. However, its performance was not compared against other metaheuristics \cite{Turgay2018}. McKendall et al. (2006) also developed a simulated annealing implementation that they used for the dynamic facility layout problem. They modified the simulating annealing algorithm to integrate a look-ahead/look-back strategy into the algorithm from the work of McKendall, A. and Shang, J. (2006) \cite{McKendall2006Ant}. They compared their modified SA with the traditional SA and a number of other algorithms, including a genetic algorithm implementation and a dynamic programming approach, through a set of experimental data. They discovered that their modified simulated annealing is effective in solving the dynamic facility layout problem, producing the best results in most of the problems in that large experimental dataset \cite{McKendall2006}. Another paper that used simulated annealing is that of Hosseini-Nasab, H., and Mobasheri, F. (2013). Their simulated annealing implementation utilized two mutation operators in generating neighbourhood solution. They added this modification to allow the algorithm to escape from local optimum, and allow for distinctions between solutions. They compared their work against GAMS, a modelling and optimization software \cite{GAMSSoftware}. Based on experimental results, their method can produce results significantly faster than GAMS, and can produce the best opttmum solution is mostly better or equal to the best optimum solution produced by GAMS \cite{Nasab2013}. It should be noted, however, that it may be better for them to have performed more runs for each method, compared to the five runs for their simulated annealing and one run for GAMS, to ensure that the results are statistically significant. Nevertheless, their work is still useful. Sahin, R. (2011) also developed a simulated annealing implementation for the facility layout problem. No modification to the simulated annealing algorithm was introduced. However, the mathematical model it is optimizing for considers the total material handling cost and the total closeness rating score. The author compared his work to two previous works, and found that the proposed SA approach produced same or better results than the previous works \cite{Sahin2011}.

Genetic algorithms are a population-based optimization algorithm that have seen wide use in solving facility layout problems. But, it is not the only population-based optimization algorithm that has been used in facility layout problems. Particle swarm optimization (PSO) is an optimization algorithm that has seen use in FLPs as well. Particle swarm optimization is an optimization algorithms inspired by the social behaviour of birds in finding safe locations in which to land on. This optimization algorithm utilizes particles that perform search in a search space but keep note of the best global solution and personal best solution found so far, to which they will tend to move towards to, with parameter settings determining the movement behaviour \cite{SeixasGomesdeAlmeida2019}. Derakhshan Asl, A. and Wong, K. Y. (2017) are two researchers that have utilized particle swarm optimization in their work. In their work, they developed a modified particle swarm optimization algorithm that solves the static and dynamic versions of an instance of the unequal-area facility layout problem. They applied local search and swapping methods into PSO to improve the quality of solutions, and prevent local optima for both version of UA-FLP. They compared this algorithm to a number of previous works to which they have determined that it produces better results than the previous works \cite{DerakhshanAsl2017}. Liu et al. (2018) developed a particle swarm optimization algorithm that optimizes a multi-objective function. Their algorithm also utilized objective space division method and a mutation operation and local search method to prevent facility overlaps. The algorithm was compared to previous works and was found to produce the best results in most of the experimental data set \cite{Liu2018}.

The metaheuristics simulated annealing, particle swarm optimization, and genetic algorithms first appeared decades ago. Simulated annealing was first proposed in 1983 \cite{Kirkpatrick1983}. while the genetic algorithm and particle swarm optimization were proposed in the 1990s \cite{Katoch2021}\cite{Kennedy1995}. Between the time the aforementioned algorithms were proposed and the time of writing of this paper, new optimization algorithms were proposed. Among these optimization algorithms is the coral reef optimization algorithm. Coral reef optimization (CRO) is based off of the formation and reproduction processes of coral reefs. In CRO, solutions are located in a grid initially partially populated by corals. A coral represents a solution, and the health of a coral represents its fitness. Corals in the grid sexually reproduce to produce larvae that are released into the water. Larvae settle in a grid depending on its health and the state of the grid cell they are attempting to settle in. Some corals are then made to asexually reproduce and occupy different parts of the grid with the same mechanism as larvae settling mentioned in the previous sentence. Some corals are also made to die to open up space for the next generation. These steps are performed until a stopping condition is met \cite{Salcedo-Sanz2014}. Garcia-Hernandez et al. (2019) utilized CRO in solving an instance of the facility layout problem and with the use flexible bay structures. No major modifications to CRO were used in their work. In their experimentations, they compared their CRO implementation with previous works, including those that do not use flexible bay structures as their layout representations. When comparing only against implementations with a flexible bay structure representation, their work produces the best results for most of the 17 cases. However, when considering a slicing tree structure layout as well, it only improves results for 7 of the cases \cite{Garcia-Hernandez2019}. The next year of the publication of their work, another paper combined coral reefs optimization with variable neighbourhood search. In this paper by Garcia-Hernandez (2020), the CRO algorithm remained as the original algorithm, but the larvae settling phase of the algorithm has been combined with VNS to further improve the larva/solution that is settling. Note that VNS is only ran when the larva is assured to occupy the grid cell it is settling towards. Their work also uses a relaxed flexible bay structure. The addition of VNS as well as the utilization of a relaxed flexible bay structure for layout representation has proven to be effective as it produced the better results than those generated in most of the previous related works \cite{Garcia-Hernandez2020}.

% Continue briefly discussing about CRO and adding papers here. Do not forget about the salp algorithm.

\chapter{Statement of the Problem} 
\label{sec:ResearchQ}

There have been numerous works that tackles image vectorization for various inputs. Most of these works have been targeted at natural imagery where many utilize image segmentation. These works do not necessarily be appropriate for semi-structured imagery. As such, many works have also sprung up to tackle these types of images. However, many of the works such as those by Hoshyari, S., et. al \cite{hoshyari2018perceptiondriven}, and Kopf, J., and Lischinski, D. \cite{depixelizingpixelart}, only handle clean, quantized non-anti-aliased images. As such, there has been no work that has been targeted at dealing with anti-aliased semi-structured input with speeds nearing those tested by Hoshyari, S., et. al.. Applying the proposed method by Hoshyari, S., et. al., as the authors have stated, have an impact on the resulting vectorization. Additionally, their work becomes computationally expensive on larger inputs. The work by Xie, G., Sun, X., Tong, X., and Nowrouzezahrai, D., may be applied to semi-structured images \cite{hierarchicaldiffusioncurves}. However, their method is computationally expensive and, as extrapolated from their experimental setup, require expensive hardware (the Nvidia Quadro 6000 was the most expensive of the components in their setup).

There are numerous raster semi-structured images that are anti-aliased, especially those that have a larger resolution. These images are used in posters, logos, and graphic designs. Vectorization of those images will improve their quality and allow them to support higher resolutions without any degradation of the quality.

\chapter{Objectives}

This study is primarily aims to apply deep learning via artificial neural networks to the problem of vectorizing semi-structured imagery. However, there are still some key objectives that this study seeks to accomplish:

\begin{enumerate}
	\item To develop an approach that considers Gestalt psychology.
	
	\item To evaluate the effectiveness of using deep learning for image vectorization.
	
	\item To evaluate the accuracy of results obtained from the proposed approach to the target raster inputs.
	
	\item To evaluate and compare the results of the proposed approach to that of previous semi-structured image vectorization methods.
	
	\item To evaluate and compare the speed of the proposed approach compared to previous semi-structured image vectorization methods.
\end{enumerate}
\chapter{Methodology}
The methodology used in this research uses a modification of the classical Grey Wolf Optimization algorithm first introduced by Mirjalili, S., Mirjalili, S., and Lewis, A. in 2014 \cite{Mirjalili2014}. As we will be discussing in this chapter, we have determined that using classical GWO as is does not result in usable solutions for the instance of facility layout problem we are solving. Hence, the necessity for the modification.

In this chapter, we will first discuss about the mathematical model of the problem being solved. Later, we will be delving into the inner workings of the solution representation, the algorithm (including the justification for the modification), and then the technologies that were used in implementing the approach.

\section{Mathematical Model}
The goal of any metaheuristic, like what is being proposed in this paper, is to optimize a certain objective function. As mentioned in the first chapter, in facility layout problems, we minimize the following function:

$$
\text{min} F = \sum_{i=1}^{n}\sum_{j=1}^{n}c_{ij}f_{ij}d_{ij}
$$

For the problem we are solving in this paper, we are optimizing the following equation that is not only a slight modification of the basic mathematical model for FLPs, but also adds penalties to solutions that are infeasible, no matter the degree of infeasiblity.

\begin{align*}
	\text{min }F &= \sum_{i=1}^{\left | B \right |}\sum_{j=i + 1}^{\left | B \right |}c_{ij}d_{ij} \\
	& + \sum_{i=1}^{\left | B \right |}\sum_{j=i + 1}^{\left | B \right |} \left ( P_{B}\frac{A_{0}(i, j)}{\text{min}(w_{i}h_{i}, w_{j}h_{j})} + P_{B} \right ) \cdot \alpha_{0}(i, j) \\
	& + \sum_{i=1}^{\left | B \right |}\left( P_{R}\frac{w_{i}h_{i} - A_{1}(i)}{w_{i}h_{i}} + P_{R} \right) \cdot \alpha_{1}(i)
\end{align*}

where:

\begin{table}[h!]
	\centering
	\begin{tabular}{| l | p{10cm} |}
	\hline
	$x_{i}$  & top-left $x$ coordinate of building $i$ \\
	\hline
	$y_{i}$  & top-left $y$ coordinate of building $i$ \\
	\hline
	$w_{i}$  & width of building $i$ \\
	\hline
	$h_{i}$  & height of building $i$ \\
	\hline
	$R_{x}$  & top-left $x$ coordinate of the bounding region \\
	\hline
	$R_{y}$  & top-left $y$ coordinate of the bounding region \\
	\hline
	$R_{w}$  & width of the bounding region \\
	\hline
	$R_{h}$  & height of the bounding region \\
	\hline
	$c_{ij}$ & flow rate from building $i$ to building $j$ \\
	\hline
	$d_{ij}$ & distance from the center of building $i$ to the
	           center of building $j$ \\
	\hline
	$P_{B}$  & penalty value for building intersection \\
	\hline
	$P_{T}$  & penalty value for any building going out of
	           bounds, even with a portion of a building \\
	\hline
	\end{tabular}
\end{table}

We elected to remove the flow rate from the basic formulation of the model that was discussed earlier in Equation \ref{mm-equation-mhc}. 

\begin{equation}\label{mm-equation-mhc}
	\sum_{i=1}^{\left | B \right |}\sum_{j=i + 1}^{\left | B \right |}c_{ij}d_{ij}
\end{equation}

This is because we can consider flow rate as simply part of the cost. In the original formulation, we were considering it from a material handling cost perspective, which requires having both a cost and flow rate variable. However, in a general problem, we can consider cost to also include the frequency of movement from one facility to another, which is essentially the flow rate. As such, we can merge cost and flow rate into one variable.

The mathematical model allows for infeasible solutions to allow for better solutions in the long run. To follow this specification, the model includes expressions that penalizes solutions that meet any of the following conditions: (1) at least one building is intersecting with another building, and (2) a building, either in whole or in part, is outside the bounding area.

\begin{equation}\label{mm-equation-intersection}
	\sum_{i=1}^{\left | B \right |}\sum_{j=i + 1}^{\left | B \right |} \left ( P_{B}\frac{A_{0}(i, j)}{\text{min}(w_{i}h_{i}, w_{j}h_{j})} + P_{B} \right ) \cdot \alpha_{0}(i, j)
\end{equation}

Equation \ref{mm-equation-intersection} is the expression that applies a penalty to solutions that meet the first condition. Notice that it has the functions $A_{0}(i, j)$ and $\alpha_{0}(i, j)$. They are defined by the following:

\begin{align}
	A_{0}(i, j) &= I_{L}(x_{i}, x_{j}, w_{i}, w_{j})
	               \cdot I_{L}(y_{i}, y_{j}, h_{i}, h_{j}) \\
	I_{L}(x_{1}, x_{2}, l_{1}, l_{2}) &= \text{max}(0, \text{min}(x_{1} + l_{1}, x_{2} + l_{2}) - \text{max}(x_{1}, x_{2})) \\
    \alpha_{0}(i, j) &=
    \left\{\begin{matrix}
    	1 & \text{if } A_{0}(i, j) > 0 \\ 
    	0 & \text{otherwise}
    \end{matrix}\right.
\end{align}

$A_{0}(i, j)$ simply gets the area of intersection of buildings $i$ and $j$. This is achieved by the use of $I_{L}(x_{1}, x_{2}, l_{1}, l_{2})$, which computes the length or width of an intersection of buildings.

In the equation, for every pair of buildings that intersect, we apply a penalty that is the percentage of the area of the smallest building by area that is intersecting with the other building multiplied by the penalty value for building intersection. This will allow for rewarding the algorithm for moving the buildings towards non-intersection. The same penalty value is also added to ensure that the algorithm prioritizes removing intersections over reducing the distance between the centers of the buildings. $\alpha_{0}(i, j)$ ensures that the penalty is only applied to pairs of buildings that intersect with one another.

\begin{equation}\label{mm-equation-oob}
	\sum_{i=1}^{\left | B \right |}\left( P_{R}\frac{w_{i}h_{i} - A_{1}(i)}{w_{i}h_{i}} + P_{R} \right) \cdot \alpha_{1}(i)
\end{equation}

The other part of the mathematical model, Equation \ref{mm-equation-oob}, works in a similar principle as Equation \ref{mm-equation-intersection}. This equation applies a penalty value when the second condition of infeasibility is met. Like in \ref{mm-equation-intersection}, it has specific functions to help compute the penalty. They are defined as:

\begin{align}
	A_{1}(i) &= I_{L}(x_{i}, R_{x}, w_{i}, R_{w})
	\cdot I_{L}(y_{i}, R_{y}, h_{i}, R_{h}) \\
	\alpha_{1}(i) &=
	\left\{
	\begin{matrix}
		0 & \text{if} &
		\begin{aligned}
			R_{x} &\leq x_{i} &\leq R_{x} + R_{w} \\
			R_{x} &\leq x_{i} + w_{i} &\leq R_{x} + R_{w} \\
			R_{y} &\leq y_{i} &\leq R_{y} + R_{h} \\
			R_{y} &\leq y_{i} + h_{i} &\leq R_{y} + R_{h} \\
		\end{aligned} \\		
		1 & \text{otherwise}
	\end{matrix}\right.
\end{align}

$A_{1}(i)$ simply computes the area of intersection of the building and the bounding area. Now, since this only computes the intersection, we must subtract the intersection with the area of the building to get the area of the building that is outside of the bounding area. This is expressed by the numerator of the fractional expression in Equation \ref{mm-equation-oob}. Similar to Equation \ref{mm-equation-intersection}, the equation applies a penalty value that is the percentage of the area of the total building area that is outside the bounding region multiplied and then added by the penalty value. The addition is also to ensure that the algorithm gives more priority to removing out-of-bounds buildings. $\alpha_{1}(i)$ ensures that the penalty is only applied to buildings that are, in part or in whole, out of bounds.

\section{Solution Representation}
The solution is represented using a one-dimensional array of floating numbers. In the array, every group of three consecutive elements are considered to be the x and y positions, and angle, respectively, of one building. While the x and y positions are allowed to be of any value, the angle value is restricted to only $0^{\circ}$ and $90^{\circ}$. A visualization of the solution representation is shown by Figure \ref{solution-repr-viz}.

\begin{figure}
	\begin{center}
		\begin{tabular}{| c | c | c | c | c | c | c |}
			\hline
			$x_{0}$ & $y_{0}$ & $\angle_{0}$ & $\ddots$ & $x_{n - 1}$ & $y_{n - 1}$ & $\angle_{n - 1}$ \\
			\hline
		\end{tabular}
	\end{center}
	\caption{Visualization of the solution representation.}
	\label{solution-repr-viz}
\end{figure}

\section{The Algorithm}
In this paper, we are adapting the Grey Wolf Optimization algorithm into solving our instance of the facility layout problem. There have been no publicly available research that have previously used the metaheuristic in solving FLP, basing from our survey. This increases the significance of this paper. As mentioned earlier, the proposed algorithm requires modifications in order to produce feasible solutions. We will first be discussing the reasons why we require them, before proceeding to detailing the algorithm we are using for this research.

\subsection{The Problem with Classical GWO}
In classical GWO, the following equations are used:

\begin{align}
	\vec{X_{1}^{'}} &= \vec{X_{\alpha}}(t) - \vec{A_{\alpha}} \cdot \vec{D_{\alpha}} \label{gwo-x1-eqn} \\
	\vec{X_{2}^{'}} &= \vec{X_{\beta}}(t) - \vec{A_{\beta}} \cdot \vec{D_{\beta}} \\
	\vec{X_{3}^{'}} &= \vec{X_{\delta}}(t) - \vec{A_{\delta}} \cdot \vec{D_{\delta}} \\
	\vec{X}(t + 1)  &= \frac{\vec{X_{1}^{'}} + \vec{X_{2}^{'}} + \vec{X_{3}^{'}}}{3} \label{gwo-xt1-eqn}
\end{align}

where $\vec{X_{\alpha}}$, $\vec{X_{\beta}}$, and $\vec{X_{\delta}}$ represent the $\alpha$, $\beta$, and $\delta$ solutions \cite{Gupta2018}. $\vec{D}$ and $\vec{A}$ are defined as:

\begin{align*}
	\vec{D}        &= \left | \vec{C} \cdot \vec{X_{l}}(t)
	- \vec{X}(t) \right | \\
	\vec{C}        &= 2 \cdot \vec{r_{2}} \\
	\vec{A}        &= 2 \cdot \vec{a} \cdot \vec{r_{1}}
					  - \vec{a}                \\
\end{align*}

The aforementioned equations may be usable as is for other problems. However, as we have discovered through our experiments, using these equations results in solutions that are infeasible and where the buildings tend to move to the axes of an origin and the origin itself. One example solution with these characteristics is shown in Figure \ref{bad-solution-unmodified-gwo}, where we have set the origin of the buildings to the center of the bounding region.

\begin{figure}[h!]
	\centering
	\includegraphics{./images/chap05-methodology/bad-solution-unmodified-gwo.png}
	\caption{Flowchart detailing the algorithm.}
	\label{bad-solution-unmodified-gwo}
\end{figure}

As one may infer, using the equations above will require setting an origin point for the buildings. Not considering the affinity of building towards the axes, having the origin point at the center or in a certain location in the bounding region restricts the possible locations where the buildings can cluster around. This restriction prevents us from exploring the solution subspace where solutions are feasible but where the cluster point is not the origin. This lead us to solutions that are less ideal. Aside from requiring setting the origin point, buildings moving towards the axes also presents another problem. Basing from our experiments, it prevents us from producing feasible solutions.

This behaviour can be attributed to the formulas, $\vec{D} = \left | \vec{C} \cdot \vec{X_{p}}(t) - \vec{X}(t) \right |$ and $\vec{A} \cdot \vec{D}$. Remember that the latter is from $\vec{X_{l}^{'}} = \vec{X_{l}}(t) - \vec{A_{l}} \cdot \vec{D_{l}}$, where $l$ represents the alpha, beta, or delta solution. To understand why the aforementioned formulas contribute to the behaviour we have discussed earlier, we should understand what any of the aforementioned formulas mean. It is helpful to simply consider that only building is being optimized in understanding the problems. Considering only the alpha solution may also provide better understanding as well.

Let us start with $\vec{C} \cdot \vec{X_{l}}(t)$ from $\vec{D} = \left | \vec{C} \cdot \vec{X_{l}}(t) - \vec{X}(t) \right |$. To simplify our explanation, let $K = \vec{C} \cdot \vec{X_{l}}(t)$. The range of each $i$th element in $K$ will be $[0, 2 \cdot \vec{C}_{l,i}]$. Note that the operation is dot product, but pairwise multiplication. This means that $K$ simply scales the x and y positions, and angle of the buildings. Figure \ref{gwo-c-effect} shows a visualization of this effect. Despite the figure only showing the effect with a building's position in the first quadrant, the same effect can be observed with other buildings located in other quadrants. Now, considering the entirety of $\vec{D}$, $\vec{D}$ would mean to be the distance between a building $i$ moved to a different point in the region $S$ (see Figure \ref{gwo-c-effect}) in $\vec{X_{l}}$ and a building $i$ in $\vec{X}(t)$. A visualization for this is provided by Figure \ref{gwo-d-effect}.

\begin{figure}[h!]
	\centering
	\includegraphics[scale=0.45]{./images/chap05-methodology/gwo-c-effect.png}
	\caption{In $K$, $C$ simply scales the x and y positions and angles of buildings. Assuming that the point $B$ represents the x and y positions of a building, the region $S$ is where $B$ may be repositioned based on the values of $C$.}
	\label{gwo-c-effect}
\end{figure}

\begin{figure}[h!]
	\centering
	\includegraphics[scale=0.45]{./images/chap05-methodology/gwo-d-effect.png}
	\caption{A visualization of how $D$ is computed and its inherent meaning.}
	\label{gwo-d-effect}
\end{figure}

Since we now understand the meaning of $\vec{D}$, let us now move on to the other formula that causes the aforementioned problematic behaviour of classical GWO, $\vec{A} \cdot \vec{D}$. First, we should take note that $\vec{A} = 2 \cdot \vec{a} \cdot \vec{r_{1}} - \vec{a}$. $a$, as mentioned before, linearly decreases over time. Since $a$ decreases linearly over time, $\vec{A}$ will also decrease over time. This behaviour of $\vec{A}$ would mean that in $\vec{A} \cdot \vec{D}$, $\vec{D}$ will eventually decrease as well. This behaviour in itself is not a problem. This is a necessity as the buildings will not cluster together (and eventually produce a solution with a lower fitness value) if not for this behaviour. The major issue lies with how this behaviour interacts with $\vec{D}$. Over time through the course of iterations, $\vec{A}$ will eventually scale down the $\vec{D}$. This scaling down will also eventually reduce the size of each of the region $S$ of all buildings. This reduction of size will eventually cause the buildings to move towards the origin or the axes. Note that the penalty value for intersection prevents them from overlapping with one another. Buildings that are already on a certain axis will find it practically impossible to move in the direction of the perpendicular axis. Buildings that are on the origin itself will practically cease to move at all. Buildings will still be able to change their orientations, however. The reason for this behaviour of being stuck on an axis is due to the nature of axes themselves, where the value in one or both axes is zero. Since $K = \vec{C} \cdot \vec{X_{l}}(t)$ and when a building is near or already on an axis, the x, y, or both x and y positions of a building will barely, if at all, move away from the axes it is currently stuck to, when multiplying with $\vec{C}$. Hence, the behaviour we are noticing.

The two aforementioned formulas make classical GWO inadequate for our problem instance. We are unable to produce feasible nor satisfying results. In order for the grey wolf optimization algorithm to be successfully adapted into solving the facility layout problems, we must introduce a few changes into the algorithm. These changes will be discussed in the next subsection.

\subsection{Modified GWO}
% Draft Note: We should maybe give a name to the modified GWO. Maybe call it "Ballais-Romero GWO variant". HAHAHA.

Mirjalili, S., Mirjalili, S., and Lewis, A. \cite{Mirjalili2014} included a figure similar to Figure \ref{gwo-positioning-update}. It visualizes how a wolf $\omega$ will update its position based on the information provided by the leading wolves.

\begin{figure}[h!]
	\centering
	\includegraphics[scale=0.3]{./images/chap05-methodology/gwo-position-updating.png}
	\caption{Visualization of how wolves in GWO update their positions. An $\omega$ wolf will move towards a random point inside the circle of the estimated prey position.}
	\label{gwo-positioning-update}
\end{figure}

Basing from the visualization, notice that the $\vec{C}$ of the leading wolves specify the radius of the circle in which a $\vec{C} \cdot \vec{X_{l}}$ will be located it. The circle does \textbf{not} include an origin point. We have discussed before that performing a pairwise multiplication between $\vec{C}$ and $\vec{X_{l}}$ simply scales the elements $i$ in the vector $\vec{X_{l}}$. This is different from the visualization. To achieve the same effect as the visualization, instead of performing pairwise multiplication, we must utilize vector addition between $\vec{C}$ and $\vec{X_{l}}$. See Figure \ref{vector-addition-visualization} for a visualization of vector addition. This is the first modification we are introducing to classical GWO.

\begin{figure}[h!]
	\centering
	\includegraphics[scale=0.45]{./images/chap05-methodology/vector-addition-visualization.png}
	\caption{Vector addition pushes the point represented by $\vec{A}$ towards the direction of $\vec{B}$ by the magnitude of the same vector.}
	\label{vector-addition-visualization}
\end{figure}

In our modified GWO, $D$ is now defined as:

\begin{align}
	D &= \left | (\vec{C} + \vec{X_{l}}) - \vec{X}(t) \right | \label{modified-gwo-d}
\end{align}

However, this alone is not enough to comply with the aforementioned visualization. Using this will only move the buildings to the right and/or top. In order for us to move the buildings, we must also modify $\vec{C}$ as shown below:

\begin{align}
	C &= c \cdot \vec{r_{3}} \label{modified-gwo-c}
\end{align}

In this equation, $c$ is a real-valued variable, and $r_{3}$ is a random vector in $[-1, 1]$. This modification will now allow us to move a building from any direction and at any magnitude. The magnitude in which the building will be moved is controlled by $c$.

These modifications remove the necessity to specify an origin point in the bounding region, and the behaviour of buildings to move towards the origin or axes. Unfortunately, this alone is not enough to produce feasible results. We have to add two more modifications before we are able to produce good results.

The first additional modification is the building clamping. Each building is restricted to the boundary. If a building is moved towards outside the boundary, it will be pulled back to within the boundary. Building clamping can be mathematically defined as the following. Given a building $B$ in a solution $X(t)$ at iteration $t$ after being updated by Equations \ref{gwo-x1-eqn} to \ref{gwo-xt1-eqn}, \ref{modified-gwo-d}, and \ref{modified-gwo-c}, clamping can be mathematically modeled as:

\begin{align}
	B_{x} &= \text{max}\left (R_{x} + \frac{B_{w}}{2}, \text{min}\left(B_{x}, R_{x} + \left(R_{w} - \frac{B_{w}}{2}\right)\right)\right ) \\
	B_{y} &= \text{max}\left (R_{y} + \frac{B_{h}}{2}, \text{min}\left(B_{y}, R_{y} + \left(R_{h} - \frac{B_{h}}{2}\right)\right)\right )
\end{align}

where $B_{x}$ and $B_{y}$ are the $x$ and $y$ positions of the centroid of a building $B$, $B_{w}$ and $B_{h}$ are the width and height from the top left corner of a building $B$, $R_{x}$ and $R_{y}$ are the $x$ and $y$ positions of the top-left corner of the bounding region $R$, and $R_{w}$ and $R_{h}$ are the width and height of the bounding region $R$. Based on our prior experiments, without this clamping, buildings will freely move to points outside the boundary, and, at the end of the run, will produce a bad solution. 

This clamping should \textit{almost} solve the positioning of the buildings and allow us to produce results that are feasible. Since GWO is a continuous metaheuristic, building attributes that must only be one of two values will eventually be a value that is between the two aforementioned values. In our problem, this attribute that is affected is the building orientation. The building orientation may only be $0^{\circ}$ or $90^{\circ}$. It must never be a value between two. To solve this problem, we simply use the orientation of a building $B$ from the $\alpha$, $\beta$, or $\delta$ solutions, which are randomly selected. This idea is based off from the nature of GWO, where the best three solutions lead the search for the local optima. The building orientation of a building $B$ is, therefore, obtained using:

\begin{align}
	B_{o} = \left\{\begin{matrix}
		\alpha_{B_{o}} & \text{if } 0 \leq r < \frac{1}{3} \\ 
		\beta_{B_{o}}  & \text{if } \frac{1}{3} \leq r < \frac{2}{3}  \\ 
		\delta_{B_{o}} & \text{otherwise}
	\end{matrix}\right.
\end{align}

where $B_{o}$ is the current orientation of a building $B$, $\alpha_{B_{o}}$, $\beta_{B_{o}}$, and $\delta_{B_{o}}$ are the orientations of building $B$ in the $\alpha$, $\beta$, and $\delta$ solutions, respectively, and $r$ is a random variable in $[0, 1]$. In our approach, assigning the building orientations is performed before clamping the buildings.

All these modifications for the classical GWO have allowed us to successfully adapt GWO to the facility layout problem. Notice that these modifications are relatively simple, and do not significantly change the characteristics of classical GWO. The simplicity of GWO is still preserved. The next chapters will discuss how our modified version of GWO performs against configurations of unequal-area facility layout ptoblems.

(\textbf{COMMENT}: Sir, can we give this modified GWO a name? mGWO is already taken so maybe The Ballais-Romero GWO Variant? Still haven't came up with an alternative name. HAHAHAHAHAHAHA. \textbf{REMOVE THIS IN THE FINAL DRAFT, SEAN.})

%\section{The Algorithm}
%The proposed algorithm contains multiple phases to solve the unequal area static facility layout problem. The basic framework of the algorithm is inspired from the works of Asl et al. (2015) \cite{Asl2015} and Asl, A. and Wong, K. (2015) \cite{Asl2015}. The Grey Wolf Optimization aspect of the algorithm is inspired from Jiang, T. and Zhang, C. (2018) \cite{Jiang2018}. Figure \ref{algo-flowchart} shows a flowchart of the algorithm. We will be further discussing the algorithm in detail in this section.
%
%\begin{figure}[h!]
%	\centering
%	\includegraphics{./images/chap05-methodology/flowchart.png}
%	\caption{Flowchart detailing the algorithm.}
%	\label{algo-flowchart}
%\end{figure}
%
%\subsection{Population Generation}
%In generating the initial population, the order in which facilities are placed in the bounding region is shuffled. Once it is shuffled, each building $i$ is then given a random x position between the inclusive range $\left [ R_{x} + \frac{w_{i}}{2}, (R_{x} + R_{w}) - \frac{w_{i}}{2} \right ]$, a random y position between the inclusive range $\left [ R_{y} + \frac{h_{i}}{2}, (R_{y} + R_{h})  - \frac{h_{i}}{2} \right ]$, and angle that is either $0^{\circ}$ or $90^{\circ}$ (chosen in a uniformly random fashion). This will generate a solution where the buildings are spread out throughout the bounding region. There will be building intersections (though this is dependent on the number of buildings and size of the bounding region), but no building will be out-of-bounds. This process is repeated until we generate a specified number of solutions. This number is determined by the population size.
%
%\subsection{Swapping Method}
%The swapping method is used to find a possible configuration for a solution that is better than the current configuration. This method is applied to all solutions in the population, but only in the first 100 iterations. Pseudocode for the swapping method is provided in Algorithm \ref{pseudocode-swapping}.
%
%\begin{algorithm}
%\caption{Pseudocode for the swapping method.}
%\label{pseudocode-swapping}
%\begin{algorithmic}[1]
%\State Let $S$ be the collection of generated solutions.
%\State Set $S_{curr}$ be the current solution.
%\State Add $S_{curr}$ to $S$.
%\State Set $N_{B}$ be the maximum number of buildings.
%\For{i = 0 until $N_{B} - 2$}
%	\For{j = i + 1 until $N_{B} - 1$}
%		\State Building $i$'s orientation in $S_{curr}$ is changed to the other orientation,\WRP and the resulting new solution is added to $S$
%		\State Building $j$'s orientation in $S_{curr}$ is changed to the other orientation,\WRP and the resulting new solution is added to $S$
%		\State Building $i$'s and $j$'s orientations in $S_{curr}$ are changed to the other\WRP orientation, and the resulting new solution is added to $S$
%		\State Building $i$'s and $j$'s positions are exchanged in $S_{curr}$, and the resulting new\WRP solutiA movement and a orientation changeon is added to $S$.
%		\State Building $i$'s and $j$'s positions are exchanged in and the orientation of\WRP building $i$ is changed in $S_{curr}$, and the resulting new solution\WRP is added to $S$.
%		\State Building $i$'s and $j$'s positions are exchanged in and the orientation of\WRP building $j$ is changed in $S_{curr}$, and the resulting new solution\WRP is added to $S$.
%		\State Building $i$'s and $j$'s positions are exchanged in and the orientations of\WRP buildings $i$ and $j$ are changed in $S_{curr}$, and the resulting new solution\WRP is added to $S$.
%	\EndFor
%\EndFor \\
%\Return the best solution in $S$.
%\end{algorithmic}
%\end{algorithm}
%
%\subsection{Crossover, Mutation, and Elitism}
%Experimentations with directly applying the formulas of Grey Wolf Optimization in solving our instance of the facility layout problem have not generated satisfying results. However, results generated by an algorithm inspired from the paper of Jiang, T., and Zhang, C. (2018) \cite{Jiang2018} showed promise. As such the GWO portion of our proposed approach has been taken from their work. Their work takes certain elements from genetic algorithms, particularly the crossover and mutation operators, but kept the general principle of Grey Wolf optimization intact. We will be discussing each operator in detail in this section. Elitism is also implemented in our proposed approach to ensure that the best solutions found so far do not get lost throughout iterations. It will also be discussed in this section.
%
%\subsubsection{Crossover}
%Grey wolf optimization algorithms breed with the other solutions with the top three solutions in the population. That is modelled by the mathematical definition of the optimization algorithm. However, directly the definition does not yield good results. Due to that, we are using a crossover operator from the work of Jiang, T., and Zhang, C. (2018) \cite{Jiang2018}.
%
%The crossover operator used here can be mathematical defined as:
%
%$$
%X(t + 1) =
%\begin{cases}
%	f(X(t), X_{alpha}(t)) & \text{if }
%	\text{rnd} \leq \frac{1}{3} \\ 
%	f(X(t), X_{beta}(t)) & \text{if }
%	\frac{1}{3} < \text{rnd} \leq \frac{2}{3} \\
%	f(X(t), X_{delta}(t)) & \text{if }
%	\text{rnd} \geq \frac{2}{3}
%\end{cases} 
%$$
%
%where $f$ is the crossover function, and $\text{rnd} \sim U(0, 1)$. This means that a solution will be crossed over by one of the best three solutions, selected randomly. In our approach, we have used the uniform crossover. Note that this crossover is applied to every solution in a population in accordance with the principle of GWO. In our approach, only one offspring is generated from the crossover.
%
%\subsubsection{Mutation}
%Over time, as more and more iterations are performed, the diversity of the population will eventually be lost. This is due to the fact that the wolves are only updated based on the best three solutions. This will also result in premature convergence \cite{Jiang2018}. To combat this, a mutation operator is performed to reintroduce diversity. The mutation operator used is an operator that we dub the "Buddy-Buddy Mutation".
%
%The \textbf{Buddy-Buddy Mutation} is a mutation operator that simply selects two pairs of buildings $D$ and $S$, and move one of them to the side of the other building. Building $D$ is referred to as the dynamic buddy, while building $S$ is referred to as the static buddy. Building $D$ will be the building that will be moved towards the other building, which is building $S$ in our case. When moving building $D$, a side $E$ of building $S$ will first be randomly chosen. Afterwards, an orientation for building $D$ will be randomly chosen, whether it will be parallel or perpendicular to $E$. Once a side and orientation has been selected, building $D$ will be moved adjacent towards building $S$ at side $E$ with the chosen orientation. The implementation of this mutation operator in our proposed approach gives buildings that intersect with another building more chances of being selected as the dynamic buddy. Pseudocode and a visualization of the operator is provided by Algorithm \ref{pseudocode-buddybuddy-mutation} and Figure \ref{buddy-buddy-mutation-viz}, respectively.
%
%\begin{algorithm}
%	\caption{Pseudocode for the Buddy-Buddy Mutation.}
%	\label{pseudocode-buddybuddy-mutation}
%	\begin{algorithmic}[1]
%		\State Randomly select two buildings $D$ and $S$, with more weight given to buildings that are intersecting with another.
%		\State Set $E$ to be a randomly selected side of building $S$.
%		\State Set $O$ to either be a parallel or perpendicular orientation, randomly selected.
%		\State Move building $D$ adjacent to side $E$ of building $S$ with the orientation $O$.
%	\end{algorithmic}
%\end{algorithm}
%
%The rate at which a solution is mutated is highly dependent on the fitness of the solution. The worse the fitness of a solution is, the more likely it is to be mutated. This encourages the proposed algorithm to improve solutions that are generally bad. This rate scheme makes this an adaptive mutation operator \cite{Jiang2018}. The mutation rate is mathematically modelled as:
%
%\begin{equation}
%	m(X, t) = 1 - \frac{fit_{max}(t) - fit(X(t))}{fit_{max}(t) - fit_{min}(t)}
%\end{equation}
%
%where $m_{k}$ refers to the mutation rate of a solution $X$ at iteration $t$, $fit$ is a function that gets the fitness of a solution, and $fit_{min}$ and $fit_{max}$ gets the minimum and maximum fitnesses of the population, respectively.
%
%\subsubsection{Elitism}
%One variant of genetic algorithms includes elitism. This elitism allows a genetic algorithm to keep a number of best solutions in the next generation, ensuring that the best solutions do not get discarded over time. Note that this elitism strategy is not only limited to genetic algorithms. Other evolutionary algorithms may also utilize this strategy \cite{Du2018}. We are also taking this principle into our proposed approach. In our proposed approach, we are keeping the best three solutions in the previous iteration to the next iteration. We chose three due to the fact that the best three solutions in a population have significance in GWO.
%
%\subsection{Local Searches}
%Remember that our implementation is based on aforementioned previous works that used local search algorithms in conjunction to genetic algorithms. They combined GAs with local search algorithms because GAs find it hard to explore within the convergence area. Hybridizing it with a local search algorithm improves performance \cite{Ripon2013}. In our proposed approach, we are keeping this aspect of the previous works. This will also ensure that we are able to search within the convergence area more intensely and find better solutions. In the previous works and in ours, there are two local search algorithms, dubbed "Local Search 1" and "Local Search 2". They vary in terms of searching intensity, but both attempts to obtain better solutions. We will be discussing details of both in this section.
%
%\subsubsection{Local Search 1}
%The first local search algorithm, "Local Search 1", performs a local search by creating a number of solutions by moving each building in different directions by a certain random amount and changing its orientations after movement and obtaining the best solution from these activities. In our approach, the certain amount of movement is a random number between 1 and 5. This search algorithm is only applied to the best solution of the current iteration, and the best solution found in this search becomes the new best solution and replaces the previously best solution. The movements of each building is defined by a set of "activities". This set of activities is shown by Table \ref{local-search-1-activities}. Additionally, pseudocode of the search algorithm is shown in Algorithm \ref{pseudocode-local-search-1}.
%
%\begin{table}[h!]
%	\centering
%	\begin{tabular}{| c | p{110mm} |}
%		\hline
%		Activity Number & Description \\
%		\hline
%		0 & A building is moved to the right along the x-axis by a random number between 1 and 5. \\
%		1 & A building is moved to the left along the x-axis by a random number between 1 and 5. \\
%		2 & A building is moved to the upwards along the y-axis by a random number between 1 and 5. \\
%		3 & A building is moved to the downwards along the y-axis by a random number between 1 and 5. \\
%		4 & Generate two random numbers from 1 and 5 and a building is moved to the right and then upward, respectively, by those numbers. \\
%		5 & Generate two random numbers from 1 and 5 and a building is moved to the right and then downward, respectively, by those numbers. \\
%		6 & Generate two random numbers from 1 and 5 and a building is moved to the left and then upward, respectively, by those numbers. \\
%		7 & Generate two random numbers from 1 and 5 and a building is moved to the left and then downward, respectively, by those numbers. \\
%		\hline
%	\end{tabular}
%	\caption{Activities for moving a building in Local Search 1}
%	\label{local-search-1-activities}
%\end{table}
%
%\begin{algorithm}[h!]
%\caption{Pseudocode for Local Search 1.}
%\label{pseudocode-local-search-1}
%\begin{algorithmic}[1]
%\State Set $S$ to be a collection of solutions.
%\State Set $S_{curr}$ to be the solution being optimized.
%\State Add $S_{curr}$ to $S$.
%\State Set $N_{B}$ be the maximum number of buildings.
%\State Set $N_{A}$ be the maximum number of activities.
%\For{i = 0 until $N_{B} - 1$}
%	\For{a = 0 until $N_{B} - 1$}
%		\State Perform activity $a$ with building $i$ in $S_{curr}$ and save the new solution in $S$.
%		\State Perform activity $a$ with building $i$, and change the orientation of the building to the other orientation in $S_{curr}$ and save the new solution in $S$.
%	\EndFor
%\EndFor \\
%\Return the best solution in $S$.
%\end{algorithmic}
%\end{algorithm}
%
%\subsubsection{Local Search 2}
%Local Search 2 is a more intense version of Local Search 1, in order to find the best solution so far. Unlike the latter that only moves one building at a time, Local Search 2 moves two buildings instead. The two buildings will also have their orientations changed after each activity. This local search is only applied to the best solution found in the last 50 iterations. The set of activities for this local search is shown by Table \ref{local-search-2-activities}, and a pseudocode of the search algorithm is shown in Algorithm \ref{pseudocode-local-search-2}.
%
%\begin{algorithm}[h!]
%	\caption{Pseudocode for Local Search 2.}
%	\label{pseudocode-local-search-2}
%	\begin{algorithmic}[1]
%		\State Set $S$ to be a collection of solutions.
%		\State Set $S_{curr}$ to be the solution being optimized.
%		\State Add $S_{curr}$ to $S$.
%		\State Set $N_{B}$ be the maximum number of buildings.
%		\State Set $N_{A}$ be the maximum number of activities.
%		\For{i = 0 until $N_{B} - 2$}
%		\For{a = 0 until $N_{B} - 1$}
%		\State Perform activity $a$ with building $i$ in $S_{curr}$ and save the new solution in $S$.
%		\State Perform activity $a$ with building $i$, and change the orientation of \WRP building $i$ to the other orientation in $S_{curr}$ and save the new \WRP solution in $S$.
%		\State Perform activity $a$ with building $i$, and change the orientation of \WRP building $i + 1$ to the other orientation in $S_{curr}$ and save the new \WRP solution in $S$.
%		\State Perform activity $a$ with building $i$, and change the orientations of \WRP buildings $i$ and $i + 1$ to the other orientations in $S_{curr}$ and save the new \WRP solution in $S$.
%		\EndFor
%		\EndFor \\
%		\Return the best solution in $S$.
%	\end{algorithmic}
%\end{algorithm}
%
%\begin{longtable}{| c | p{120mm} |}
%	\hline
%	Activity Number & Description \\
%	\hline
%	0  & Building $i$ and $i + 1$ are moved to the right along the x-axis by a random number between 1 and 5. \\
%	1  & Building $i$ and $i + 1$ are moved to the left along the x-axis by a random number between 1 and 5. \\
%	2  & Building $i$ and $i + 1$ are moved upwards along the x-axis by a random number between 1 and 5. \\
%	3  & Building $i$ and $i + 1$ are moved downwards along the x-axis by a random number between 1 and 5. \\
%	4  & Generate two random numbers from 1 and 5 and buildings $i$ and $i + 1$ are moved to the right and then upward, respectively, by those numbers. \\
%	5  & Generate two random numbers from 1 and 5 and buildings $i$ and $i + 1$ are moved to the right and then downward, respectively, by those numbers. \\
%	6  & Generate two random numbers from 1 and 5 and buildings $i$ and $i + 1$ are moved to the left and then upward, respectively, by those numbers. \\
%	7  & Generate two random numbers from 1 and 5 and buildings $i$ and $i + 1$ are moved to the left and then downward, respectively, by those numbers. \\
%	8  & Generate two random numbers $a$ and $b$ that are from 1 to 5, and building $i$ is moved upward by $a$ and building $i + 1$ is moved to the right by $b$. \\
%	9  & Generate two random numbers $a$ and $b$ that are from 1 to 5, and building $i$ is moved upward by $a$ and building $i + 1$ is moved downward by $b$. \\
%	10 & Generate two random numbers $a$ and $b$ that are from 1 to 5, and building $i$ is moved upward by $a$ and building $i + 1$ is moved to the left by $b$. \\
%	11 & Generate two random numbers $a$ and $b$ that are from 1 to 5, and building $i$ is moved to the right by $a$ and building $i + 1$ is moved downward by $b$. \\
%	12 & Generate two random numbers $a$ and $b$ that are from 1 to 5, and building $i$ is moved to the right by $a$ and building $i + 1$ is moved upward by $b$. \\
%	13 & Generate two random numbers $a$ and $b$ that are from 1 to 5, and building $i$ is moved to the right by $a$ and building $i + 1$ is moved to the left by $b$. \\
%	14 & Generate two random numbers $a$ and $b$ that are from 1 to 5, and building $i$ is moved to the left by $a$ and building $i + 1$ is moved to downward by $b$. \\
%	15 & Generate two random numbers $a$ and $b$ that are from 1 to 5, and building $i$ is moved to the left by $a$ and building $i + 1$ is moved to the right by $b$. \\
%	16 & Generate two random numbers $a$ and $b$ that are from 1 to 5, and building $i$ is moved to the left by $a$ and building $i + 1$ is moved upward by $b$. \\
%	17 & Generate two random numbers $a$ and $b$ that are from 1 to 5, and building $i$ is moved downward by $a$ and building $i + 1$ is moved to the right by $b$. \\
%	18 & Generate two random numbers $a$ and $b$ that are from 1 to 5, and building $i$ is moved downward by $a$ and building $i + 1$ is moved to the left by $b$. \\
%	19 & Generate two random numbers $a$ and $b$ that are from 1 to 5, and building $i$ is moved downward by $a$ and building $i + 1$ is moved upward by $b$. \\
%	\hline
%	\caption{Activities for moving a building in Local Search 2}
%	\label{local-search-2-activities}
%\end{longtable}

\section{Implementation Technologies}
Our proposed approach was developed using C++17 compiled using the Clang 11 compiler in an elementaryOS Hera environment. Building was handled by CMake, and package management was handled by Conan. Our implementation is built on top of CoreX, a custom-developed 2D game engine. Using a game engine allowed us to visualize the results and configure experiments in a graphical manner. Using a custom engine over an over-the-shelf engine ensures that the implementation remains light and does not carry unnecessary features that are typically used in commercial game engines. The libraries EASTL, ImGUI, SDL 2, SDL 2 TTF. sdl-gpu, nlohmann JSON, and EnTT were used in developing the engine, with EnTT and ImGUI directly used by our implementation itself.
\chapter{Describing How You Validated Your Approach.}


\chapter{Results and Discussion}
The results of our experiments with our data set will be presented in this chapter. 30 runs of each approach that produced feasible solutions are included in the results. Note that some runs produce an infeasible solution. This due to the non-deterministic nature of metaheuristics, which will cause it to produce infeasible solutions sometimes.

\section{Environment}
All of the approaches were run in the following hardware and software configurations:

\begin{itemize}
	\item Hardware
	\begin{itemize}
		\item \textbf{CPU}: AMD Ryzen 5 5600X
		\item \textbf{GPU}: NVIDIA GeForce GTX 1050 Ti
		\item \textbf{RAM}: Crucial Ballistix RGB 3600 MHz DDR4 16 GB (8 GB x 2) CL16
	\end{itemize}
	\item Software
	\begin{itemize}
		\item \textbf{OS}: elementaryOS 5.1.7 Hera
		\item \textbf{Linux Kernel Version}: 5.4.0-99-generic
	\end{itemize}
\end{itemize}

\section{Experiments}
We have conducted two sets of experiments in order for us to evaluate the performance of our proposed GWO approach. The first set of experiments varies the GWO parameter, $c$, and the population size. It shows the impact of the parameters to the algorithm. The second set are experiments with the competing approaches. It shows how our proposed approach compares against other approaches. A population size of $50$ is used for all the experiment runs in the second set. This set will then be compared to the GWO experiment runs that have a population of the same size.

\subsection{Results with Different GWO Parameter Values}
Our proposed GWO approach only has one parameter, other than the population size, and number of iterations, the $c$ value. We used four values for the parameter: $2$, $4$, $8$, and $12$. We also varied the population size for this experiment set. The population sizes we used are: $25$, $50$, and $75$. Tables \ref{approach-gwo-c2-p25-results} to \ref{approach-gwo-c12-p75-results} show the results. We will refer to each GWO experiment as $G_{n,c}$, where $n$ is the population size, and $c$ is the value of the $c$ parameter in each experiment. So, for example, the GWO experiment with a population size of $25$ and $c = 2$ will be referred to as $G_{25,2}$, and so on. Each experiment for each parameter configuration combination has been run 30 times.

% Start of the GWO table of results for the one with a population of 25.
\begin{table}[h!]
\begin{adjustwidth}{-1.18in}{}
	\centering
	\begin{tabular}{|l|l|l|l|l|l|}
	\hline
	\multicolumn{1}{|c|}{\multirow{2}{*}{\textbf{Problem}}} & \multicolumn{5}{c|}{\textbf{GWO (c = 2, Pop. Size of 25)}} \\ \cline{2-6} 
	\multicolumn{1}{|c|}{}                                  & \multicolumn{1}{c|}{\textbf{Best}} & \multicolumn{1}{c|}{\textbf{Worst}} & \multicolumn{1}{c|}{\textbf{Avg.}} & \multicolumn{1}{c|}{\textbf{Std. Dev.}} & \multicolumn{1}{c|}{\textbf{Avg. Runtime (s)}} \\ \hline
	SFLP-II                                                 & 226.149871                                  & 328.546146                                   & 287.749326366667                      & 24.7018482581174                                 & 6.03333333333333                                  \\ \hline
	mSFLP-III                                               & 50874.238564                                & 60974.998642                                 & 55624.0061857667						         & 2544.70550235336                              & 22.4333333333333                               \\ \hline
	mKra30a                                               & 94640.759514                                & 120397.403767                                 &
	107874.742523333							&
	6706.6049593287							&
	35.9666666666667						\\ \hline
	\end{tabular}
\end{adjustwidth}
\caption{Results obtained from our proposed GWO approach with $c = 2$ and a population of $25$.}
\label{approach-gwo-c2-p25-results}
\end{table}

\begin{table}[h!]
\begin{adjustwidth}{-1.18in}{}
	\centering
	\begin{tabular}{|l|l|l|l|l|l|}
	\hline
	\multicolumn{1}{|c|}{\multirow{2}{*}{\textbf{Problem}}} & \multicolumn{5}{c|}{\textbf{GWO (c = 4, Pop. Size of 25)}} \\ \cline{2-6} 
	\multicolumn{1}{|c|}{}                                  & \multicolumn{1}{c|}{\textbf{Best}} & \multicolumn{1}{c|}{\textbf{Worst}} & \multicolumn{1}{c|}{\textbf{Avg.}} & \multicolumn{1}{c|}{\textbf{Std. Dev.}} & \multicolumn{1}{c|}{\textbf{Avg. Runtime (s)}} \\ \hline
	SFLP-II                                                 & 228.710226                                  & 373.604858                                   & 298.836421533333                      & 34.1522620177737                                 & 5.9                                  \\ \hline
	mSFLP-III                                               & 50825.278824                                & 59194.486832                                 & 55236.4286594						         & 2301.71070299619                              & 21.3333333333333                               \\ \hline
	mKra30a                                               & 87110.57618                                & 116746.599121                                 &
	104270.487286567							&
	8041.05756072353							&
	36.3333333333333						\\ \hline
	\end{tabular}
\end{adjustwidth}
\caption{Results obtained from our proposed GWO approach with $c = 4$ and a population of $25$.}
\label{approach-gwo-c4-p25-results}
\end{table}

\begin{table}[h!]
\begin{adjustwidth}{-1.18in}{}
	\centering
	\begin{tabular}{|l|l|l|l|l|l|}
	\hline
	\multicolumn{1}{|c|}{\multirow{2}{*}{\textbf{Problem}}} & \multicolumn{5}{c|}{\textbf{GWO (c = 8, Pop. Size of 25)}} \\ \cline{2-6} 
	\multicolumn{1}{|c|}{}                                  & \multicolumn{1}{c|}{\textbf{Best}} & \multicolumn{1}{c|}{\textbf{Worst}} & \multicolumn{1}{c|}{\textbf{Avg.}} & \multicolumn{1}{c|}{\textbf{Std. Dev.}} & \multicolumn{1}{c|}{\textbf{Avg. Runtime (s)}} \\ \hline
	SFLP-II                                                 & 249.624192                                  & 368.435807                                   & 313.640670633333                     & 29.7958232730557                                 & 6.23333333333333                                  \\ \hline
	mSFLP-III                                               & 51250.638187                                & 57894.186882                                 & 54206.6467008333						         & 1334.40810225102                              & 23.6                               \\ \hline
	mKra30a                                               & 95254.061554                                & 118719.490257                                 &
	105760.743434367							&
	6557.69877516131							&
	37.0666666666667						\\ \hline
	\end{tabular}
\end{adjustwidth}
\caption{Results obtained from our proposed GWO approach with $c = 8$ and a population of $25$.}
\label{approach-gwo-c8-p25-results}
\end{table}

\begin{table}[h!]
\begin{adjustwidth}{-1.18in}{}
	\centering
	\begin{tabular}{|l|l|l|l|l|l|}
	\hline
	\multicolumn{1}{|c|}{\multirow{2}{*}{\textbf{Problem}}} & \multicolumn{5}{c|}{\textbf{GWO (c = 12, Pop. Size of 25)}} \\ \cline{2-6} 
	\multicolumn{1}{|c|}{}                                  & \multicolumn{1}{c|}{\textbf{Best}} & \multicolumn{1}{c|}{\textbf{Worst}} & \multicolumn{1}{c|}{\textbf{Avg.}} & \multicolumn{1}{c|}{\textbf{Std. Dev.}} & \multicolumn{1}{c|}{\textbf{Avg. Runtime (s)}} \\ \hline
	SFLP-II                                                 & 255.639347                                  & 358.033844                                   & 315.207093933333                      & 26.4399229476443                                 & 6.36666666666667                                  \\ \hline
	mSFLP-III                                               & 51328.437737                                & 62758.004044                                 & 55331.2312675333						         & 2540.54370041386                              & 21.2333333333333                              \\ \hline
	mKra30a                                               & 93525.765816                                & 124181.531029                                 &
	108251.9895637							&
	8151.19724950765							&
	37.3						\\ \hline
	\end{tabular}
\end{adjustwidth}
\caption{Results obtained from our proposed GWO approach with $c = 12$ and a population of $25$.}
\label{approach-gwo-c12-p25-results}
\end{table}

% Start of the GWO table of results for the one with a population of 50.
\begin{table}[h!]
	\begin{adjustwidth}{-1.18in}{}
		\centering
		\begin{tabular}{|l|l|l|l|l|l|}
			\hline
			\multicolumn{1}{|c|}{\multirow{2}{*}{\textbf{Problem}}} & \multicolumn{5}{c|}{\textbf{GWO (c = 2, Pop. Size of 50)}} \\ \cline{2-6} 
			\multicolumn{1}{|c|}{}                                  & \multicolumn{1}{c|}{\textbf{Best}} & \multicolumn{1}{c|}{\textbf{Worst}} & \multicolumn{1}{c|}{\textbf{Avg.}} & \multicolumn{1}{c|}{\textbf{Std. Dev.}} & \multicolumn{1}{c|}{\textbf{Avg. Runtime (s)}} \\ \hline
			SFLP-II                                                 & 221.18019                                  & 341.568304                                   & 283.795019233333                      & 28.9742518867792                                 & 13.2666666666667                                  \\ \hline
			mSFLP-III                                               & 47597.794662                                & 62827.738159                                 & 54495.2676476667						         & 3328.18744058766                              & 40.1333333333333                               \\ \hline
			mKra30a                                               & 88657.824898                                & 121124.59779                                 &
			102742.1803823							&
			7156.18271087496							&
			73.3333333333333						\\ \hline
		\end{tabular}
	\end{adjustwidth}
	\caption{Results obtained from our proposed GWO approach with $c = 2$ and a population of $50$.}
	\label{approach-gwo-c2-p50-results}
\end{table}

\begin{table}[h!]
	\begin{adjustwidth}{-1.18in}{}
		\centering
		\begin{tabular}{|l|l|l|l|l|l|}
			\hline
			\multicolumn{1}{|c|}{\multirow{2}{*}{\textbf{Problem}}} & \multicolumn{5}{c|}{\textbf{GWO (c = 4, Pop. Size of 50)}} \\ \cline{2-6} 
			\multicolumn{1}{|c|}{}                                  & \multicolumn{1}{c|}{\textbf{Best}} & \multicolumn{1}{c|}{\textbf{Worst}} & \multicolumn{1}{c|}{\textbf{Avg.}} & \multicolumn{1}{c|}{\textbf{Std. Dev.}} & \multicolumn{1}{c|}{\textbf{Avg. Runtime (s)}} \\ \hline
			SFLP-II                                                 & 241.862298                                  & 389.711568                                   & 294.2461242                      & 33.7641257773172                                 & 13.1333333333333                                  \\ \hline
			mSFLP-III                                               & 50250.080536                                & 57916.673454                                 & 53421.9267731333						         & 2239.06725435468                              & 41.9666666666667                               \\ \hline
			mKra30a                                               & 90599.06601                                & 122300.269909                                 &
			102855.4831497							&
			8820.10238434929							&
			70.9666666666667						\\ \hline
		\end{tabular}
	\end{adjustwidth}
	\caption{Results obtained from our proposed GWO approach with $c = 4$ and a population of $50$.}
	\label{approach-gwo-c4-p50-results}
\end{table}

\begin{table}[h!]
	\begin{adjustwidth}{-1.18in}{}
		\centering
		\begin{tabular}{|l|l|l|l|l|l|}
			\hline
			\multicolumn{1}{|c|}{\multirow{2}{*}{\textbf{Problem}}} & \multicolumn{5}{c|}{\textbf{GWO (c = 8, Pop. Size of 50)}} \\ \cline{2-6} 
			\multicolumn{1}{|c|}{}                                  & \multicolumn{1}{c|}{\textbf{Best}} & \multicolumn{1}{c|}{\textbf{Worst}} & \multicolumn{1}{c|}{\textbf{Avg.}} & \multicolumn{1}{c|}{\textbf{Std. Dev.}} & \multicolumn{1}{c|}{\textbf{Avg. Runtime (s)}} \\ \hline
			SFLP-II                                                 & 240.638127                                  & 413.077936                                   & 299.553292466667                     & 40.469222577263                                 & 12.7333333333333                                  \\ \hline
			mSFLP-III                                               & 48844.175789                                & 59710.615997                                 & 52699.5983075667						         & 2062.76885562279                              & 41.2666666666667                               \\ \hline
			mKra30a                                               & 84929.672058                                & 112251.415863                                 &
			101570.163644533							&
			6490.61277032704							&
			72.1						\\ \hline
		\end{tabular}
	\end{adjustwidth}
	\caption{Results obtained from our proposed GWO approach with $c = 8$ and a population of $50$.}
	\label{approach-gwo-c8-p50-results}
\end{table}

\begin{table}[h!]
	\begin{adjustwidth}{-1.18in}{}
		\centering
		\begin{tabular}{|l|l|l|l|l|l|}
			\hline
			\multicolumn{1}{|c|}{\multirow{2}{*}{\textbf{Problem}}} & \multicolumn{5}{c|}{\textbf{GWO (c = 12, Pop. Size of 50)}} \\ \cline{2-6} 
			\multicolumn{1}{|c|}{}                                  & \multicolumn{1}{c|}{\textbf{Best}} & \multicolumn{1}{c|}{\textbf{Worst}} & \multicolumn{1}{c|}{\textbf{Avg.}} & \multicolumn{1}{c|}{\textbf{Std. Dev.}} & \multicolumn{1}{c|}{\textbf{Avg. Runtime (s)}} \\ \hline
			SFLP-II                                                 & 261.869799                                  & 381.586061                                   & 315.831642166667                      & 31.7204847308938                                 & 13.6666666666667                                  \\ \hline
			mSFLP-III                                               & 48920.979538                                & 56477.689476                                 & 52837.3591700333						         & 1980.22102755171                              & 42.2333333333333                              \\ \hline
			mKra30a                                               & 92563.720146                                & 128598.716599                                 &
			105348.4949903							&
			9267.72959691125							&
			69.5666666666667						\\ \hline
		\end{tabular}
	\end{adjustwidth}
	\caption{Results obtained from our proposed GWO approach with $c = 12$ and a population of $50$.}
	\label{approach-gwo-c12-p50-results}
\end{table}

% Start of the GWO table of results for the one with a population of 75.
\begin{table}[h!]
	\begin{adjustwidth}{-1.18in}{}
		\centering
		\begin{tabular}{|l|l|l|l|l|l|}
			\hline
			\multicolumn{1}{|c|}{\multirow{2}{*}{\textbf{Problem}}} & \multicolumn{5}{c|}{\textbf{GWO (c = 2, Pop. Size of 75)}} \\ \cline{2-6} 
			\multicolumn{1}{|c|}{}                                  & \multicolumn{1}{c|}{\textbf{Best}} & \multicolumn{1}{c|}{\textbf{Worst}} & \multicolumn{1}{c|}{\textbf{Avg.}} & \multicolumn{1}{c|}{\textbf{Std. Dev.}} & \multicolumn{1}{c|}{\textbf{Avg. Runtime (s)}} \\ \hline
			SFLP-II                                                 & 205.666955                                  & 386.356476                                   & 290.063388433333                      & 38.0436802516604                                 & 18.8666666666667                                  \\ \hline
			mSFLP-III                                               & 50179.684898                                & 57659.66436                                 & 54015.3749653						         & 2050.7167136713                              & 61.1                               \\ \hline
			mKra30a                                               & 88740.484344                                & 117173.305939                                 &
			99149.0616948							&
			5833.24082935413							&
			104.7						\\ \hline
		\end{tabular}
	\end{adjustwidth}
	\caption{Results obtained from our proposed GWO approach with $c = 2$ and a population of $75$.}
	\label{approach-gwo-c2-p75-results}
\end{table}

\begin{table}[h!]
	\begin{adjustwidth}{-1.18in}{}
		\centering
		\begin{tabular}{|l|l|l|l|l|l|}
			\hline
			\multicolumn{1}{|c|}{\multirow{2}{*}{\textbf{Problem}}} & \multicolumn{5}{c|}{\textbf{GWO (c = 4, Pop. Size of 75)}} \\ \cline{2-6} 
			\multicolumn{1}{|c|}{}                                  & \multicolumn{1}{c|}{\textbf{Best}} & \multicolumn{1}{c|}{\textbf{Worst}} & \multicolumn{1}{c|}{\textbf{Avg.}} & \multicolumn{1}{c|}{\textbf{Std. Dev.}} & \multicolumn{1}{c|}{\textbf{Avg. Runtime (s)}} \\ \hline
			SFLP-II                                                 & 239.258536                                  & 406.790997                                   & 302.005947566667                      & 39.1344289013742                                 & 18.9                                  \\ \hline
			mSFLP-III                                               & 48752.443314                                & 56156.624268                                 & 52037.6629276						         & 2313.4004317195                              & 61.5333333333333                               \\ \hline
			mKra30a                                               & 89197.608078                                & 121878.61042                                 &
			99482.2352776							&
			7069.6968084522							&
			107.966666666667						\\ \hline
		\end{tabular}
	\end{adjustwidth}
	\caption{Results obtained from our proposed GWO approach with $c = 4$ and a population of $75$.}
	\label{approach-gwo-c4-p75-results}
\end{table}

\begin{table}[h!]
	\begin{adjustwidth}{-1.18in}{}
		\centering
		\begin{tabular}{|l|l|l|l|l|l|}
			\hline
			\multicolumn{1}{|c|}{\multirow{2}{*}{\textbf{Problem}}} & \multicolumn{5}{c|}{\textbf{GWO (c = 8, Pop. Size of 75)}} \\ \cline{2-6} 
			\multicolumn{1}{|c|}{}                                  & \multicolumn{1}{c|}{\textbf{Best}} & \multicolumn{1}{c|}{\textbf{Worst}} & \multicolumn{1}{c|}{\textbf{Avg.}} & \multicolumn{1}{c|}{\textbf{Std. Dev.}} & \multicolumn{1}{c|}{\textbf{Avg. Runtime (s)}} \\ \hline
			SFLP-II                                                 & 246.916627                                  & 393.744452                                   & 309.957982766667                     & 32.0156308267039                                 & 19.5                                  \\ \hline
			mSFLP-III                                               & 49276.248596                                & 54977.558044                                 & 51801.8837937333						         & 1419.1918023338                              & 63.0333333333333                               \\ \hline
			mKra30a                                               & 87299.715054                                & 113760.078079                                 &
			98968.6223121							&
			6443.60722715266							&
			108.7						\\ \hline
		\end{tabular}
	\end{adjustwidth}
	\caption{Results obtained from our proposed GWO approach with $c = 8$ and a population of $75$.}
	\label{approach-gwo-c8-p75-results}
\end{table}

\begin{table}[h!]
	\begin{adjustwidth}{-1.18in}{}
		\centering
		\begin{tabular}{|l|l|l|l|l|l|}
			\hline
			\multicolumn{1}{|c|}{\multirow{2}{*}{\textbf{Problem}}} & \multicolumn{5}{c|}{\textbf{GWO (c = 12, Pop. Size of 75)}} \\ \cline{2-6} 
			\multicolumn{1}{|c|}{}                                  & \multicolumn{1}{c|}{\textbf{Best}} & \multicolumn{1}{c|}{\textbf{Worst}} & \multicolumn{1}{c|}{\textbf{Avg.}} & \multicolumn{1}{c|}{\textbf{Std. Dev.}} & \multicolumn{1}{c|}{\textbf{Avg. Runtime (s)}} \\ \hline
			SFLP-II                                                 & 243.386427                                  & 413.874466                                   & 307.3213231                      & 36.2239957721235                                 & 19.3333333333333                                  \\ \hline
			mSFLP-III                                               & 49644.232903                                & 55524.684891                                 & 51837.8766529667						         & 1430.57988385005                              & 61.7666666666667                              \\ \hline
			mKra30a                                               & 86942.304199                                & 108422.175175                                 &
			98108.9092933667							&
			6511.43059062118							&
			111.266666666667						\\ \hline
		\end{tabular}
	\end{adjustwidth}
	\caption{Results obtained from our proposed GWO approach with $c = 12$ and a population of $75$.}
	\label{approach-gwo-c12-p75-results}
\end{table}

Let us first discuss the results with the SFLP-II problem. Table \ref{full-data-gwo-sflp-ii} provides a summary of the results of the experiments performed for the problem using the GWO approach. Figure \ref{approach-gwo-sflp-ii-average-fitness-as-c-value-increases}, on the other hand, shows a line graph that displays the average fitness of solutions of each population size when solving the SFLP-II problem as the $c$ value increases. For the problem, $G_{50,2}$ has the best average compared to the other configurations with a value of $283.795019233333$. The worst average belonged to $G_{50,12}$ with a value of $315.831642166667$. The configuration with the best solution produced would be $G_{75,2}$ with the solution having a value of $205.666955$. Figure \ref{approach-gwo-sflp-ii-best-solution-fitness-over-time} shows the progress of the solution as the number of iterations increase. On the other hand, the worst solution was produced by $G_{75,12}$ with a value of $413.874466$. Figure \ref{approach-gwo-sflp-ii-best-and-worst-solutions-visualization} shows what these solutions look like. The average runtime of the experiments increase as the population size increases. In a similar fashion, the experiments with population sizes of $25$ and $50$, their average fitness value worsens (increases) as the value of $c$ increases. However, with a population size of $75$, the same trend is reflected until $c = 12$, where the average fitness improves.

\begin{figure}[h!]
\centering
\begin{adjustwidth}{-0.45in}{}
	\includegraphics[scale=1.0]{./images/chap07-rd/gwo-sflp2-best-and-worst-solutions.png}
\end{adjustwidth}
\caption{Visualizations of the best and worst solutions generated by our GWO approach for the SFLP-II problem. The best solution was generated by $G_{75,2}$, and the worst generated by $G_{75,12}$.}
\label{approach-gwo-sflp-ii-best-and-worst-solutions-visualization}
\end{figure}

\begin{figure}[h!]
\centering
\begin{adjustwidth}{-0.45in}{}
	\includegraphics[scale=0.5]{./images/chap07-rd/gwo-only-sflp2-average-fitness-as-c-value-increases.png}
\end{adjustwidth}
\caption{The average fitness of the solutions produced by our GWO approach as the $c$ value increases when solving the SFLP-II problem. Each line uses a different population with the blue line representing experiments using a population size ($n$) of 25, the yellow lines representing those with $n = 50$, and the green lines representing those with $n = 75$.}
\label{approach-gwo-sflp-ii-average-fitness-as-c-value-increases}
\end{figure}

\begin{figure}[h!]
\centering
\begin{adjustwidth}{-0.45in}{}
	\includegraphics[scale=0.5]{./images/chap07-rd/gwo-only-sflp2-best-solution-fitness-graph.png}
\end{adjustwidth}
\caption{The fitness of the best solution generated by $G_{75,2}$ as the number of iterations increase while solving the SFLP-II problem.}
\label{approach-gwo-sflp-ii-best-solution-fitness-over-time}
\end{figure}

As with the mSFLP-III problem, of which Table \ref{full-data-gwo-msflp-iii} provides the summary of the experimental results, the best average was produced by $G_{75,8}$ with a value of $51801.8837937333$. $G_{25,2}$ produced the worst average with a value of $55624.0061857667$. For this problem, the best solution produced has a fitness of $47597.794662$, and is generated by $G_{50,2}$. Figure \ref{approach-gwo-msflp-iii-best-solution-fitness-over-time} shows the progress of this solution as the number of iterations increase. Interestingly, $G_{50,2}$ has also generated the worst solution with a fitness value of $62827.738159$. These solutions are visualized by Figure \ref{approach-gwo-msflp-iii-best-and-worst-solutions-visualization}. Parallel to the observed behaviour with the SFLP-II problem, the average runtime of the experiments also increases as the size of the population increases. A trend is also observable as the $c$ value increase that applies to all population sizes used. Increasing the $c$ value shows an improvement in the fitness value (value decreases). Unfortunately, this behaviour changes when $c = 12$, where the fitness worsens. To supplement Table \ref{full-data-gwo-msflp-iii}, we are also providing Figure \ref{approach-gwo-msflp-iii-average-fitness-as-c-value-increases}, which presents a line graph version of the results displaying the relationship between the $c$ value and the average fitness of the experiments solving the problem in every population size we are using.

\begin{figure}[h!]
\centering
\begin{adjustwidth}{-0.45in}{}
	\includegraphics[scale=1.0]{./images/chap07-rd/gwo-msflp-iii-best-and-worst-solutions.png}
\end{adjustwidth}
\caption{Visualizations of the best and worst solutions generated by our GWO approach for the mSFLP-III problem. The best and worst solutions were generated by $G_{50,2}$.}
\label{approach-gwo-msflp-iii-best-and-worst-solutions-visualization}
\end{figure}

\begin{figure}[h!]
\centering
\begin{adjustwidth}{-0.45in}{}
	\includegraphics[scale=0.5]{./images/chap07-rd/gwo-only-msflp3-average-fitness-as-c-value-increases.png}
\end{adjustwidth}
\caption{The average fitness of the solutions produced by our GWO approach as the $c$ value increases when solving the mSFLP-III problem. Each line uses a different population with the blue line representing experiments using a population size ($n$) of 25, the yellow lines representing those with $n = 50$, and the green lines representing those with $n = 75$.}
\label{approach-gwo-msflp-iii-average-fitness-as-c-value-increases}
\end{figure}

\begin{figure}[h!]
\centering
\begin{adjustwidth}{-0.45in}{}
	\includegraphics[scale=0.5]{./images/chap07-rd/gwo-only-sflp2-best-solution-fitness-graph.png}
\end{adjustwidth}
\caption{The fitness of the best solution generated by $G_{50,2}$ as the number of iterations increasee while solving the mSFLP-III problem.}
\label{approach-gwo-msflp-iii-best-solution-fitness-over-time}
\end{figure}

Lastly, we present a brief overview of the results we obtained for the mKra30a problem. Table \ref{full-data-gwo-mkra30a} shows the summary of the experimental results for the problem, and Figure \ref{approach-gwo-mkra30a-average-fitness-as-c-value-increases} provides a graph version of the results showcasing the impact of the value of $c$ on the average fitness of the experiments solving the problem for each population size we are using. The best average for this problem was produced by $G_{75,12}$ with a value of $98108.9092933667$. On the contrary, the worst average was produced by $G_{25,12}$ with a value of $108251.9895637$. The best solution has a fitness value of $84929.672058$, and is generated by $G_{50,8}$. Figure \ref{approach-gwo-mkra30a-best-solution-fitness-over-time} shows its progress over iterations as it solves the mKra30a problem. The worse solution, on the other hand, with a value of $128598.716599$, was produced by $G_{50,12}$. These solutions are visualized by Figure \ref{approach-gwo-mkra30a-best-and-worst-solutions-visualization}. In the same vein as the two aforementioned problems, the average runtime of the experiments increase as the population size increased. As for the values of the average fitness values with respect to the population size and $c$ values, no common trend can be observed for the three different population sizes, unlike with the previous problems. Problems with population sizes of $50$ and $75$ share a common trend from $c = 2$ to $c = 8$, where the increasing the $c$ value first worsens the average fitness, but the fitness then improves. However, when increasing the $c$ value from $8$ to $12$, the behaviour differs. With a population size of $50$, the average fitness worsens. On the other hand, with a population size of $75$, the average fitness improves instead. Lastly, the configuration with a population size of $25$ acts differently from the configurations with the aforementioned population sizes. With a population size of $25$, the average fitness value when increasing the $c$ value from $2$ to $4$ initially shows an improvement of the average fitness. However, increasing the $c$ value further results in worsening average fitness values.

\begin{figure}[h!]
\centering
\begin{adjustwidth}{-0.45in}{}
	\includegraphics[scale=1.0]{./images/chap07-rd/gwo-mkra30a-best-and-worst-solutions.png}
\end{adjustwidth}
\caption{Visualizations of the best and worst solutions generated by our GWO approach for the mKra30a problem. The best solution was generated by $G_{50,8}$, with the worst generated by $G_{50,12}$.}
\label{approach-gwo-mkra30a-best-and-worst-solutions-visualization}
\end{figure}

\begin{figure}[h!]
\centering
\begin{adjustwidth}{-0.45in}{}
	\includegraphics[scale=0.5]{./images/chap07-rd/gwo-only-mkra30a-average-fitness-as-c-value-increases.png}
\end{adjustwidth}
\caption{The average fitness of the solutions produced by our GWO approach as the $c$ value increases when solving the mKra30a problem. Each line uses a different population with the blue line representing experiments using a population size ($n$) of 25, the yellow lines representing those with $n = 50$, and the green lines representing those with $n = 75$.}
\label{approach-gwo-mkra30a-average-fitness-as-c-value-increases}
\end{figure}

\begin{figure}[h!]
\centering
\begin{adjustwidth}{-0.45in}{}
	\includegraphics[scale=0.5]{./images/chap07-rd/gwo-only-mkra30a-best-solution-fitness-graph.png}
\end{adjustwidth}
\caption{The fitness of the best solution generated by $G_{50,8}$ as the number of iterations increasee while solving the mKra30a problem.}
\label{approach-gwo-mkra30a-best-solution-fitness-over-time}
\end{figure}

\begin{table}
\centering
\resizebox{0.5\textwidth}{!}{\rotatebox{90}{
\begin{tabular}{|r|r|r|r|r|r|r|} 
	\hline
	\multicolumn{2}{|c|}{\textbf{Parameters}}                                  & \multicolumn{5}{c|}{\textbf{SFLP-II}}                                                                                                                                                                     \\ 
	\hline
	\multicolumn{1}{|c|}{\textbf{Pop. Size}} & \multicolumn{1}{c|}{\textbf{c}} & \multicolumn{1}{c|}{\textbf{Best}} & \multicolumn{1}{c|}{\textbf{Worst}} & \multicolumn{1}{c|}{\textbf{Avg.}} & \multicolumn{1}{c|}{\textbf{Std. Dev.}} & \multicolumn{1}{c|}{\textbf{Avg. Runtime (s)}}  \\ 
	\hline
	\multirow{4}{*}{25}                      & 2                               & 226.149871                         & 328.546146                          & 287.749326366667                   & 24.7018482581174                        & 6.03333333333333                                \\ 
	\cline{2-7}
	& 4                               & 228.710226                         & 373.604858                          & 298.836421533333                   & 34.1522620177737                        & 5.9                                             \\ 
	\cline{2-7}
	& 8                               & 249.624192                         & 368.435807                          & 313.640670633333                   & 29.7958232730557                        & 6.23333333333333                                \\ 
	\cline{2-7}
	& 12                              & 255.639347                         & 358.033844                          & 315.207093933333                   & 26.4399229476443                        & 6.36666666666667                                \\ 
	\hline
	\multirow{4}{*}{50}                      & 2                               & 221.18019                          & 341.568304                          & 283.795019233333                   & 28.9742518867792                        & 13.2666666666667                                \\ 
	\cline{2-7}
	& 4                               & 241.862298                         & 389.711568                          & 294.2461242                        & 33.7641257773172                        & 13.1333333333333                                \\ 
	\cline{2-7}
	& 8                               & 240.638127                         & 413.077936                          & 299.553292466667                   & 40.469222577263                         & 12.7333333333333                                \\ 
	\cline{2-7}
	& 12                              & 261.869799                         & 381.586061                          & 315.831642166667                   & 31.7204847308938                        & 13.6666666666667                                \\ 
	\hline
	\multirow{4}{*}{75}                      & 2                               & 205.666955                         & 386.356476                          & 290.063388433333                   & 38.0436802516604                        & 18.8666666666667                                \\ 
	\cline{2-7}
	& 4                               & 239.258536                         & 406.790997                          & 302.005947566667                   & 39.1344289013742                        & 18.9                                            \\ 
	\cline{2-7}
	& 8                               & 246.916627                         & 393.744452                          & 309.957982766667                   & 32.0156308267039                        & 19.5                                            \\ 
	\cline{2-7}
	& 12                              & 243.386427                         & 413.874466                          & 307.3213231                        & 36.2239957721235                        & 19.3333333333333                                \\
	\hline
\end{tabular}}}
\caption{Summary of the experiments using the GWO approach with the SFLP-II problem.}
\label{full-data-gwo-sflp-ii}
\end{table}

\begin{table}
\centering
\resizebox{0.5\textwidth}{!}{\rotatebox{90}{
\begin{tabular}{|r|r|r|r|r|r|r|} 
	\hline
	\multicolumn{2}{|c|}{\textbf{Parameters}}                                  & \multicolumn{5}{c|}{\textbf{mSFLP-III}}                                                                                                                                                                   \\ 
	\hline
	\multicolumn{1}{|c|}{\textbf{Pop. Size}} & \multicolumn{1}{c|}{\textbf{c}} & \multicolumn{1}{c|}{\textbf{Best}} & \multicolumn{1}{c|}{\textbf{Worst}} & \multicolumn{1}{c|}{\textbf{Avg.}} & \multicolumn{1}{c|}{\textbf{Std. Dev.}} & \multicolumn{1}{c|}{\textbf{Avg. Runtime (s)}}  \\ 
	\hline
	\multirow{4}{*}{25}                      & 2                               & 50874.238564                       & 60974.998642                        & 55624.0061857667                   & 2544.70550235336                        & 22.4333333333333                                \\ 
	\cline{2-7}
	& 4                               & 50825.278824                       & 59194.486832                        & 55236.4286594                      & 2301.71070299619                        & 21.3333333333333                                \\ 
	\cline{2-7}
	& 8                               & 51250.638187                       & 57894.186882                        & 54206.6467008333                   & 1334.40810225102                        & 23.6                                            \\ 
	\cline{2-7}
	& 12                              & 51328.437737                       & 62758.004044                        & 55331.2312675333                   & 2540.54370041386                        & 21.2333333333333                                \\ 
	\hline
	\multirow{4}{*}{50}                      & 2                               & 47597.794662                       & 62827.738159                        & 54495.2676476667                   & 3328.18744058766                        & 40.1333333333333                                \\ 
	\cline{2-7}
	& 4                               & 50250.080536                       & 57916.673454                        & 53421.9267731333                   & 2239.06725435468                        & 41.9666666666667                                \\ 
	\cline{2-7}
	& 8                               & 48844.175789                       & 59710.615997                        & 52699.5983075667                   & 2062.76885562279                        & 41.2666666666667                                \\ 
	\cline{2-7}
	& 12                              & 48920.979538                       & 56477.689476                        & 52837.3591700333                   & 1980.22102755171                        & 42.2333333333333                                \\ 
	\hline
	\multirow{4}{*}{75}                      & 2                               & 50179.684898                       & 57659.66436                         & 54015.3749653                      & 2050.7167136713                         & 61.1                                            \\ 
	\cline{2-7}
	& 4                               & 48752.443314                       & 56156.624268                        & 52037.6629276                      & 2313.4004317195                         & 61.5333333333333                                \\ 
	\cline{2-7}
	& 8                               & 49276.248596                       & 54977.558044                        & 51801.8837937333                   & 1419.1918023338                         & 63.0333333333333                                \\ 
	\cline{2-7}
	& 12                              & 49644.232903                       & 55524.684891                        & 51837.8766529667                   & 1430.57988385005                        & 61.7666666666667                                \\
	\hline
\end{tabular}}}
\caption{Summary of the experiments using the GWO approach with the mSFLP-III problem.}
\label{full-data-gwo-msflp-iii}
\end{table}

\begin{table}
\centering
\resizebox{0.5\textwidth}{!}{\rotatebox{90}{
\begin{tabular}{|r|r|r|r|r|r|r|} 
	\hline
	\multicolumn{2}{|c|}{\textbf{Parameters}}                                  & \multicolumn{5}{c|}{\textbf{mKra30a}}                                                                                                                                                                     \\ 
	\hline
	\multicolumn{1}{|c|}{\textbf{Pop. Size}} & \multicolumn{1}{c|}{\textbf{c}} & \multicolumn{1}{l|}{\textbf{Best}} & \multicolumn{1}{l|}{\textbf{Worst}} & \multicolumn{1}{l|}{\textbf{Avg.}} & \multicolumn{1}{l|}{\textbf{Std. Dev.}} & \multicolumn{1}{l|}{\textbf{Avg. Runtime (s)}}  \\ 
	\hline
	\multirow{4}{*}{25}                      & 2                               & 94640.759514                       & 120397.403767                       & 107874.742523333                   & 6706.6049593287                         & 35.9666666666667                                \\ 
	\cline{2-7}
	& 4                               & 87110.57618                        & 116746.599121                       & 104270.487286567                   & 8041.05756072353                        & 36.3333333333333                                \\ 
	\cline{2-7}
	& 8                               & 95254.061554                       & 118719.490257                       & 105760.743434367                   & 6557.69877516131                        & 37.0666666666667                                \\ 
	\cline{2-7}
	& 12                              & 93525.765816                       & 124181.531029                       & 108251.9895637                     & 8151.19724950765                        & 37.3                                            \\ 
	\hline
	\multirow{4}{*}{50}                      & 2                               & 88657.824898                       & 121124.59779                        & 102742.1803823                     & 7156.18271087496                        & 73.3333333333333                                \\ 
	\cline{2-7}
	& 4                               & 90599.06601                        & 122300.269909                       & 102855.4831497                     & 8820.10238434929                        & 70.9666666666667                                \\ 
	\cline{2-7}
	& 8                               & 84929.672058                       & 112251.415863                       & 101570.163644533                   & 6490.61277032704                        & 72.1                                            \\ 
	\cline{2-7}
	& 12                              & 92563.720146                       & 128598.716599                       & 105348.4949903                     & 9267.72959691125                        & 69.5666666666667                                \\ 
	\hline
	\multirow{4}{*}{75}                      & 2                               & 88740.484344                       & 117173.305939                       & 99149.0616948                      & 5833.24082935413                        & 104.7                                           \\ 
	\cline{2-7}
	& 4                               & 89197.608078                       & 121878.61042                        & 99482.2352776                      & 7069.6968084522                         & 107.966666666667                                \\ 
	\cline{2-7}
	& 8                               & 87299.715054                       & 113760.078079                       & 98968.6223121                      & 6443.60722715266                        & 108.7                                           \\ 
	\cline{2-7}
	& 12                              & 86942.304199                       & 108422.175175                       & 98108.9092933667                   & 6511.43059062118                        & 111.266666666667                                \\
	\hline
\end{tabular}}}
\caption{Summary of the experiments using the GWO approach with the mKra30a problem.}
\label{full-data-gwo-mkra30a}
\end{table}

Each data set used in the experiments uses differently-sized bounding regions. For SFLP-II, a $12x12$ bounding region is used. mSFLP-III uses a $260x260$ bounding region, while mKra30a uses a $250x250$ one. Notice that, for the SFLP-II problem, configurations using $c = 2$ in each population size produce the best average fitness. For the mSFLP-III problems, regardless of population size, using $c = 8$ produces the best average fitness. This suggests to us that, for certain problems with at least less than 20 buildings, the ideal $c$ value to be used with our GWO approach have some correlation with the size of the bounding box. Smaller $c$ values are more likely to be better for problems with smaller bounding boxes. Similarly, larger $c$ values are more likely to better fit problems with larger bounding boxes. However, too high of a value for $c$ may produce worse results on average. This is shown to us by our results with the mSFLP-III problem, where, regardless of population size, configurations with $c = 8$ consistently perform better on average than those with $c = 12$. We can attribute this behaviour to the fact that the $c$ parameter determines how much a building can be shifted away in the formulas of $D_{\alpha}$, $D_{\beta}$, and $D_{\delta}$ (see equations \ref{summary-modified-gwo-a} to \ref{summary-modified-gwo-xt1} in Methodology). A smaller $c$ value introduces a smaller shift, while a larger value will shift the buildings further. In smaller sized bounding regions, a smaller shift is important due to the limited space available. Larger shifts in such a space will make it harder for buildings to reach feasibility. On the other hand, a larger shift is more appropriate in a larger space since it will allow buildings to move closer to each other faster. Moreover, such a larger amount of available space can be better utilized. A larger space will allow buildings to more easily move away from intersections (and, consequently, infeasibility). However, as shown by our experiments with the mSFLP-III problem, a configuration with the ability to shift too much (e.g. having a $c$ value of $12$) can perform poorer than one with a slightly lower amount of shifting. This is due to the fact that shifts that are too large can push and orient buildings in positions and orientations where, over time, it would become more difficult for them to move towards a position where they are close to other buildings as much as possible yet not intersecting with any of them. Buildings may actually move further away from or move in such a way that hinders them from progressing towards these better positions due to this amount of shifting, contributing to the difficulty. Other causes further exacerbate this behaviour. The gradual reduction of the amount of shifting buildings can do as time progresses (see Equation \ref{summary-modified-gwo-a}, which is the factor for this gradual reduction), makes it harder for buildings to shift towards a superior position and influencing them to only move within a gradually smaller local area. Another contributing factor to the behaviour is that buildings cluster together over time in our approach. This increases the risk of building intersections, especially those building that are nearer to the "inside" of a cluster.

Intriguingly, for the mKra30a problem (which has 30 buildings), the ideal $c$ value seems to have a correlation with the population size used. As the population size increases, the $c$ value must also be increased to be able to produce the best solutions possible. This is in contrary with the trend suggested by the experiments dealing with the first two problems. This may suggest that the parameters of a problem are also factors in determining the ideal value for $c$. However, additional experiments are needed to determine if this behaviour with the mKra30a problem is not a merely quirk caused by the random nature of metaheuristics. If ever it is found out to be an expected behaviour of our GWO approach when dealing with the mKra30a problem or a problem of similar or greater parameters, additional experiments should be able to provide insights as to why our approach has this behaviour.

It should also be mentioned that the average runtime of each experiment setup in each population size category are generally almost equal to each other. However, the experiments dealing with the mKra30a problem with population sizes of $50$ and $75$ have larger deviations from each other. It is of interesting note that for the experiment configurations with a population size of $75$ solving the mKra30a problem, the average runtime increases as the $c$ value increases.

We also observe that the population size has an impact on the performance of our GWO approach. For larger-sized problems with large bounding boxes such as mSFLP-III and mKra30a, our experiments show that, on average, a larger population size is more ideal. We argue that this is due to the larger amount of space the wolves/solutions cover in the abstract search space, which becomes larger as the size of the problem increases. A diverse set of initial solutions resulting from the larger population size allows for this larger amount of covered space. This naturally allows us to more easily find the best possible solution for a problem within a reasonable amount of time compared to with a smaller population size. On the contrary and quite interestingly, for small-sized problems with smaller-sized bounding boxes, such as SFLP-II, our experiments show that the higher population sizes do not necessarily translate to better solutions on average. It is suggested by our experiments that a medium-sized population with the right $c$ value produces the best possible solutions for these small-sized problems compared to using the other two population sizes, with small-sized populations performing surprisingly better than large-sized ones, again with both using the best $c$ value for the population size. This is rather odd given the advantage of having a larger population. It would make sense for experiments with medium-sized populations and the right $c$ value to perform than their small-sized population counterparts. However, it does not immediately make sense why the large-sized population experiments would perform poorly than the small-sized ones. In any case, they should perform better, especially when against the medium-sized population. A possible explanation to this is that a larger population size for small-sized problems may result in achieving local optimum too fast. As per the behaviour of GWO itself, wolves/solutions would gradually cluster around this local optimum until a new better local optimum is found to cluster towards to. We argue that finding another local optimum would be more difficult, since SFLP-II has smaller bounding region size. Intersections between buildings will occur more frequently compared to when using a larger bounding region. Hence, our approach finds it harder to find a better local optimum, resulting in worse average results. Further studies would be needed to gain a better understanding of this behaviour, to confirm our hypothesis, and to determine whether this is simply a quirk of randomness or not.

Later research may also want to focus on determining the ideal $c$ value and population size for a certain problem based on the problem parameters. Figuring out whether the $c$ value can be mathematically modelled rather than being a parameter is another possible avenue for research.

% Start of the GWO table of results for the one with a population of 25.
\begin{table}
\centering
\begin{adjustwidth}{}{}
\resizebox{\textwidth}{!}{\rotatebox{90}{
\begin{tabular}{|r|r|r|r|r|r|r|}
\hline
\multicolumn{1}{|c|}{\multirow{2}{*}{Run}} & \multicolumn{6}{c|}{GWO (c = 2, Pop. Size of 25)}                                                                                                                                                                                      \\ 
\cline{2-7}
\multicolumn{1}{|c|}{}                     & \multicolumn{1}{l|}{SFLP-II} & \multicolumn{1}{l|}{Elapsed Time (s)} & \multicolumn{1}{l|}{mSFLP-III} & \multicolumn{1}{l|}{Elapsed Time (s)} & \multicolumn{1}{l|}{mKra30a} & \multicolumn{1}{l|}{Elapsed Time (s)}  \\ 
\hline
1                                          & 284.274332                   & 5                                     & 54517.212997                   & 20                                    & 113444.272751                & 34                                     \\ 
\hline
2                                          & 279.224484                   & 6                                     & 55100.110451                   & 25                                    & 108634.12896                 & 35                                     \\ 
\hline
3                                          & 290.766824                   & 7                                     & 54424.891159                   & 20                                    & 101057.257011                & 36                                     \\ 
\hline
4                                          & 255.89164                    & 6                                     & 52281.025963                   & 21                                    & 106933.341133                & 36                                     \\ 
\hline
5                                          & 303.41135                    & 6                                     & 56333.281975                   & 24                                    & 110198.061668                & 37                                     \\ 
\hline
6                                          & 307.247426                   & 6                                     & 56191.257446                   & 23                                    & 103150.565979                & 35                                     \\ 
\hline
7                                          & 297.32725                    & 6                                     & 57228.114368                   & 26                                    & 100159.379417                & 34                                     \\ 
\hline
8                                          & 298.901143                   & 6                                     & 54501.928665                   & 25                                    & 94640.759514                 & 36                                     \\ 
\hline
9                                          & 290.924892                   & 6                                     & 60974.998642                   & 22                                    & 101369.649643                & 35                                     \\ 
\hline
10                                         & 328.546146                   & 5                                     & 52108.460861                   & 23                                    & 114400.855766                & 34                                     \\ 
\hline
11                                         & 310.501635                   & 6                                     & 55957.510098                   & 21                                    & 108862.321945                & 39                                     \\ 
\hline
12                                         & 260.87289                    & 5                                     & 55527.02594                    & 24                                    & 106595.717178                & 39                                     \\ 
\hline
13                                         & 282.531157                   & 5                                     & 58757.132118                   & 20                                    & 115065.923737                & 34                                     \\ 
\hline
14                                         & 260.606181                   & 7                                     & 60825.496319                   & 22                                    & 109135.383606                & 35                                     \\ 
\hline
15                                         & 315.86567                    & 6                                     & 52600.283459                   & 20                                    & 120397.403767                & 35                                     \\ 
\hline
16                                         & 277.906986                   & 7                                     & 59222.901581                   & 24                                    & 114673.857002                & 39                                     \\ 
\hline
17                                         & 313.385026                   & 6                                     & 53981.477821                   & 24                                    & 96989.450607                 & 36                                     \\ 
\hline
18                                         & 289.163818                   & 6                                     & 57612.964134                   & 23                                    & 104600.307762                & 37                                     \\ 
\hline
19                                         & 275.85824                    & 5                                     & 58171.204727                   & 20                                    & 104245.289139                & 36                                     \\ 
\hline
20                                         & 317.423512                   & 7                                     & 57049.199783                   & 20                                    & 114823.94133                 & 36                                     \\ 
\hline
21                                         & 288.800416                   & 6                                     & 52996.785149                   & 23                                    & 108857.258942                & 34                                     \\ 
\hline
22                                         & 311.202401                   & 6                                     & 54449.246704                   & 23                                    & 102069.259422                & 36                                     \\ 
\hline
23                                         & 226.149871                   & 7                                     & 53806.471344                   & 21                                    & 95024.52264                  & 35                                     \\ 
\hline
24                                         & 258.801216                   & 5                                     & 58588.215469                   & 22                                    & 113719.784393                & 36                                     \\ 
\hline
25                                         & 268.787345                   & 7                                     & 53098.863617                   & 22                                    & 105550.2397                  & 36                                     \\ 
\hline
26                                         & 290.496099                   & 6                                     & 50874.238564                   & 21                                    & 108321.213905                & 35                                     \\ 
\hline
27                                         & 318.591564                   & 7                                     & 54556.366089                   & 25                                    & 110759.756439                & 35                                     \\ 
\hline
28                                         & 253.100093                   & 6                                     & 55594.022305                   & 22                                    & 119703.695831                & 37                                     \\ 
\hline
29                                         & 257.999482                   & 6                                     & 54408.740974                   & 25                                    & 108653.213264                & 39                                     \\ 
\hline
30                                         & 317.920702                   & 6                                     & 56980.756851                   & 22                                    & 114205.463249                & 38                                     \\ 
\hline
\multicolumn{1}{|l|}{Average}              & 287.749326366667             & 6.03333333333333                      & 55624.0061857667               & 22.4333333333333                      & 107874.742523333             & 35.9666666666667                       \\ 
\hline
\multicolumn{1}{|l|}{Std. Dev}             & 24.7018482581174             & 0.668675135459372                     & 2544.70550235336               & 1.81342376380328                      & 6706.6049593287              & 1.56432938883779                       \\
\hline
\end{tabular}}}
\end{adjustwidth}
\caption{The entire experiment data we have collected using our GWO approach with $c = 2$ and a population of $25$.}
\label{full-data-gwo-c2-p25}
\end{table}

\begin{table}
\centering
\begin{adjustwidth}{}{}
\resizebox{\textwidth}{!}{\rotatebox{90}{
\begin{tabular}{|r|r|r|r|r|r|r|}
\hline
\multicolumn{1}{|c|}{\multirow{2}{*}{Run}} & \multicolumn{6}{c|}{GWO (c = 4, Pop. Size of 25)}                                                                                                                                                                     \\ 
\cline{2-7}
\multicolumn{1}{|c|}{}                     & \multicolumn{1}{l|}{SFLP-II} & \multicolumn{1}{l|}{Elapsed Time (s)} & \multicolumn{1}{l|}{mSFLP-III} & \multicolumn{1}{l|}{Elapsed Time (s)} & \multicolumn{1}{l|}{mKra30a} & \multicolumn{1}{l|}{Elapsed Time (s)}  \\ 
\hline
1                                          & 295.290119                   & 5                                     & 57590.204044                   & 21                                    & 99250.338013                 & 36                                     \\ 
\hline
2                                          & 364.30381                    & 6                                     & 55786.4907                     & 19                                    & 115392.5942                  & 36                                     \\ 
\hline
3                                          & 283.958384                   & 6                                     & 56478.70369                    & 19                                    & 100622.106056                & 34                                     \\ 
\hline
4                                          & 308.185705                   & 6                                     & 59149.522892                   & 21                                    & 100370.279572                & 37                                     \\ 
\hline
5                                          & 279.100698                   & 5                                     & 52092.835846                   & 24                                    & 109341.424805                & 40                                     \\ 
\hline
6                                          & 254.694568                   & 7                                     & 54517.030029                   & 23                                    & 111078.602165                & 34                                     \\ 
\hline
7                                          & 299.122021                   & 5                                     & 57337.25061                    & 25                                    & 87110.57618                  & 38                                     \\ 
\hline
8                                          & 308.245055                   & 5                                     & 56083.913353                   & 21                                    & 116746.599121                & 36                                     \\ 
\hline
9                                          & 310.631233                   & 6                                     & 56986.768219                   & 19                                    & 112465.605843                & 37                                     \\ 
\hline
10                                         & 293.218722                   & 6                                     & 53868.101875                   & 21                                    & 110742.670616                & 37                                     \\ 
\hline
11                                         & 290.615551                   & 7                                     & 55411.816826                   & 21                                    & 102156.222603                & 39                                     \\ 
\hline
12                                         & 310.314302                   & 6                                     & 50825.278824                   & 24                                    & 112943.161697                & 34                                     \\ 
\hline
13                                         & 284.625909                   & 6                                     & 51448.845734                   & 21                                    & 96147.747742                 & 34                                     \\ 
\hline
14                                         & 330.488841                   & 6                                     & 57670.047066                   & 22                                    & 106707.319443                & 37                                     \\ 
\hline
15                                         & 271.799747                   & 6                                     & 57224.530266                   & 21                                    & 106312.161995                & 41                                     \\ 
\hline
16                                         & 308.069251                   & 7                                     & 53504.969574                   & 21                                    & 109508.693184                & 39                                     \\ 
\hline
17                                         & 307.75226                    & 6                                     & 54594.552979                   & 22                                    & 98866.16114                  & 34                                     \\ 
\hline
18                                         & 241.580049                   & 6                                     & 56112.939285                   & 19                                    & 103499.234871                & 33                                     \\ 
\hline
19                                         & 228.710226                   & 6                                     & 56761.539284                   & 23                                    & 88510.688866                 & 35                                     \\ 
\hline
20                                         & 291.547085                   & 6                                     & 56455.574303                   & 22                                    & 97130.734337                 & 38                                     \\ 
\hline
21                                         & 373.604858                   & 5                                     & 59194.486832                   & 23                                    & 115666.25502                 & 32                                     \\ 
\hline
22                                         & 288.904785                   & 7                                     & 51962.636154                   & 20                                    & 100993.999092                & 37                                     \\ 
\hline
23                                         & 351.234066                   & 5                                     & 54615.561333                   & 20                                    & 108507.541492                & 35                                     \\ 
\hline
24                                         & 271.269831                   & 6                                     & 52326.159271                   & 20                                    & 102196.612885                & 35                                     \\ 
\hline
25                                         & 345.786381                   & 7                                     & 55869.035088                   & 21                                    & 106251.775101                & 40                                     \\ 
\hline
26                                         & 251.753436                   & 6                                     & 53282.332359                   & 20                                    & 94795.74794                  & 36                                     \\ 
\hline
27                                         & 316.554537                   & 5                                     & 53018.602661                   & 22                                    & 110122.478073                & 36                                     \\ 
\hline
28                                         & 321.758455                   & 5                                     & 58369.596535                   & 20                                    & 94860.329987                 & 42                                     \\ 
\hline
29                                         & 263.652587                   & 6                                     & 55446.598206                   & 21                                    & 95583.080429                 & 33                                     \\ 
\hline
30                                         & 318.320174                   & 6                                     & 53106.935944                   & 24                                    & 114233.876129                & 35                                     \\ 
\hline
\multicolumn{1}{|l|}{Average}              & 298.836421533333             & 5.9                                   & 55236.4286594                  & 21.3333333333333                      & 104270.487286567             & 36.3333333333333                       \\ 
\hline
\multicolumn{1}{|l|}{Std. Dev}             & 34.1522620177737             & 0.661763578993857                     & 2301.71070299619               & 1.62593916273628                      & 8041.05756072353             & 2.48211996894386                       \\
\hline
\end{tabular}}}
\end{adjustwidth}
\caption{The entire experiment data we have collected using our GWO approach with $c = 4$ and a population of $25$.}
\label{full-data-gwo-c4-p25}
\end{table}

\begin{table}
\centering
\begin{adjustwidth}{}{}
\resizebox{\textwidth}{!}{\rotatebox{90}{
\begin{tabular}{|r|r|r|r|r|r|r|}
\hline
\multicolumn{1}{|c|}{\multirow{2}{*}{Run}} & \multicolumn{6}{c|}{GWO (c = 8, Pop. Size of 25)}                                                                                                                                                                     \\ 
\cline{2-7}
\multicolumn{1}{|c|}{}                     & \multicolumn{1}{l|}{SFLP-II} & \multicolumn{1}{l|}{Elapsed Time (s)} & \multicolumn{1}{l|}{mSFLP-III} & \multicolumn{1}{l|}{Elapsed Time (s)} & \multicolumn{1}{l|}{mKra30a} & \multicolumn{1}{l|}{Elapsed Time (s)}  \\ 
\hline
1                                          & 262.707255                   & 7                                     & 55466.377586                   & 20                                    & 104697.724777                & 37                                     \\ 
\hline
2                                          & 311.009165                   & 6                                     & 52879.782005                   & 20                                    & 114784.794098                & 41                                     \\ 
\hline
3                                          & 291.360188                   & 6                                     & 54291.537682                   & 22                                    & 106627.593742                & 34                                     \\ 
\hline
4                                          & 362.348913                   & 5                                     & 51966.878838                   & 22                                    & 118225.260605                & 36                                     \\ 
\hline
5                                          & 305.725908                   & 7                                     & 57894.186882                   & 22                                    & 107491.358063                & 37                                     \\ 
\hline
6                                          & 344.391756                   & 5                                     & 54056.77681                    & 24                                    & 118719.490257                & 36                                     \\ 
\hline
7                                          & 337.822041                   & 6                                     & 53829.962364                   & 24                                    & 100109.010674                & 33                                     \\ 
\hline
8                                          & 319.080408                   & 7                                     & 53949.482574                   & 23                                    & 96509.502808                 & 38                                     \\ 
\hline
9                                          & 249.624192                   & 7                                     & 55062.295784                   & 22                                    & 95254.061554                 & 38                                     \\ 
\hline
10                                         & 317.502579                   & 8                                     & 54612.372421                   & 22                                    & 111357.790108                & 34                                     \\ 
\hline
11                                         & 347.870933                   & 6                                     & 55273.090096                   & 25                                    & 99750.327438                 & 37                                     \\ 
\hline
12                                         & 294.693992                   & 7                                     & 54738.413105                   & 25                                    & 106767.40667                 & 35                                     \\ 
\hline
13                                         & 282.647407                   & 7                                     & 54133.133018                   & 25                                    & 111462.286797                & 34                                     \\ 
\hline
14                                         & 292.732835                   & 6                                     & 54017.253056                   & 24                                    & 107217.431229                & 39                                     \\ 
\hline
15                                         & 283.872915                   & 5                                     & 53602.699944                   & 26                                    & 106198.506428                & 44                                     \\ 
\hline
16                                         & 321.749494                   & 7                                     & 54680.752884                   & 23                                    & 97637.581284                 & 36                                     \\ 
\hline
17                                         & 310.098661                   & 5                                     & 53654.185509                   & 23                                    & 97815.004234                 & 34                                     \\ 
\hline
18                                         & 292.552563                   & 5                                     & 54514.880905                   & 26                                    & 104283.744904                & 39                                     \\ 
\hline
19                                         & 330.776446                   & 6                                     & 54886.626793                   & 23                                    & 108294.5578                  & 47                                     \\ 
\hline
20                                         & 368.435807                   & 5                                     & 51250.638187                   & 25                                    & 101209.694763                & 35                                     \\ 
\hline
21                                         & 333.208825                   & 5                                     & 53291.312008                   & 22                                    & 104063.487877                & 39                                     \\ 
\hline
22                                         & 307.813316                   & 6                                     & 54018.854462                   & 26                                    & 106063.043762                & 34                                     \\ 
\hline
23                                         & 326.428337                   & 6                                     & 55304.592007                   & 26                                    & 99203.966431                 & 37                                     \\ 
\hline
24                                         & 358.536934                   & 5                                     & 56567.198547                   & 25                                    & 114660.357567                & 36                                     \\ 
\hline
25                                         & 299.608566                   & 6                                     & 52027.762772                   & 26                                    & 101201.104996                & 38                                     \\ 
\hline
26                                         & 288.695351                   & 8                                     & 55043.384789                   & 23                                    & 100895.692963                & 38                                     \\ 
\hline
27                                         & 320.946527                   & 7                                     & 54383.856281                   & 22                                    & 110844.852486                & 40                                     \\ 
\hline
28                                         & 363.365827                   & 9                                     & 53825.792953                   & 23                                    & 104716.713264                & 33                                     \\ 
\hline
29                                         & 289.365061                   & 6                                     & 54623.883316                   & 25                                    & 100086.231628                & 35                                     \\ 
\hline
30                                         & 294.247917                   & 6                                     & 52351.437447                   & 24                                    & 116673.723824                & 38                                     \\ 
\hline
\multicolumn{1}{|l|}{Average}              & 313.640670633333             & 6.23333333333333                      & 54206.6467008333               & 23.6                                  & 105760.743434367             & 37.0666666666667                       \\ 
\hline
\multicolumn{1}{|l|}{Std. Dev}             & 29.7958232730557             & 1.04000442085709                      & 1334.40810225102               & 1.7340405277052                       & 6557.69877516131             & 3.12865877767558                       \\
\hline
\end{tabular}}}
\end{adjustwidth}
\caption{The entire experiment data we have collected using our GWO approach with $c = 8$ and a population of $25$.}
\label{full-data-gwo-c8-p25}
\end{table}

\begin{table}
\centering
\begin{adjustwidth}{}{}
\resizebox{\textwidth}{!}{\rotatebox{90}{
\begin{tabular}{|r|r|r|r|r|r|r|}
\hline
\multicolumn{1}{|c|}{\multirow{2}{*}{Run}} & \multicolumn{6}{c|}{GWO (c = 12, Pop. Size of 25)}                                                                                                                                                                    \\ 
\cline{2-7}
\multicolumn{1}{|c|}{}                     & \multicolumn{1}{l|}{SFLP-II} & \multicolumn{1}{l|}{Elapsed Time (s)} & \multicolumn{1}{l|}{mSFLP-III} & \multicolumn{1}{l|}{Elapsed Time (s)} & \multicolumn{1}{l|}{mKra30a} & \multicolumn{1}{l|}{Elapsed Time (s)}  \\ 
\hline
1                                          & 280.100084                   & 6                                     & 54382.193893                   & 22                                    & 110420.231773                & 35                                     \\ 
\hline
2                                          & 326.000024                   & 6                                     & 56657.682411                   & 19                                    & 107836.336555                & 39                                     \\ 
\hline
3                                          & 339.59429                    & 6                                     & 56128.932846                   & 20                                    & 105101.680847                & 34                                     \\ 
\hline
4                                          & 291.153993                   & 6                                     & 56149.892471                   & 22                                    & 99260.377151                 & 36                                     \\ 
\hline
5                                          & 324.951407                   & 6                                     & 57060.862228                   & 22                                    & 97755.446457                 & 39                                     \\ 
\hline
6                                          & 322.430838                   & 6                                     & 62758.004044                   & 20                                    & 114021.624924                & 38                                     \\ 
\hline
7                                          & 325.058099                   & 6                                     & 54246.820145                   & 20                                    & 100348.282265                & 36                                     \\ 
\hline
8                                          & 301.274588                   & 6                                     & 54187.122147                   & 20                                    & 97531.603271                 & 40                                     \\ 
\hline
9                                          & 341.434416                   & 8                                     & 53573.432259                   & 19                                    & 108106.979767                & 37                                     \\ 
\hline
10                                         & 357.480109                   & 7                                     & 56897.296356                   & 23                                    & 116657.657318                & 37                                     \\ 
\hline
11                                         & 345.915623                   & 7                                     & 51811.845238                   & 21                                    & 109709.232689                & 33                                     \\ 
\hline
12                                         & 321.136856                   & 6                                     & 53061.549057                   & 24                                    & 100745.31324                 & 40                                     \\ 
\hline
13                                         & 327.757973                   & 6                                     & 52248.577141                   & 24                                    & 107016.641022                & 36                                     \\ 
\hline
14                                         & 338.574615                   & 7                                     & 55548.896332                   & 26                                    & 105929.138687                & 39                                     \\ 
\hline
15                                         & 354.648388                   & 6                                     & 53266.573128                   & 23                                    & 110405.852989                & 36                                     \\ 
\hline
16                                         & 305.538115                   & 7                                     & 54944.865646                   & 22                                    & 121917.463058                & 36                                     \\ 
\hline
17                                         & 274.450367                   & 6                                     & 52603.816597                   & 20                                    & 118235.825241                & 37                                     \\ 
\hline
18                                         & 299.440072                   & 7                                     & 54888.153915                   & 22                                    & 116887.253868                & 36                                     \\ 
\hline
19                                         & 320.987788                   & 8                                     & 56474.669212                   & 20                                    & 102034.555359                & 37                                     \\ 
\hline
20                                         & 317.335714                   & 7                                     & 59091.504066                   & 18                                    & 93525.765816                 & 38                                     \\ 
\hline
21                                         & 358.033844                   & 5                                     & 53225.246391                   & 20                                    & 105195.959213                & 37                                     \\ 
\hline
22                                         & 296.381484                   & 6                                     & 56367.015659                   & 22                                    & 107206.72065                 & 39                                     \\ 
\hline
23                                         & 340.041138                   & 6                                     & 55241.328114                   & 21                                    & 122687.700439                & 39                                     \\ 
\hline
24                                         & 307.116007                   & 7                                     & 54713.23642                    & 20                                    & 105925.402702                & 37                                     \\ 
\hline
25                                         & 255.639347                   & 7                                     & 58269.998817                   & 20                                    & 124181.531029                & 35                                     \\ 
\hline
26                                         & 296.414082                   & 5                                     & 56489.390518                   & 18                                    & 99701.346191                 & 36                                     \\ 
\hline
27                                         & 264.475464                   & 6                                     & 51328.437737                   & 22                                    & 98267.529518                 & 40                                     \\ 
\hline
28                                         & 304.718431                   & 6                                     & 60483.979401                   & 26                                    & 116802.36039                 & 36                                     \\ 
\hline
29                                         & 305.344853                   & 6                                     & 53979.122162                   & 20                                    & 112001.866364                & 38                                     \\ 
\hline
30                                         & 312.784809                   & 7                                     & 53856.493675                   & 21                                    & 112142.008118                & 43                                     \\ 
\hline
\multicolumn{1}{|l|}{Average}              & 315.207093933333             & 6.36666666666667                      & 55331.2312675333               & 21.2333333333333                      & 108251.9895637               & 37.3                                   \\ 
\hline
\multicolumn{1}{|l|}{Std. Dev}             & 26.4399229476443             & 0.718395402284138                     & 2540.54370041386               & 2.01174711054387                      & 8151.19724950765             & 2.07031565142896                       \\
\hline
\end{tabular}}}
\end{adjustwidth}
\caption{The entire experiment data we have collected using our GWO approach with $c = 12$ and a population of $25$.}
\label{full-data-gwo-c12-p25}
\end{table}

% Start of the GWO table of results for the one with a population of 50.
\begin{table}
	\centering
	\begin{adjustwidth}{}{}
		\resizebox{\textwidth}{!}{\rotatebox{90}{
				\begin{tabular}{|r|r|r|r|r|r|r|}
					\hline
					\multicolumn{1}{|c|}{\multirow{2}{*}{Run}} & \multicolumn{6}{c|}{GWO (c = 2, Pop. 50)}                                                                                                                                                                             \\ 
					\cline{2-7}
					\multicolumn{1}{|c|}{}                     & \multicolumn{1}{l|}{SFLP-II} & \multicolumn{1}{l|}{Elapsed Time (s)} & \multicolumn{1}{l|}{mSFLP-III} & \multicolumn{1}{l|}{Elapsed Time (s)} & \multicolumn{1}{l|}{mKra30a} & \multicolumn{1}{l|}{Elapsed Time (s)}  \\ 
					\hline
					1                                          & 252.959127                   & 13                                    & 54847.391106                   & 39                                    & 115453.917747                & 71                                     \\ 
					\hline
					2                                          & 310.451593                   & 13                                    & 53073.76783                    & 40                                    & 105585.975105                & 71                                     \\ 
					\hline
					3                                          & 324.7657                     & 14                                    & 52564.010193                   & 39                                    & 96569.906059                 & 73                                     \\ 
					\hline
					4                                          & 302.253564                   & 12                                    & 62827.738159                   & 42                                    & 88657.824898                 & 69                                     \\ 
					\hline
					5                                          & 287.59426                    & 15                                    & 50270.968304                   & 38                                    & 99648.274292                 & 72                                     \\ 
					\hline
					6                                          & 246.890955                   & 12                                    & 51021.870609                   & 39                                    & 121124.59779                 & 68                                     \\ 
					\hline
					7                                          & 329.037109                   & 11                                    & 53698.933483                   & 39                                    & 109386.855927                & 74                                     \\ 
					\hline
					8                                          & 270.753061                   & 12                                    & 48938.991478                   & 37                                    & 100284.722839                & 68                                     \\ 
					\hline
					9                                          & 297.364378                   & 14                                    & 51457.838638                   & 39                                    & 102671.318954                & 72                                     \\ 
					\hline
					10                                         & 297.396819                   & 14                                    & 55029.144333                   & 40                                    & 100138.233009                & 72                                     \\ 
					\hline
					11                                         & 264.955059                   & 13                                    & 54034.453362                   & 44                                    & 94063.746262                 & 78                                     \\ 
					\hline
					12                                         & 239.019053                   & 13                                    & 54843.412094                   & 39                                    & 107732.370728                & 76                                     \\ 
					\hline
					13                                         & 269.378189                   & 15                                    & 60974.241631                   & 40                                    & 100197.140274                & 78                                     \\ 
					\hline
					14                                         & 270.275415                   & 13                                    & 53123.451248                   & 41                                    & 112261.793114                & 75                                     \\ 
					\hline
					15                                         & 245.398331                   & 13                                    & 56983.71854                    & 42                                    & 112999.618805                & 72                                     \\ 
					\hline
					16                                         & 288.427367                   & 13                                    & 55207.566151                   & 43                                    & 105895.490013                & 74                                     \\ 
					\hline
					17                                         & 233.585352                   & 13                                    & 57260.047424                   & 39                                    & 98834.100918                 & 75                                     \\ 
					\hline
					18                                         & 280.347926                   & 16                                    & 58842.573586                   & 39                                    & 101081.708069                & 76                                     \\ 
					\hline
					19                                         & 307.761939                   & 14                                    & 52738.607117                   & 39                                    & 99121.191231                 & 73                                     \\ 
					\hline
					20                                         & 284.018757                   & 13                                    & 53894.438091                   & 42                                    & 99270.817657                 & 75                                     \\ 
					\hline
					21                                         & 280.025665                   & 14                                    & 53150.059753                   & 39                                    & 113995.070282                & 72                                     \\ 
					\hline
					22                                         & 281.777046                   & 13                                    & 47597.794662                   & 42                                    & 99077.873138                 & 73                                     \\ 
					\hline
					23                                         & 221.18019                    & 14                                    & 53971.207901                   & 39                                    & 103581.175766                & 81                                     \\ 
					\hline
					24                                         & 308.542643                   & 14                                    & 53825.842094                   & 38                                    & 100780.563087                & 69                                     \\ 
					\hline
					25                                         & 302.184198                   & 13                                    & 59349.959862                   & 40                                    & 94936.180672                 & 73                                     \\ 
					\hline
					26                                         & 293.481107                   & 12                                    & 56370.238155                   & 43                                    & 97659.466835                 & 69                                     \\ 
					\hline
					27                                         & 300.88667                    & 13                                    & 51595.317627                   & 43                                    & 93650.395554                 & 76                                     \\ 
					\hline
					28                                         & 308.646566                   & 13                                    & 57071.090172                   & 37                                    & 103402.549973                & 74                                     \\ 
					\hline
					29                                         & 272.924234                   & 13                                    & 55864.520546                   & 41                                    & 103434.275482                & 77                                     \\ 
					\hline
					30                                         & 341.568304                   & 13                                    & 54428.835281                   & 42                                    & 100768.256989                & 74                                     \\ 
					\hline
					\multicolumn{1}{|l|}{Average}              & 283.795019233333             & 13.2666666666667                      & 54495.2676476667               & 40.1333333333333                      & 102742.1803823               & 73.3333333333333                       \\ 
					\hline
					\multicolumn{1}{|l|}{Std. Dev}             & 28.9742518867792             & 1.01483252680985                      & 3328.18744058766               & 1.87052147133163                      & 7156.18271087496             & 3.110974271758                         \\
					\hline
		\end{tabular}}}
	\end{adjustwidth}
	\caption{The entire experiment data we have collected using our GWO approach with $c = 2$ and a population of $50$.}
	\label{full-data-gwo-c2-p50}
\end{table}

\begin{table}
	\centering
	\begin{adjustwidth}{}{}
		\resizebox{\textwidth}{!}{\rotatebox{90}{
				\begin{tabular}{|r|r|r|r|r|r|r|}
					\hline
					\multicolumn{1}{|c|}{\multirow{2}{*}{Run}} & \multicolumn{6}{c|}{GWO (c = 4, Pop. 50)}                                                                                                                                                                             \\ 
					\cline{2-7}
					\multicolumn{1}{|c|}{}                     & \multicolumn{1}{l|}{SFLP-II} & \multicolumn{1}{l|}{Elapsed Time (s)} & \multicolumn{1}{l|}{mSFLP-III} & \multicolumn{1}{l|}{Elapsed Time (s)} & \multicolumn{1}{l|}{mKra30a} & \multicolumn{1}{l|}{Elapsed Time (s)}  \\ 
					\hline
					1                                          & 312.340316                   & 12                                    & 53062.125809                   & 43                                    & 91596.268288                 & 75                                     \\ 
					\hline
					2                                          & 292.071671                   & 14                                    & 51271.728905                   & 41                                    & 118655.595436                & 74                                     \\ 
					\hline
					3                                          & 329.653256                   & 12                                    & 55060.696541                   & 41                                    & 102449.075523                & 72                                     \\ 
					\hline
					4                                          & 306.763128                   & 16                                    & 57916.673454                   & 43                                    & 104931.357796                & 71                                     \\ 
					\hline
					5                                          & 252.772092                   & 12                                    & 50881.148052                   & 45                                    & 122300.269909                & 79                                     \\ 
					\hline
					6                                          & 351.977179                   & 14                                    & 53539.74173                    & 42                                    & 93781.166649                 & 67                                     \\ 
					\hline
					7                                          & 330.62476                    & 15                                    & 54803.86525                    & 38                                    & 108907.169487                & 71                                     \\ 
					\hline
					8                                          & 389.711568                   & 11                                    & 55295.007942                   & 39                                    & 95312.145584                 & 68                                     \\ 
					\hline
					9                                          & 297.27606                    & 13                                    & 53283.531433                   & 38                                    & 90647.4767                   & 67                                     \\ 
					\hline
					10                                         & 277.617715                   & 14                                    & 54563.107239                   & 40                                    & 93732.829643                 & 78                                     \\ 
					\hline
					11                                         & 266.562111                   & 12                                    & 52247.270363                   & 43                                    & 109087.634438                & 72                                     \\ 
					\hline
					12                                         & 300.58802                    & 13                                    & 53049.529968                   & 43                                    & 95105.831413                 & 71                                     \\ 
					\hline
					13                                         & 263.685979                   & 14                                    & 56017.20726                    & 44                                    & 110646.846546                & 69                                     \\ 
					\hline
					14                                         & 301.834932                   & 12                                    & 50960.144783                   & 37                                    & 101672.54467                 & 72                                     \\ 
					\hline
					15                                         & 295.790956                   & 15                                    & 54146.704651                   & 38                                    & 116249.322273                & 74                                     \\ 
					\hline
					16                                         & 268.800616                   & 13                                    & 50250.080536                   & 42                                    & 90599.06601                  & 70                                     \\ 
					\hline
					17                                         & 287.1534                     & 12                                    & 51163.562675                   & 39                                    & 112135.693672                & 69                                     \\ 
					\hline
					18                                         & 294.190074                   & 12                                    & 56055.587646                   & 40                                    & 102759.575096                & 70                                     \\ 
					\hline
					19                                         & 257.069981                   & 12                                    & 52603.173042                   & 43                                    & 99055.68399                  & 76                                     \\ 
					\hline
					20                                         & 247.273698                   & 13                                    & 54079.614361                   & 38                                    & 106487.916885                & 73                                     \\ 
					\hline
					21                                         & 241.862298                   & 13                                    & 54716.540192                   & 44                                    & 114099.514786                & 70                                     \\ 
					\hline
					22                                         & 314.056355                   & 12                                    & 51900.551003                   & 46                                    & 101358.406052                & 68                                     \\ 
					\hline
					23                                         & 304.7226                     & 14                                    & 50370.735947                   & 43                                    & 91041.143768                 & 69                                     \\ 
					\hline
					24                                         & 325.064058                   & 12                                    & 51801.091309                   & 45                                    & 95614.268036                 & 67                                     \\ 
					\hline
					25                                         & 298.000466                   & 13                                    & 56852.344414                   & 46                                    & 102250.452179                & 70                                     \\ 
					\hline
					26                                         & 243.562194                   & 15                                    & 54890.816574                   & 45                                    & 111223.215179                & 70                                     \\ 
					\hline
					27                                         & 275.810019                   & 12                                    & 57815.042976                   & 45                                    & 104611.12146                 & 71                                     \\ 
					\hline
					28                                         & 286.811285                   & 14                                    & 50459.697983                   & 39                                    & 93398.191818                 & 67                                     \\ 
					\hline
					29                                         & 338.195192                   & 15                                    & 53087.217148                   & 47                                    & 102300.34594                 & 70                                     \\ 
					\hline
					30                                         & 275.541747                   & 13                                    & 50513.264008                   & 42                                    & 103654.365265                & 69                                     \\ 
					\hline
					\multicolumn{1}{|l|}{Average}              & 294.2461242                  & 13.1333333333333                      & 53421.9267731333               & 41.9666666666667                      & 102855.4831497               & 70.9666666666667                       \\ 
					\hline
					\multicolumn{1}{|l|}{Std. Dev}             & 33.7641257773172             & 1.25212463115859                      & 2239.06725435468               & 2.83431355593084                      & 8820.10238434929             & 3.12369512986908                       \\
					\hline
		\end{tabular}}}
	\end{adjustwidth}
	\caption{The entire experiment data we have collected using our GWO approach with $c = 4$ and a population of $50$.}
	\label{full-data-gwo-c4-p50}
\end{table}

\begin{table}
	\centering
	\begin{adjustwidth}{}{}
		\resizebox{\textwidth}{!}{\rotatebox{90}{
				\begin{tabular}{|r|r|r|r|r|r|r|}
					\hline
					\multicolumn{1}{|c|}{\multirow{2}{*}{Run}} & \multicolumn{6}{c|}{GWO (c = 8, Pop. 50)}                                                                                                                                                                             \\ 
					\cline{2-7}
					\multicolumn{1}{|c|}{}                     & \multicolumn{1}{l|}{SFLP-II} & \multicolumn{1}{l|}{Elapsed Time (s)} & \multicolumn{1}{l|}{mSFLP-III} & \multicolumn{1}{l|}{Elapsed Time (s)} & \multicolumn{1}{l|}{mKra30a} & \multicolumn{1}{l|}{Elapsed Time (s)}  \\ 
					\hline
					1                                          & 259.516314                   & 12                                    & 54177.974075                   & 38                                    & 94851.361023                 & 69                                     \\ 
					\hline
					2                                          & 324.636963                   & 15                                    & 52181.587135                   & 41                                    & 108371.091469                & 75                                     \\ 
					\hline
					3                                          & 269.794718                   & 13                                    & 59710.615997                   & 39                                    & 105659.087982                & 72                                     \\ 
					\hline
					4                                          & 242.578002                   & 13                                    & 54392.617035                   & 41                                    & 95788.281204                 & 71                                     \\ 
					\hline
					5                                          & 277.550035                   & 13                                    & 51893.418335                   & 36                                    & 104205.210045                & 70                                     \\ 
					\hline
					6                                          & 268.449108                   & 12                                    & 52690.644062                   & 39                                    & 108677.065521                & 72                                     \\ 
					\hline
					7                                          & 295.814127                   & 12                                    & 51080.219994                   & 40                                    & 94337.464722                 & 79                                     \\ 
					\hline
					8                                          & 302.657652                   & 11                                    & 52444.058189                   & 40                                    & 98890.396248                 & 77                                     \\ 
					\hline
					9                                          & 251.909466                   & 13                                    & 48844.175789                   & 44                                    & 103281.15654                 & 67                                     \\ 
					\hline
					10                                         & 310.410302                   & 13                                    & 52941.323196                   & 42                                    & 94691.424179                 & 69                                     \\ 
					\hline
					11                                         & 300.186725                   & 13                                    & 50015.423599                   & 40                                    & 111466.642097                & 70                                     \\ 
					\hline
					12                                         & 271.950485                   & 12                                    & 53271.293297                   & 41                                    & 100257.817604                & 76                                     \\ 
					\hline
					13                                         & 328.766632                   & 12                                    & 50271.230789                   & 42                                    & 103823.807419                & 69                                     \\ 
					\hline
					14                                         & 318.317183                   & 12                                    & 52223.022621                   & 44                                    & 92512.471001                 & 73                                     \\ 
					\hline
					15                                         & 286.950144                   & 12                                    & 52495.684402                   & 41                                    & 104429.74781                 & 70                                     \\ 
					\hline
					16                                         & 319.746528                   & 13                                    & 53486.564278                   & 40                                    & 107520.309296                & 77                                     \\ 
					\hline
					17                                         & 401.791988                   & 15                                    & 52635.489326                   & 44                                    & 104163.873627                & 68                                     \\ 
					\hline
					18                                         & 269.226305                   & 14                                    & 54915.506729                   & 40                                    & 100961.08773                 & 71                                     \\ 
					\hline
					19                                         & 278.087246                   & 13                                    & 52754.325737                   & 41                                    & 98450.999786                 & 75                                     \\ 
					\hline
					20                                         & 240.638127                   & 12                                    & 52209.049438                   & 42                                    & 109202.223549                & 72                                     \\ 
					\hline
					21                                         & 329.440071                   & 12                                    & 50330.414803                   & 40                                    & 108166.858864                & 70                                     \\ 
					\hline
					22                                         & 280.454088                   & 12                                    & 49972.284843                   & 44                                    & 97141.092979                 & 69                                     \\ 
					\hline
					23                                         & 413.077936                   & 12                                    & 55224.250671                   & 39                                    & 102738.584953                & 71                                     \\ 
					\hline
					24                                         & 298.426238                   & 12                                    & 53442.302917                   & 41                                    & 95437.557976                 & 77                                     \\ 
					\hline
					25                                         & 362.974943                   & 13                                    & 52942.692451                   & 43                                    & 109236.066574                & 76                                     \\ 
					\hline
					26                                         & 279.583288                   & 11                                    & 52528.093407                   & 42                                    & 112251.415863                & 71                                     \\ 
					\hline
					27                                         & 314.159356                   & 14                                    & 54601.247261                   & 43                                    & 102901.634148                & 69                                     \\ 
					\hline
					28                                         & 279.102985                   & 14                                    & 54642.370102                   & 46                                    & 96261.835789                 & 73                                     \\ 
					\hline
					29                                         & 296.096529                   & 13                                    & 51280.921593                   & 43                                    & 84929.672058                 & 71                                     \\ 
					\hline
					30                                         & 314.30529                    & 14                                    & 51389.147156                   & 42                                    & 96498.67128                  & 74                                     \\ 
					\hline
					\multicolumn{1}{|l|}{Average}              & 299.553292466667             & 12.7333333333333                      & 52699.5983075667               & 41.2666666666667                      & 101570.163644533             & 72.1                                   \\ 
					\hline
					\multicolumn{1}{|l|}{Std. Dev}             & 40.469222577263              & 1.01483252680985                      & 2062.76885562279               & 2.09980842038002                      & 6490.61277032704             & 3.14423391250583                       \\
					\hline
		\end{tabular}}}
	\end{adjustwidth}
	\caption{The entire experiment data we have collected using our GWO approach with $c = 8$ and a population of $50$.}
	\label{full-data-gwo-c8-p50}
\end{table}

\begin{table}
	\centering
	\begin{adjustwidth}{}{}
		\resizebox{\textwidth}{!}{\rotatebox{90}{
				\begin{tabular}{|r|r|r|r|r|r|r|}
					\hline
					\multicolumn{1}{|c|}{\multirow{2}{*}{Run}} & \multicolumn{6}{c|}{GWO (c = 12, Pop. Size of 50)}                                                                                                                                                                    \\ 
					\cline{2-7}
					\multicolumn{1}{|c|}{}                     & \multicolumn{1}{l|}{SFLP-II} & \multicolumn{1}{l|}{Elapsed Time (s)} & \multicolumn{1}{l|}{mSFLP-III} & \multicolumn{1}{l|}{Elapsed Time (s)} & \multicolumn{1}{l|}{mKra30a} & \multicolumn{1}{l|}{Elapsed Time (s)}  \\ 
					\hline
					1                                          & 330.967894                   & 13                                    & 51707.706673                   & 43                                    & 106886.297325                & 69                                     \\ 
					\hline
					2                                          & 352.688213                   & 14                                    & 52477.265709                   & 42                                    & 102138.644287                & 70                                     \\ 
					\hline
					3                                          & 326.198721                   & 13                                    & 54071.188835                   & 38                                    & 127234.147255                & 71                                     \\ 
					\hline
					4                                          & 316.33189                    & 14                                    & 51498.448105                   & 46                                    & 109019.598618                & 71                                     \\ 
					\hline
					5                                          & 290.179763                   & 12                                    & 54326.654686                   & 44                                    & 103742.514435                & 65                                     \\ 
					\hline
					6                                          & 358.362449                   & 13                                    & 51799.437698                   & 42                                    & 110403.709251                & 68                                     \\ 
					\hline
					7                                          & 312.598053                   & 14                                    & 55847.500244                   & 47                                    & 99383.761021                 & 66                                     \\ 
					\hline
					8                                          & 280.485977                   & 13                                    & 51694.484573                   & 47                                    & 112262.345184                & 65                                     \\ 
					\hline
					9                                          & 346.624847                   & 17                                    & 56477.689476                   & 46                                    & 99664.26989                  & 68                                     \\ 
					\hline
					10                                         & 307.655417                   & 15                                    & 55350.96994                    & 45                                    & 103295.469814                & 69                                     \\ 
					\hline
					11                                         & 318.316489                   & 13                                    & 52508.811115                   & 42                                    & 120704.796318                & 72                                     \\ 
					\hline
					12                                         & 313.9823                     & 16                                    & 50773.498726                   & 45                                    & 112647.377289                & 73                                     \\ 
					\hline
					13                                         & 286.658271                   & 14                                    & 52073.785706                   & 40                                    & 97144.798492                 & 69                                     \\ 
					\hline
					14                                         & 285.225881                   & 14                                    & 51173.391113                   & 40                                    & 128598.716599                & 65                                     \\ 
					\hline
					15                                         & 266.707759                   & 14                                    & 53180.308609                   & 40                                    & 93067.630051                 & 65                                     \\ 
					\hline
					16                                         & 324.595953                   & 14                                    & 49284.544159                   & 40                                    & 105843.944901                & 69                                     \\ 
					\hline
					17                                         & 380.423649                   & 13                                    & 53861.001526                   & 49                                    & 94020.885498                 & 69                                     \\ 
					\hline
					18                                         & 323.238861                   & 13                                    & 49585.285339                   & 40                                    & 93486.066071                 & 68                                     \\ 
					\hline
					19                                         & 357.150067                   & 13                                    & 48920.979538                   & 42                                    & 102269.656761                & 75                                     \\ 
					\hline
					20                                         & 303.404414                   & 14                                    & 53799.463486                   & 43                                    & 97102.425781                 & 69                                     \\ 
					\hline
					21                                         & 277.139452                   & 13                                    & 51834.894127                   & 45                                    & 98196.685638                 & 71                                     \\ 
					\hline
					22                                         & 261.869799                   & 13                                    & 53773.521938                   & 42                                    & 106061.312057                & 72                                     \\ 
					\hline
					23                                         & 342.69634                    & 13                                    & 53797.093964                   & 40                                    & 106350.569275                & 68                                     \\ 
					\hline
					24                                         & 328.36665                    & 12                                    & 49982.629631                   & 41                                    & 108921.795296                & 70                                     \\ 
					\hline
					25                                         & 267.253918                   & 15                                    & 53048.273376                   & 40                                    & 94603.179474                 & 72                                     \\ 
					\hline
					26                                         & 315.065236                   & 13                                    & 56021.777359                   & 39                                    & 92563.720146                 & 70                                     \\ 
					\hline
					27                                         & 381.586061                   & 16                                    & 53813.8256                     & 39                                    & 103393.719864                & 70                                     \\ 
					\hline
					28                                         & 299.772833                   & 13                                    & 53787.978058                   & 40                                    & 107972.391861                & 75                                     \\ 
					\hline
					29                                         & 315.251937                   & 14                                    & 54349.40506                    & 40                                    & 109941.11869                 & 71                                     \\ 
					\hline
					30                                         & 304.150171                   & 12                                    & 54298.960732                   & 40                                    & 113533.302567                & 72                                     \\ 
					\hline
					\multicolumn{1}{|l|}{Average}              & 315.831642166667             & 13.6666666666667                      & 52837.3591700333               & 42.2333333333333                      & 105348.4949903               & 69.5666666666667                       \\ 
					\hline
					\multicolumn{1}{|l|}{Std. Dev}             & 31.7204847308938             & 1.18418699983352                      & 1980.22102755171               & 2.86095394427529                      & 9267.72959691125             & 2.69972327232375                       \\
					\hline
		\end{tabular}}}
	\end{adjustwidth}
	\caption{The entire experiment data we have collected using our GWO approach with $c = 12$ and a population of $50$.}
	\label{full-data-gwo-c12-p50}
\end{table}

% Start of the GWO table of results for the one with a population of 75.
\begin{table}
	\centering
	\begin{adjustwidth}{}{}
		\resizebox{\textwidth}{!}{\rotatebox{90}{
				\begin{tabular}{|r|r|r|r|r|r|r|}
					\hline
					\multicolumn{1}{|c|}{\multirow{2}{*}{Run}} & \multicolumn{6}{c|}{GWO (c = 2, Pop. 75)}                                                                                                                                                                             \\ 
					\cline{2-7}
					\multicolumn{1}{|c|}{}                     & \multicolumn{1}{l|}{SFLP-II} & \multicolumn{1}{l|}{Elapsed Time (s)} & \multicolumn{1}{l|}{mSFLP-III} & \multicolumn{1}{l|}{Elapsed Time (s)} & \multicolumn{1}{l|}{mKra30a} & \multicolumn{1}{l|}{Elapsed Time (s)}  \\ 
					\hline
					1                                          & 298.520938                   & 20                                    & 53141.273369                   & 58                                    & 101530.62925                 & 107                                    \\ 
					\hline
					2                                          & 247.876691                   & 17                                    & 53633.342445                   & 62                                    & 89274.679817                 & 102                                    \\ 
					\hline
					3                                          & 381.271954                   & 18                                    & 50851.902184                   & 62                                    & 95287.577271                 & 101                                    \\ 
					\hline
					4                                          & 255.177597                   & 22                                    & 53696.3871                     & 60                                    & 101152.594696                & 103                                    \\ 
					\hline
					5                                          & 294.831714                   & 23                                    & 50902.627678                   & 61                                    & 104005.611374                & 107                                    \\ 
					\hline
					6                                          & 205.666955                   & 19                                    & 57068.075058                   & 57                                    & 105304.380791                & 99                                     \\ 
					\hline
					7                                          & 302.115958                   & 19                                    & 56211.556992                   & 62                                    & 88740.484344                 & 103                                    \\ 
					\hline
					8                                          & 314.284962                   & 17                                    & 52866.921768                   & 62                                    & 101882.857147                & 107                                    \\ 
					\hline
					9                                          & 266.953485                   & 20                                    & 54239.673317                   & 62                                    & 93008.283638                 & 102                                    \\ 
					\hline
					10                                         & 308.163814                   & 17                                    & 53867.634491                   & 62                                    & 99406.877457                 & 99                                     \\ 
					\hline
					11                                         & 320.826071                   & 19                                    & 50179.684898                   & 62                                    & 101618.364464                & 110                                    \\ 
					\hline
					12                                         & 298.084522                   & 18                                    & 54544.619156                   & 61                                    & 100320.698097                & 102                                    \\ 
					\hline
					13                                         & 308.806339                   & 17                                    & 54300.903725                   & 63                                    & 107151.828575                & 107                                    \\ 
					\hline
					14                                         & 282.147528                   & 18                                    & 55166.191742                   & 66                                    & 97411.479393                 & 103                                    \\ 
					\hline
					15                                         & 298.474119                   & 18                                    & 53445.058487                   & 68                                    & 89903.97784                  & 106                                    \\ 
					\hline
					16                                         & 291.704501                   & 19                                    & 50309.586052                   & 61                                    & 98993.719925                 & 108                                    \\ 
					\hline
					17                                         & 270.690811                   & 18                                    & 57659.66436                    & 64                                    & 117173.305939                & 112                                    \\ 
					\hline
					18                                         & 386.356476                   & 17                                    & 56617.206116                   & 59                                    & 95635.212196                 & 110                                    \\ 
					\hline
					19                                         & 256.02258                    & 19                                    & 52217.748032                   & 64                                    & 102206.51368                 & 105                                    \\ 
					\hline
					20                                         & 250.591168                   & 17                                    & 52619.802849                   & 59                                    & 98741.184265                 & 103                                    \\ 
					\hline
					21                                         & 297.072077                   & 18                                    & 53319.960991                   & 61                                    & 96027.794487                 & 104                                    \\ 
					\hline
					22                                         & 306.206314                   & 19                                    & 57618.996025                   & 59                                    & 103661.234585                & 107                                    \\ 
					\hline
					23                                         & 245.324803                   & 19                                    & 53255.321243                   & 58                                    & 97855.144623                 & 99                                     \\ 
					\hline
					24                                         & 300.076243                   & 23                                    & 53322.684452                   & 62                                    & 100661.008976                & 109                                    \\ 
					\hline
					25                                         & 341.507297                   & 22                                    & 54297.791504                   & 61                                    & 100964.315514                & 109                                    \\ 
					\hline
					26                                         & 256.656206                   & 17                                    & 54966.582184                   & 60                                    & 105386.165337                & 102                                    \\ 
					\hline
					27                                         & 262.987007                   & 18                                    & 57045.288116                   & 59                                    & 97001.08107                  & 101                                    \\ 
					\hline
					28                                         & 297.544054                   & 20                                    & 52647.000465                   & 61                                    & 94089.306458                 & 103                                    \\ 
					\hline
					29                                         & 258.89821                    & 20                                    & 54702.205833                   & 57                                    & 93904.516655                 & 109                                    \\ 
					\hline
					30                                         & 297.061259                   & 18                                    & 55745.558327                   & 60                                    & 96171.02298                  & 102                                    \\ 
					\hline
					\multicolumn{1}{|l|}{Average}              & 290.063388433333             & 18.8666666666667                      & 54015.3749653                  & 61.1                                  & 99149.0616948                & 104.7                                  \\ 
					\hline
					\multicolumn{1}{|l|}{Std. Dev}             & 38.0436802516604             & 1.75643316732074                      & 2050.7167136713                & 2.44032219466879                      & 5833.24082935413             & 3.62129715757327                       \\
					\hline
		\end{tabular}}}
	\end{adjustwidth}
	\caption{The entire experiment data we have collected using our GWO approach with $c = 2$ and a population of $75$.}
	\label{full-data-gwo-c2-p75}
\end{table}

\begin{table}
	\centering
	\begin{adjustwidth}{}{}
		\resizebox{\textwidth}{!}{\rotatebox{90}{
				\begin{tabular}{|r|r|r|r|r|r|r|}
					\hline
					\multicolumn{1}{|c|}{\multirow{2}{*}{Run}} & \multicolumn{6}{c|}{GWO (c = 4, Pop. 75)}                                                                                                                                                                             \\ 
					\cline{2-7}
					\multicolumn{1}{|c|}{}                     & \multicolumn{1}{l|}{SFLP-II} & \multicolumn{1}{l|}{Elapsed Time (s)} & \multicolumn{1}{l|}{mSFLP-III} & \multicolumn{1}{l|}{Elapsed Time (s)} & \multicolumn{1}{l|}{mKra30a} & \multicolumn{1}{l|}{Elapsed Time (s)}  \\ 
					\hline
					1                                          & 289.002406                   & 19                                    & 52142.283493                   & 60                                    & 98677.828976                 & 106                                    \\ 
					\hline
					2                                          & 277.052858                   & 19                                    & 49880.523918                   & 60                                    & 92124.010956                 & 105                                    \\ 
					\hline
					3                                          & 335.462648                   & 23                                    & 53328.857437                   & 62                                    & 103596.954788                & 106                                    \\ 
					\hline
					4                                          & 377.541344                   & 17                                    & 49640.026917                   & 61                                    & 103328.22345                 & 106                                    \\ 
					\hline
					5                                          & 239.258536                   & 19                                    & 51617.14695                    & 57                                    & 96614.5653                   & 106                                    \\ 
					\hline
					6                                          & 325.637233                   & 19                                    & 53735.167488                   & 62                                    & 91132.62336                  & 105                                    \\ 
					\hline
					7                                          & 288.088882                   & 17                                    & 49864.33725                    & 70                                    & 97078.708122                 & 108                                    \\ 
					\hline
					8                                          & 285.496838                   & 19                                    & 55286.558464                   & 61                                    & 109522.060501                & 101                                    \\ 
					\hline
					9                                          & 323.256618                   & 20                                    & 51300.184471                   & 59                                    & 105238.711536                & 103                                    \\ 
					\hline
					10                                         & 287.471357                   & 17                                    & 49627.393288                   & 61                                    & 96470.319412                 & 104                                    \\ 
					\hline
					11                                         & 265.513964                   & 20                                    & 55577.590172                   & 62                                    & 107900.23465                 & 104                                    \\ 
					\hline
					12                                         & 278.710724                   & 18                                    & 48762.911812                   & 58                                    & 92805.567352                 & 100                                    \\ 
					\hline
					13                                         & 262.309272                   & 18                                    & 53302.249931                   & 61                                    & 106591.955357                & 107                                    \\ 
					\hline
					14                                         & 288.189661                   & 21                                    & 51965.122269                   & 64                                    & 101124.411228                & 110                                    \\ 
					\hline
					15                                         & 291.547712                   & 19                                    & 48824.390396                   & 63                                    & 98799.394608                 & 114                                    \\ 
					\hline
					16                                         & 267.058788                   & 17                                    & 49248.646812                   & 61                                    & 90563.94965                  & 107                                    \\ 
					\hline
					17                                         & 265.283893                   & 19                                    & 56156.624268                   & 62                                    & 99513.980186                 & 113                                    \\ 
					\hline
					18                                         & 365.171428                   & 18                                    & 48752.443314                   & 60                                    & 94829.357605                 & 118                                    \\ 
					\hline
					19                                         & 406.790997                   & 18                                    & 51401.382912                   & 59                                    & 89197.608078                 & 112                                    \\ 
					\hline
					20                                         & 295.505686                   & 18                                    & 54511.452782                   & 63                                    & 89742.548553                 & 115                                    \\ 
					\hline
					21                                         & 329.868032                   & 19                                    & 50549.009491                   & 59                                    & 121878.61042                 & 108                                    \\ 
					\hline
					22                                         & 263.242765                   & 18                                    & 56138.764824                   & 66                                    & 100696.10968                 & 101                                    \\ 
					\hline
					23                                         & 260.896212                   & 18                                    & 51281.46727                    & 64                                    & 105140.868805                & 107                                    \\ 
					\hline
					24                                         & 332.599415                   & 19                                    & 53231.359383                   & 61                                    & 98453.998238                 & 115                                    \\ 
					\hline
					25                                         & 355.053306                   & 19                                    & 56071.344505                   & 58                                    & 94238.087372                 & 110                                    \\ 
					\hline
					26                                         & 310.583911                   & 22                                    & 51826.126656                   & 63                                    & 95764.557251                 & 109                                    \\ 
					\hline
					27                                         & 261.280996                   & 18                                    & 51597.582993                   & 62                                    & 98053.700539                 & 115                                    \\ 
					\hline
					28                                         & 305.331404                   & 18                                    & 52095.70153                    & 62                                    & 97999.711472                 & 108                                    \\ 
					\hline
					29                                         & 305.927029                   & 24                                    & 50398.410248                   & 63                                    & 98793.284912                 & 112                                    \\ 
					\hline
					30                                         & 321.044512                   & 17                                    & 53014.826584                   & 62                                    & 108595.115971                & 104                                    \\ 
					\hline
					\multicolumn{1}{|l|}{Average}              & 302.005947566667             & 18.9                                  & 52037.6629276                  & 61.5333333333333                      & 99482.2352776                & 107.966666666667                       \\ 
					\hline
					\multicolumn{1}{|l|}{Std. Dev}             & 39.1344289013742             & 1.70900253223113                      & 2313.4004317195                & 2.54251210736389                      & 7069.6968084522              & 4.62737978538189                       \\
					\hline
		\end{tabular}}}
	\end{adjustwidth}
	\caption{The entire experiment data we have collected using our GWO approach with $c = 4$ and a population of $75$.}
	\label{full-data-gwo-c4-p75}
\end{table}

\begin{table}
	\centering
	\begin{adjustwidth}{}{}
		\resizebox{\textwidth}{!}{\rotatebox{90}{
				\begin{tabular}{|r|r|r|r|r|r|r|}
					\hline
					\multicolumn{1}{|c|}{\multirow{2}{*}{Run}} & \multicolumn{6}{c|}{GWO (c = 8, Pop. 75)}                                                                                                                                                                             \\ 
					\cline{2-7}
					\multicolumn{1}{|c|}{}                     & \multicolumn{1}{l|}{SFLP-II} & \multicolumn{1}{l|}{Elapsed Time (s)} & \multicolumn{1}{l|}{mSFLP-III} & \multicolumn{1}{l|}{Elapsed Time (s)} & \multicolumn{1}{l|}{mKra30a} & \multicolumn{1}{l|}{Elapsed Time (s)}  \\ 
					\hline
					1                                          & 319.414222                   & 19                                    & 53533.50576                    & 64                                    & 89710.066864                 & 115                                    \\ 
					\hline
					2                                          & 328.165713                   & 20                                    & 50595.545044                   & 62                                    & 104972.593979                & 108                                    \\ 
					\hline
					3                                          & 306.412776                   & 20                                    & 50424.89698                    & 61                                    & 99951.159401                 & 110                                    \\ 
					\hline
					4                                          & 376.979948                   & 19                                    & 52540.922691                   & 61                                    & 100989.191406                & 109                                    \\ 
					\hline
					5                                          & 300.174307                   & 18                                    & 51199.14978                    & 60                                    & 89169.450142                 & 112                                    \\ 
					\hline
					6                                          & 279.19583                    & 18                                    & 50648.013687                   & 60                                    & 113760.078079                & 112                                    \\ 
					\hline
					7                                          & 312.371843                   & 19                                    & 52296.237488                   & 62                                    & 99162.493286                 & 108                                    \\ 
					\hline
					8                                          & 260.155633                   & 20                                    & 51729.699924                   & 61                                    & 101318.449509                & 106                                    \\ 
					\hline
					9                                          & 309.446858                   & 19                                    & 50810.120876                   & 65                                    & 98910.665161                 & 106                                    \\ 
					\hline
					10                                         & 300.916353                   & 17                                    & 52049.903336                   & 64                                    & 104835.718559                & 105                                    \\ 
					\hline
					11                                         & 302.721873                   & 20                                    & 53897.510803                   & 66                                    & 106039.799339                & 111                                    \\ 
					\hline
					12                                         & 321.177588                   & 19                                    & 53753.174721                   & 63                                    & 87299.715054                 & 117                                    \\ 
					\hline
					13                                         & 285.325578                   & 18                                    & 54977.558044                   & 64                                    & 91432.348927                 & 112                                    \\ 
					\hline
					14                                         & 312.951696                   & 18                                    & 49276.248596                   & 66                                    & 92447.678101                 & 108                                    \\ 
					\hline
					15                                         & 342.242961                   & 19                                    & 51142.870007                   & 66                                    & 102956.340828                & 116                                    \\ 
					\hline
					16                                         & 246.916627                   & 20                                    & 53668.493233                   & 61                                    & 103917.16843                 & 106                                    \\ 
					\hline
					17                                         & 288.087718                   & 20                                    & 53224.681351                   & 62                                    & 96208.562988                 & 106                                    \\ 
					\hline
					18                                         & 292.023611                   & 21                                    & 49689.811363                   & 63                                    & 91004.287216                 & 110                                    \\ 
					\hline
					19                                         & 306.479581                   & 21                                    & 50058.597248                   & 61                                    & 94997.346939                 & 107                                    \\ 
					\hline
					20                                         & 307.767267                   & 19                                    & 51180.542244                   & 62                                    & 92252.138172                 & 102                                    \\ 
					\hline
					21                                         & 281.911149                   & 22                                    & 50156.850952                   & 63                                    & 101269.8825                  & 106                                    \\ 
					\hline
					22                                         & 365.744095                   & 21                                    & 52686.439079                   & 65                                    & 103977.26178                 & 113                                    \\ 
					\hline
					23                                         & 316.998798                   & 18                                    & 51797.785599                   & 61                                    & 112025.688389                & 107                                    \\ 
					\hline
					24                                         & 293.730703                   & 20                                    & 51566.563713                   & 68                                    & 98071.09903                  & 105                                    \\ 
					\hline
					25                                         & 281.707827                   & 20                                    & 51542.716793                   & 61                                    & 98420.038445                 & 108                                    \\ 
					\hline
					26                                         & 358.463471                   & 20                                    & 53318.161484                   & 64                                    & 92146.580208                 & 105                                    \\ 
					\hline
					27                                         & 393.744452                   & 20                                    & 50811.437714                   & 66                                    & 101289.536804                & 116                                    \\ 
					\hline
					28                                         & 304.854541                   & 19                                    & 53189.377037                   & 63                                    & 101786.061531                & 107                                    \\ 
					\hline
					29                                         & 299.836358                   & 20                                    & 50725.26915                    & 64                                    & 100487.72049                 & 104                                    \\ 
					\hline
					30                                         & 302.820106                   & 21                                    & 51564.429115                   & 62                                    & 98249.547806                 & 104                                    \\ 
					\hline
					\multicolumn{1}{|l|}{Average}              & 309.957982766667             & 19.5                                  & 51801.8837937333               & 63.0333333333333                      & 98968.6223121                & 108.7                                  \\ 
					\hline
					\multicolumn{1}{|l|}{Std. Dev}             & 32.0156308267039             & 1.13714706536836                      & 1419.1918023338                & 2.07586015962031                      & 6443.60722715266             & 3.94924698191161                       \\
					\hline
		\end{tabular}}}
	\end{adjustwidth}
	\caption{The entire experiment data we have collected using our GWO approach with $c = 8$ and a population of $75$.}
	\label{full-data-gwo-c8-p75}
\end{table}

\begin{table}
	\centering
	\begin{adjustwidth}{}{}
		\resizebox{\textwidth}{!}{\rotatebox{90}{
				\begin{tabular}{|r|r|r|r|r|r|r|}
					\hline
					\multicolumn{1}{|c|}{\multirow{2}{*}{Run}} & \multicolumn{6}{c|}{GWO (c = 12, Pop. 75)}                                                                                                                                                                            \\ 
					\cline{2-7}
					\multicolumn{1}{|c|}{}                     & \multicolumn{1}{l|}{SFLP-II} & \multicolumn{1}{l|}{Elapsed Time (s)} & \multicolumn{1}{l|}{mSFLP-III} & \multicolumn{1}{l|}{Elapsed Time (s)} & \multicolumn{1}{l|}{mKra30a} & \multicolumn{1}{l|}{Elapsed Time (s)}  \\ 
					\hline
					1                                          & 311.266407                   & 18                                    & 50203.176975                   & 60                                    & 104539.105591                & 108                                    \\ 
					\hline
					2                                          & 313.716906                   & 21                                    & 49644.232903                   & 62                                    & 96584.774765                 & 113                                    \\ 
					\hline
					3                                          & 288.457501                   & 19                                    & 50234.59729                    & 66                                    & 86942.304199                 & 108                                    \\ 
					\hline
					4                                          & 321.340706                   & 18                                    & 51165.054924                   & 60                                    & 98779.496071                 & 115                                    \\ 
					\hline
					5                                          & 272.457565                   & 19                                    & 55524.684891                   & 60                                    & 108422.175175                & 111                                    \\ 
					\hline
					6                                          & 288.239145                   & 20                                    & 54490.212311                   & 61                                    & 93106.942497                 & 107                                    \\ 
					\hline
					7                                          & 263.701073                   & 18                                    & 50293.834431                   & 58                                    & 105715.713463                & 109                                    \\ 
					\hline
					8                                          & 309.448008                   & 20                                    & 51443.199959                   & 65                                    & 95049.082172                 & 122                                    \\ 
					\hline
					9                                          & 287.407971                   & 20                                    & 51524.595062                   & 59                                    & 89120.334511                 & 111                                    \\ 
					\hline
					10                                         & 293.245684                   & 20                                    & 52417.916771                   & 65                                    & 91566.401367                 & 117                                    \\ 
					\hline
					11                                         & 284.921627                   & 18                                    & 50844.285988                   & 63                                    & 104532.725182                & 118                                    \\ 
					\hline
					12                                         & 334.502174                   & 21                                    & 51783.373306                   & 58                                    & 99873.148727                 & 113                                    \\ 
					\hline
					13                                         & 335.122006                   & 18                                    & 51240.802536                   & 65                                    & 87224.183014                 & 118                                    \\ 
					\hline
					14                                         & 342.360371                   & 20                                    & 51226.104279                   & 58                                    & 94446.328514                 & 119                                    \\ 
					\hline
					15                                         & 386.133494                   & 17                                    & 53765.692108                   & 62                                    & 99027.918419                 & 111                                    \\ 
					\hline
					16                                         & 243.386427                   & 19                                    & 51797.691086                   & 67                                    & 105792.314682                & 103                                    \\ 
					\hline
					17                                         & 278.076036                   & 18                                    & 50207.38295                    & 60                                    & 103782.227905                & 109                                    \\ 
					\hline
					18                                         & 342.271969                   & 19                                    & 51941.683334                   & 66                                    & 93258.144852                 & 108                                    \\ 
					\hline
					19                                         & 286.475345                   & 18                                    & 52165.466835                   & 61                                    & 95130.892349                 & 110                                    \\ 
					\hline
					20                                         & 290.548313                   & 20                                    & 50326.696167                   & 63                                    & 98527.791496                 & 113                                    \\ 
					\hline
					21                                         & 291.333399                   & 19                                    & 54636.728546                   & 61                                    & 101805.933556                & 105                                    \\ 
					\hline
					22                                         & 331.904596                   & 20                                    & 50238.804039                   & 61                                    & 97895.015785                 & 116                                    \\ 
					\hline
					23                                         & 264.303243                   & 23                                    & 53149.875965                   & 59                                    & 92221.346848                 & 110                                    \\ 
					\hline
					24                                         & 348.721866                   & 20                                    & 52677.171204                   & 60                                    & 90693.638336                 & 113                                    \\ 
					\hline
					25                                         & 413.874466                   & 20                                    & 52407.621346                   & 60                                    & 107340.909904                & 111                                    \\ 
					\hline
					26                                         & 282.0196                     & 21                                    & 51187.63504                    & 67                                    & 100840.685257                & 113                                    \\ 
					\hline
					27                                         & 294.260799                   & 20                                    & 52864.498299                   & 60                                    & 107007.960152                & 101                                    \\ 
					\hline
					28                                         & 322.869648                   & 19                                    & 51778.318222                   & 62                                    & 89651.275627                 & 111                                    \\ 
					\hline
					29                                         & 303.557749                   & 19                                    & 51775.92057                    & 64                                    & 107554.843956                & 108                                    \\ 
					\hline
					30                                         & 293.715599                   & 18                                    & 52179.042252                   & 60                                    & 96833.664429                 & 107                                    \\ 
					\hline
					\multicolumn{1}{|l|}{Average}              & 307.3213231                  & 19.3333333333333                      & 51837.8766529667               & 61.7666666666667                      & 98108.9092933667             & 111.266666666667                       \\ 
					\hline
					\multicolumn{1}{|l|}{Std. Dev}             & 36.2239957721235             & 1.26854065851231                      & 1430.57988385005               & 2.72514894668332                      & 6511.43059062118             & 4.75563791241753                       \\
					\hline
		\end{tabular}}}
	\end{adjustwidth}
	\caption{The entire experiment data we have collected using our GWO approach with $c = 12$ and a population of $75$.}
	\label{full-data-gwo-c12-p75}
\end{table}

\subsection{Results of Other Approaches}
Each approach has their own parameters, and the values we have set for those parameters are shown in Table \ref{approach-parameters}. Both approaches use a population size of 50, and a maximum number of iterations of 400. The following parameter values for the PSO approach were taken from the work of Jolai, F., Tavakkoli-Moghaddam, R., and Taghipour, M. \cite{Jolai2012}. Our proposed GWO approach is not included in the following results since we have already discussed them in the previous subsection.

\begin{table}[h!]
	\centering
	\begin{tabular}{|l|l|l|}
		\hline
		\textbf{Approach}   & \textbf{Parameter} & \textbf{Value} \\ \hline
		\multirow{3}{*}{GA} & Mutation Rate      & 0.05           \\ \cline{2-3} 
		& Tournament Size    & 4              \\ \cline{2-3} 
		& No. of Elites (EN) & 5              \\ \hline
		\multirow{3}{*}{PSO} & w      & 0.05           \\ \cline{2-3} 
		& c1    & 2              \\ \cline{2-3} 
		& c2 	& 2              \\ \hline
	\end{tabular}
	\caption{Parameter values of the GWO, GA, and PSO approaches.}
	\label{approach-parameters}
\end{table}

The results obtained for each approach is shown in Tables \ref{approach-ga-results} and \ref{approach-pso-results}, respectively. As what the tables show, the competing genetic algorithm approach produces a solution that is better than our proposed GWO approach (with any value of $c$) and the PSO approach, with a fitness averages of $274.120352366667$, $50676.8183791$, and $87715.8254635$ for the SFLP-II, mSFLP-III, and mKra30a problem configurations respectively. This is compared to our proposed approach's fitness averages. Fortunately for our approach, the PSO approach obtained the fitness averages of $321.292520833333$, $64289.8051163$, and $121057.4221481$, proving that our GWO is not the worst approach. For SFLP-II, the best and worst solutions have fitnesses of $232.839593$ and $353.006875$ for the GA, respectively, compared to the PSO approach which obtained $273.754488$ and $382.774055$. For mSFLP-III, the best and worst solutions have a fitness of $47177.914444$ and $53668.451469$ for the GA, respectively, compared to the PSO approach's $59673.997963$ and $68433.641548$. Lastly, for mKra30a, the best and worst solutions have fitnesses of $76454.204788$ and $96215.108131$, respectively, for the GA. PSO produces the poorest best and worst solutions with fitnesses of $107996.773666$ and $131275.843658$. This behaviour of producing the best average solution of the GA approach is attributed to the local search methods, which relatively exhaustively finds a better solution in a small area near the best solution found so far in each iteration. These local search methods intensifies the exploitation phase of the approach. Our GWO approach also exploits the local area, but it is not as intensive as the GA approach and only occurs at a later time in a run, similar to how the PSO approach behaves. Notice as well how PSO produces the worst solutions on average among the three. This can be explained with how particles in the PSO approach are equally influenced by their personal best position and their swarm's global best position. This reduces the chances of particles from exploiting the area around the global best position. This behaviour also explains the observation we have with PSO during our experiemnts where the approach struggles to produce good results for the mKra30a data set. It requires multiple runs just to produce a single feasible solution. This is unlike our GWO approach where all wolves are influenced/led by the best three wolves, enabling them to exploit the space around the best found solution. In future studies, different behaviour may be observed when the PSO parameters are tweaked to different values.

\begin{figure}[h!]
\centering
\begin{adjustwidth}{-0.45in}{}
\includegraphics[scale=0.5]{./images/chap07-rd/approaches-average-runtime-over-no-of-buildings.png}
\end{adjustwidth}
\caption{The average runtime (s) of each of the approaches as the number of buildings in a data set increase.}
\label{graph-approaches-runtime-no-buildings}
\end{figure}

The genetic algorithm approach is also the fastest when SFLP-II is being used with an average run time of $27.3666666666667$s, compared to our approach's time and the PSO approach's $29.0666666666667$s. However, as the number of buildings increase, the average runtime of the GA approach becomes worse compared to the two other approaches. With mSFLP-III, GA takes $166.866666666667$s, while the PSO approach takes $223.566666666667$s and $77.0333333333333$s, respectively. GA is still faster than GWO in this data set, but it is already significantly slower than the PSO approach, unlike with the previous data set. Moving to mKra30a, we can see that GA now takes $455.5$s. This is longer than our approach, and the PSO approach's $117.4$s. Figure \ref{graph-approaches-runtime-no-buildings} shows this observation. We can attribute this faster increase in average runtime as the number of buildings increase in the GA approach to what enables it produce better solutions on average\textemdash its local search methods. Since the local search methods perform a relatively exhaustive search in order to find a better solution, the GA will take more time to finish executing. Hence, we observe this phenomenon. This is not the case with GWO and PSO, due to the lack of local search methods. GWO may have taken a longer time due to the amount of operations that are performed in the metaheuristic compared to PSO. Better implementations, especially those that utilize SIMD operations, for both approaches may reduce the gap in terms of average run time between the two. However, basing from the equations in both metaheuristics, it is likely that PSO will remain faster than GWO. Further studies, however, are required to exactly determine how well each approach scales with regards to the number of buildings.

\begin{table}[h!]
\begin{adjustwidth}{-1.15in}{}
\centering
\begin{tabular}{|l|l|l|l|l|l|}
	\hline
	\multicolumn{1}{|c|}{\multirow{2}{*}{\textbf{Problem}}} & \multicolumn{5}{c|}{\textbf{Genetic Algorithm}}                                                                                                                                                                                            \\ \cline{2-6} 
	\multicolumn{1}{|c|}{}                                  & \multicolumn{1}{c|}{\textbf{Best}} & \multicolumn{1}{c|}{\textbf{Worst}} & \multicolumn{1}{c|}{\textbf{Avg.}} & \multicolumn{1}{c|}{\textbf{Std. Dev.}} & \multicolumn{1}{c|}{\textbf{Avg. Runtime (s)}} \\ \hline
	SFLP-II                                                 & 232.839593                                  & 353.006875                                   & 274.120352366667                      &
	31.3302957404409						& 27.3666666666667                                   \\ \hline
	mSFLP-III                                               & 47177.914444                               & 53668.451469                                 & 50676.8183791                      & 1325.4427345102                                  & 166.866666666667                           \\ \hline
	mKra30a                                               & 76454.204788                                & 96215.108131                                 &
	87715.8254635							&
	4281.86238995314	                        &
	455.5								\\ \hline
\end{tabular}
\end{adjustwidth}
\caption{Results obtained from using the competing GA approach.}
\label{approach-ga-results}
\end{table}

\begin{table}[h!]
\begin{adjustwidth}{-1.18in}{}
\centering
\begin{tabular}{|l|l|l|l|l|l|}
	\hline
	\multicolumn{1}{|c|}{\multirow{2}{*}{\textbf{Problem}}} & \multicolumn{5}{c|}{\textbf{PSO}} \\ \cline{2-6} 
	\multicolumn{1}{|c|}{}                                  & \multicolumn{1}{c|}{\textbf{Best}} & \multicolumn{1}{c|}{\textbf{Worst}} & \multicolumn{1}{c|}{\textbf{Avg.}} & \multicolumn{1}{c|}{\textbf{Std. Dev.}} & \multicolumn{1}{c|}{\textbf{Avg. Runtime (s)}} \\ \hline
	SFLP-II                                                 & 273.754488                                  & 382.774055                                   &
	321.292520833333							&
	32.8103600821748							&
	29.0666666666667							\\ \hline
	mSFLP-III                                               & 59673.997963                                & 68433.641548                                 &
	64289.8051163					          &
	2356.26248290771						&
	77.0333333333333						\\ \hline
	mKra30a                                               & 107996.773666                                & 131275.843658                                 &
	121057.4221481							&
	5981.21601161922							&
	117.4						\\ \hline
\end{tabular}
\end{adjustwidth}
\caption{Results obtained from our proposed PSO approach.}
\label{approach-pso-results}
\end{table}

We can further obtain insights from our results, by looking at the best solutions generated by each approach. Figures \ref{graph-approaches-best-solutions-sflp-ii} to \ref{graph-approaches-best-solutions-mkra30a} show the fitness graphs of the best solutions using the SFLP-II, mSFLP-III, and mKra30a data sets. The non-linearity of the graph of our GWO approach that is observable in the figures is due to the nature of our approach. All solutions in a population in our approach are replaced after an iteration. Hence, the best solutions may be replaced by poorer solutions. This is unlike with the GA and PSO approaches where  the fitness continuously lower over time. In the case of the GA approach, this is due to the fact that there is elitism. The best $EN$ solutions are kept in the next generation, taking the place of the worst generated solutions produced in the current generation. The PSO approach has a similar characteristic that enables it to produce a continuously lowering fitness graph. In the PSO approach, as discussed before, each particle keeps track of the best position it has found. After an iteration, if a particle finds a position that is better than its personal best, then that position will replace the particle's personal. And if a particle's personal best is better than the swarm's global best, then that position/solution becomes the swarm's global best. It is this global best that is tracked in the graph. The selection procedure of the global best explains the graph of the PSO approach.

\begin{figure}[h!]
\centering
\begin{adjustwidth}{-0.45in}{}
\includegraphics[scale=0.5]{./images/chap07-rd/best-fitness-over-time-sflp2.png}
\end{adjustwidth}
\caption{Fitness over time of the best solutions for the SFLP-II produced by the GA, GWO, and PSO approaches.}
\label{graph-approaches-best-solutions-sflp-ii}
\end{figure}

\begin{figure}[h!]
\centering
\begin{adjustwidth}{-0.45in}{}
\includegraphics[scale=0.5]{./images/chap07-rd/best-fitness-over-time-msflp3.png}
\end{adjustwidth}
\caption{Fitness over time of the best solutions for the mSFLP-III produced by the GA, GWO, and PSO approaches.}
\label{graph-approaches-best-solutions-msflp-iii}
\end{figure}

\begin{figure}[h!]
\centering
\begin{adjustwidth}{-0.45in}{}
\includegraphics[scale=0.5]{./images/chap07-rd/best-fitness-over-time-mKra30a.png}
\end{adjustwidth}
\caption{Fitness over time of the best solutions for the mKra30a produced by the GA, GWO, and PSO approaches.}
\label{graph-approaches-best-solutions-mkra30a}
\end{figure}

Another avenue we can use to gather insights is through the visualization of the results produced by the approaches mentioned in this study. Figures \ref{best-results-ga} to \ref{best-results-pso} show a visualization of the best results. Notice that with the hybrid GA approach and our GWO approach, the buildings tend to clump together, which is what we want to happen, based on our objective function. For our hybrid GA approach, we can attribute the result to the local search method as well as the mutation operators as they were key to ensure that the buildings are close to each other. The crossover operator is also instrumental in achieving this result by finding combinations that will lead to the result. Our GWO approach also makes buildings clump together but not to the same degree as the GA approach, as can be observed from one building being far from the rest of the buildings in mKra30a data set in Figure \ref{best-results-gwo}. The clumping ability of our approach is attributable to how solutions are allowed to perturb their buildings to positions relatively far from the buildings positions in the best three solution initially. Eventually, our approach will decrease the distance of the buildings in a solution from the leading solutions. Remember that the leading solutions eventually become similar to each other, which help drive the reduction of the degree of building shifting. This gradually decreasing shifting of the buildings will lead to intersections from being resolved and reducing the distance of buildings from each other. The intersections are resolved by reducing the chances of buildings being to moved to a relatively further position where they would still intersect with another building, and gradually pushing intersecting buildings away from each other towards non-intersection. Note that the objective function has a lower penalty for solutions with buildings that do not significantly intersect. The decreasing shifting also encourages buildings to move towards each other due to the fact that smaller shifts have lower probability of causing buildings to intersect with one another too deeply or at all, which allows buildings to move to positions that are closer to the other buildings but without any intersections. Finally, as one can notice in Figure \ref{best-results-pso}, the PSO approach struggles to produce a solution where the buildings are clumped together. This deficiency is not necessarily clear with a small number of buildings, but it does as the number increases. We can attribute this difficulty of the PSO approach to the fact that the buildings are continuously being influenced on the same degree throughout all of the iterations by a particle's personal best and the swarm's global best. This encourages more exploitation and fewer exploitation. As a result, this reduces the chances in which buildings would be able to shift their positions by a small amount, making it more difficult for the approach to find better solutions and have buildings clump together.

\begin{figure}[h!]
\centering
\includegraphics[scale=1.85]{./images/chap07-rd/ga-best-solutions.png}
\caption{Visualization of the best solutions produced by the hybrid GA approach for the three data sets used in this study.}
\label{best-results-ga}
\end{figure}

\begin{figure}[h!]
\centering
\includegraphics[scale=1.85]{./images/chap07-rd/gwo-c2-best-solutions.png}
\caption{Visualization of the best solutions produced by our GWO approach with $c=2$ for the three data sets used in this study.}
\label{best-results-gwo-c2}
\end{figure}

\begin{figure}[h!]
\centering
\includegraphics[scale=1.85]{./images/chap07-rd/gwo-c4-best-solutions.png}
\caption{Visualization of the best solutions produced by our GWO approach with $c=4$ for the three data sets used in this study.}
\label{best-results-gwo-c4}
\end{figure}

\begin{figure}[h!]
\centering
\includegraphics[scale=1.85]{./images/chap07-rd/gwo-c8-best-solutions.png}
\caption{Visualization of the best solutions produced by our GWO approach with $c=12$ for the three data sets used in this study.}
\label{best-results-gwo-c8}
\end{figure}

\begin{figure}[h!]
\centering
\includegraphics[scale=1.85]{./images/chap07-rd/gwo-c12-best-solutions.png}
\caption{Visualization of the best solutions produced by our GWO approach with $c=12$ for the three data sets used in this study.}
\label{best-results-gwo}
\end{figure}

\begin{figure}[h!]
\centering
\includegraphics[scale=1.85]{./images/chap07-rd/pso-best-solutions.png}
\caption{Visualization of the best solutions produced by the PSO approach for the three data sets used in this study.}
\label{best-results-pso}
\end{figure}

For reference, Tables \ref{full-data-ga} to \ref{full-data-pso} provide the detailed numbers we have obtained in our experiments for each approach and data set. Table \ref{full-data-ga} shows the entire experiment data for our hybrid GA approach, table \ref{full-data-gwo} shows the data for our GWO approach, and lastly, table \ref{full-data-pso} shows the data for the PSO approach.

\begin{table}
\centering
\begin{adjustwidth}{}{}
\resizebox{\textwidth}{!}{\rotatebox{90}{
\begin{tabular}{|r|r|r|r|r|r|r|} 
\hline
\multicolumn{1}{|c|}{\multirow{2}{*}{Run}} & \multicolumn{6}{c|}{GA Experimental Results}                                                                                                                                                                          \\ 
\cline{2-7}
\multicolumn{1}{|c|}{}                     & \multicolumn{1}{l|}{SFLP-II} & \multicolumn{1}{l|}{Elapsed Time (s)} & \multicolumn{1}{l|}{mSFLP-III} & \multicolumn{1}{l|}{Elapsed Time (s)} & \multicolumn{1}{l|}{mKra30a} & \multicolumn{1}{l|}{Elapsed Time (s)}  \\ 
\hline
1                                          & 241.353251                   & 31                                    & 51469.406654                   & 155                                   & 76454.204788                 & 417                                    \\ 
\hline
2                                          & 260.666385                   & 27                                    & 50203.262684                   & 166                                   & 90215.278652                 & 454                                    \\ 
\hline
3                                          & 317.934947                   & 34                                    & 49903.403969                   & 159                                   & 80078.077255                 & 442                                    \\ 
\hline
4                                          & 257.12075                    & 26                                    & 51214.094765                   & 149                                   & 85424.753021                 & 453                                    \\ 
\hline
5                                          & 252.747648                   & 26                                    & 49754.550606                   & 152                                   & 85276.805458                 & 416                                    \\ 
\hline
6                                          & 291.574655                   & 27                                    & 49526.201752                   & 160                                   & 88156.523018                 & 472                                    \\ 
\hline
7                                          & 292.021864                   & 27                                    & 49669.613991                   & 157                                   & 83149.07745                  & 423                                    \\ 
\hline
8                                          & 281.474234                   & 27                                    & 51731.675354                   & 176                                   & 88411.142647                 & 413                                    \\ 
\hline
9                                          & 269.290927                   & 26                                    & 51245.732857                   & 172                                   & 90734.373741                 & 478                                    \\ 
\hline
10                                         & 269.423721                   & 27                                    & 48920.435341                   & 175                                   & 87220.602196                 & 488                                    \\ 
\hline
11                                         & 240.832309                   & 26                                    & 52856.025909                   & 212                                   & 81419.135918                 & 446                                    \\ 
\hline
12                                         & 259.251911                   & 24                                    & 49490.654926                   & 144                                   & 89084.201248                 & 498                                    \\ 
\hline
13                                         & 257.123175                   & 27                                    & 51225.387764                   & 164                                   & 82219.564217                 & 449                                    \\ 
\hline
14                                         & 232.839593                   & 26                                    & 49141.684452                   & 197                                   & 91560.704735                 & 440                                    \\ 
\hline
15                                         & 305.979567                   & 27                                    & 52062.448891                   & 161                                   & 85431.277306                 & 446                                    \\ 
\hline
16                                         & 263.951872                   & 26                                    & 51269.011093                   & 175                                   & 93689.351776                 & 451                                    \\ 
\hline
17                                         & 271.432117                   & 25                                    & 52881.041428                   & 176                                   & 85395.590744                 & 431                                    \\ 
\hline
18                                         & 278.044098                   & 33                                    & 51087.689514                   & 180                                   & 87894.849144                 & 421                                    \\ 
\hline
19                                         & 266.951438                   & 28                                    & 51001.627296                   & 171                                   & 91152.267059                 & 445                                    \\ 
\hline
20                                         & 246.816533                   & 27                                    & 50198.549866                   & 178                                   & 88895.677979                 & 471                                    \\ 
\hline
21                                         & 275.481676                   & 27                                    & 50003.058311                   & 162                                   & 87140.735931                 & 468                                    \\ 
\hline
22                                         & 353.006875                   & 26                                    & 51480.186226                   & 162                                   & 88911.361496                 & 533                                    \\ 
\hline
23                                         & 336.045243                   & 26                                    & 51068.014679                   & 182                                   & 88503.003464                 & 499                                    \\ 
\hline
24                                         & 267.427919                   & 28                                    & 53668.451469                   & 177                                   & 90751.106781                 & 487                                    \\ 
\hline
25                                         & 318.71513                    & 28                                    & 49688.022858                   & 167                                   & 87554.161316                 & 480                                    \\ 
\hline
26                                         & 239.385818                   & 31                                    & 50890.728996                   & 161                                   & 89753.353699                 & 516                                    \\ 
\hline
27                                         & 235.181557                   & 27                                    & 51466.78817                    & 160                                   & 92770.250797                 & 431                                    \\ 
\hline
28                                         & 256.815948                   & 27                                    & 47177.914444                   & 159                                   & 93206.425659                 & 443                                    \\ 
\hline
29                                         & 329.646141                   & 27                                    & 50171.089684                   & 146                                   & 84805.798279                 & 473                                    \\ 
\hline
30                                         & 255.073269                   & 27                                    & 49837.797424                   & 151                                   & 96215.108131                 & 381                                    \\ 
\hline
\multicolumn{1}{|l|}{Average}              & 274.120352366667             & 27.3666666666667                      & 50676.8183791                  & 166.866666666667                      & 87715.8254635                & 455.5                                  \\ 
\hline
\multicolumn{1}{|l|}{Std. Dev}             & 31.3302957404409             & 2.17324377503931                      & 1325.4427345102                & 14.6775299372024                      & 4281.86238995314             & 33.3639801644497                       \\
\hline
\end{tabular}}}
\end{adjustwidth}
\caption{The entire experiment data we have collected using our hybrid GA approach.}
\label{full-data-ga}
\end{table}

\begin{table}
\centering
\begin{adjustwidth}{}{}
\resizebox{\textwidth}{!}{\rotatebox{90}{
\begin{tabular}{|r|r|r|r|r|r|r|} 
\hline
\multicolumn{1}{|c|}{\multirow{2}{*}{Run}} & \multicolumn{6}{c|}{PSO Experimental Results}                                                                                                                                                                         \\ 
\cline{2-7}
\multicolumn{1}{|c|}{}                     & \multicolumn{1}{l|}{SFLP-II} & \multicolumn{1}{l|}{Elapsed Time (s)} & \multicolumn{1}{l|}{mSFLP-III} & \multicolumn{1}{l|}{Elapsed Time (s)} & \multicolumn{1}{l|}{mKra30a} & \multicolumn{1}{l|}{Elapsed Time (s)}  \\ 
\hline
1                                          & 318.852793                   & 30                                    & 64984.15004                    & 76                                    & 126400.957298                & 130                                    \\ 
\hline
2                                          & 322.977083                   & 28                                    & 65613.267347                   & 81                                    & 121843.241837                & 115                                    \\ 
\hline
3                                          & 319.688772                   & 27                                    & 66182.148331                   & 77                                    & 131064.62114                 & 117                                    \\ 
\hline
4                                          & 382.774055                   & 29                                    & 65321.947243                   & 74                                    & 123193.080185                & 109                                    \\ 
\hline
5                                          & 328.181972                   & 26                                    & 68316.333054                   & 90                                    & 113509.606079                & 116                                    \\ 
\hline
6                                          & 273.754488                   & 27                                    & 63051.351074                   & 80                                    & 126382.984985                & 104                                    \\ 
\hline
7                                          & 277.929458                   & 35                                    & 64844.383484                   & 80                                    & 118793.658676                & 113                                    \\ 
\hline
8                                          & 333.791855                   & 29                                    & 62119.790359                   & 78                                    & 116690.745407                & 115                                    \\ 
\hline
9                                          & 331.588456                   & 25                                    & 62520.697372                   & 81                                    & 112328.409836                & 114                                    \\ 
\hline
10                                         & 312.062043                   & 28                                    & 60370.992424                   & 70                                    & 125547.949913                & 117                                    \\ 
\hline
11                                         & 347.300041                   & 26                                    & 63191.887421                   & 74                                    & 128467.742325                & 111                                    \\ 
\hline
12                                         & 276.72711                    & 29                                    & 66079.518539                   & 75                                    & 115259.262817                & 123                                    \\ 
\hline
13                                         & 346.501261                   & 38                                    & 68433.641548                   & 72                                    & 120682.394836                & 123                                    \\ 
\hline
14                                         & 324.057936                   & 29                                    & 59673.997963                   & 73                                    & 115606.839714                & 136                                    \\ 
\hline
15                                         & 294.477526                   & 33                                    & 65232.604935                   & 77                                    & 107996.773666                & 135                                    \\ 
\hline
16                                         & 354.944439                   & 28                                    & 62623.302681                   & 77                                    & 117466.287628                & 119                                    \\ 
\hline
17                                         & 278.580433                   & 25                                    & 63571.512451                   & 78                                    & 119787.763885                & 111                                    \\ 
\hline
18                                         & 373.62812                    & 25                                    & 65604.954414                   & 73                                    & 121933.62674                 & 111                                    \\ 
\hline
19                                         & 277.326199                   & 27                                    & 60121.135582                   & 71                                    & 120619.96003                 & 128                                    \\ 
\hline
20                                         & 312.888415                   & 26                                    & 65799.8442                     & 77                                    & 114528.809418                & 112                                    \\ 
\hline
21                                         & 349.500618                   & 28                                    & 62835.800159                   & 86                                    & 123371.321442                & 113                                    \\ 
\hline
22                                         & 324.432381                   & 30                                    & 67968.856182                   & 80                                    & 125755.185791                & 107                                    \\ 
\hline
23                                         & 259.640869                   & 29                                    & 61263.364929                   & 77                                    & 114837.336021                & 109                                    \\ 
\hline
24                                         & 324.809072                   & 28                                    & 64277.566399                   & 78                                    & 119911.446892                & 114                                    \\ 
\hline
25                                         & 315.892269                   & 30                                    & 63997.453415                   & 72                                    & 130395.024284                & 101                                    \\ 
\hline
26                                         & 367.985685                   & 31                                    & 62791.55909                    & 73                                    & 126777.9534                  & 109                                    \\ 
\hline
27                                         & 306.720735                   & 37                                    & 66069.317642                   & 77                                    & 116590.569717                & 132                                    \\ 
\hline
28                                         & 373.704857                   & 29                                    & 65694.84201                    & 77                                    & 131275.843658                & 145                                    \\ 
\hline
29                                         & 288.776388                   & 30                                    & 67406.459572                   & 76                                    & 118132.533211                & 120                                    \\ 
\hline
30                                         & 339.280296                   & 30                                    & 62731.473629                   & 81                                    & 126570.733612                & 113                                    \\ 
\hline
\multicolumn{1}{|l|}{Average}              & 321.292520833333             & 29.0666666666667                      & 64289.8051163                  & 77.0333333333333                      & 121057.4221481               & 117.4                                  \\ 
\hline
\multicolumn{1}{|l|}{Std. Dev}             & 32.8103600821748             & 3.20488133443597                      & 2356.26248290771               & 4.29501180868943                      & 5981.21601161922             & 10.1526283330459                       \\
\hline
\end{tabular}}}
\end{adjustwidth}
\caption{The entire experiment data we have collected using our PSO approach.}
\label{full-data-pso}
\end{table}

The performance of the GA approach in this study is definitely noteworthy. It produces the best solutions on average among the three approaches. However, based on the results, the GA approach does not scale well as the number of buildings increase, compared to our approach and the PSO approach. PSO definitely shows the best average runtimes. However, it produces the worst average fitness. For faster speed, we traded performance. This is where our approach shines. Our approach is the second best when it comes to solution quality as the problem scales higher. It is also the second best in terms of speed. This shows to us that our GWO approach provides a balance between speed and performance. Our approach also requires only a few parameters. We argue that this will simplify and speed up experimental setups and configuration in later studies and applications. Importantly, the results also indicate that there is promise in further exploring the applicability of the grey wolf optimization algorithm in solving the facility layout problem.

\chapter{Conclusion and Summary}

In this study, we proposed an alternative approach to the static unequal area facility layout problem, which was previously solved using, among other approaches, genetic algorithms and particle swarm optimization. Our approach utilizes the grey wolf optimization to solve the problem. We have introduced modifications to this metaheuristic in order for it to be able to produce feasible solutions. We have compared this modified GWO against a GA-based hybrid approach and a PSO approach. Results from our experiment indicate that the GA-based approach is generally better than our modified GWO approach and the PSO approach. However, they showed that there is promise in GWO as an algorithm for solving FLPs. The GA approach was shown to take longer to finish as the number of buildings increase. The PSO approach is the fastest among the three, but produces the worst solutions on average. Our approach, on the other hand, is the second best in both speed and solution quality. Hence, it provides a balance in speed and balance. Our approach is also simpler, making it easier to understand and experiment with. In the future, our proposed modified GWO may be further improved to produce significantly better results. Additionally, GWO is relatively new to the field, providing researchers with plentiful opportunities to improve the algorithm. Modifying the equations of our modified GWO, such as the decay rate of $\alpha$, is one avenue in which researchers may take to build upon our study.

\appendix

\chapter{What should be in the Appendix}

What goes in the appendices? Any material which impedes the smooth
development of your presentation, but which is important to justify the
results of a thesis. Generally it is material that is of too nitty-gritty
a level of detail for inclusion in the main body of the thesis, but which
should be available for perusal by the examiners to convince them
sufficiently. Examples include program listings, immense tables of data,
lengthy mathematical proofs or derivations, etc.



% ------------------------------------------------------------------------
\INPUT{biblio.bib} 
\setlinespacing{1.44}
\bibliographystyle{plain}
\bibliography{biblio}
\end{document}
% ------------------------------------------------------------------------
