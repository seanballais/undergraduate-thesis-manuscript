\chapter{Introduction} \label{sec:intro}

A common activity in universities and all academic institutions is course timetabling. This activity is a necessity in academic institutions as this is where classes are scheduled for the incoming semester. In some instances, even entire courses are scheduled for at least four years \cite{alves-novel-recursive}. Performing timetabling manually is tedious and schedules conflicts may not immediately be determined \cite{nlgdrl-obit}. As such, automating this procedure will be beneficial to academic institutions. This problem of automating course scheduling is actually formally known as the University Course Timetabling Problem (UTP) \cite{yousef-gpu-ga}\cite{socha-maxmin-ant-system}. Many works have already proposed different approaches to automate the timetabling process. Some of the algorithms that were utilized by proposed appraoches include great deluge \cite{gd-burked}\cite{nlgd-landa-silva}, machine learning \cite{nlgdrl-obit}, and genetic algorithms \cite{bedoya-non-standard-ga}\cite{raghavjee-ga-south-africa}\cite{yik-ga-timetabling}. Their results produce optimal timetables, or near to the optimal, at least for the case they were originally intended for. Applying their approaches to other institutions \textit{may} require some tweaking to fit in with the new environment they are going to be used in. There is no one approach that works for all cases as is. Due to this, there is a considerable amount of research that is done and is being done in further refining and discovering methods for automatic timetabling.

\section{The University Course Timetabling Problem}
The University Course Timetabling (UCT) Problem is a problem where $l$ lectures are being placed into a $t$ timeslots and $r$ rooms in such a way that it will result in a feasible timetable \cite{nlgd-landa-silva}. It was proven by Cooper, T. and Kingston, J. that this problem is an NP-complete problem which means it is difficult to produce feasible timetables \cite{cooper-timetable-problem-complexity}. Identifying whether a timetable is feasible or not is determined by two types of constraints: (a) hard constraints, and (b) soft constraints

Hard constraints are constraints that are not supposed to be violated. Violation of these constraints immediately leads infeasibility of a potential. Some proposed approaches immediately discard timetables that are infeasible \cite{nlgd-landa-silva}\cite{nlgdrl-obit}, while others would give high penalty costs for each hard constraint violated \cite{supachate-noval-approach-ga-thai}. The set hard constraints may differ from one approach to another. However, a common hard constraint that can be observed across approaches is that a student, and, naturally, a teacher as well, cannot attend two or more classes in the same timeslot \cite{nlgd-landa-silva}\cite{bedoya-non-standard-ga}. This is obvious as a person cannot be in two or more places at the same time. These hard constraints are what all constructed timetables must satisfy to be considered as a potential solution.

The counterpart to hard constraints are soft constraints. Unlike, hard constraints, soft constraints are \textit{technically} optional for timetables to satisfy. However, finding the best timetables for a given problem require minimization of the number of soft constraints being violated \cite{nlgd-landa-silva}. The set soft constraints differ from one case to another. Another case might require soft constraints that are not present in another case. For example, one approach \cite{nlgdrl-obit} has a soft constraint where students should not have only one class in a single day, while another \cite{supachate-noval-approach-ga-thai} does not and instead does not allow having more than three lectures of the same class schedule adjacently on the same day.	Many approaches do not consider hard constraints in computing the fitness value of a timetable. Instead, many only use the soft constraints as part of their fitness function \cite{sanjay-an-application-of-ga}\cite{nlgdrl-obit}\cite{bedoya-non-standard-ga}.

%
%
%
%
%