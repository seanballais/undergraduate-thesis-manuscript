\chapter{Introduction} \label{sec:intro}

Many images in typical use, from photographs to pixel art, are composed of pixels. Each pixel in an image contains a value that represents the colour it displays. These images are called, \textit{raster images}. This type of image are ideal when creating images that are rich and detailed. However, raster images suffer from their inability to scale well. Scaling raster images will output grainy and distorted results, sometimes even blurry ones\cite{rastervsvectorgraphics}. \textit{Vector graphics} are an alternative to raster graphics, being resolution-independent and based on objects (often called \textit{primitives}) and mathematical statements\cite{barendrecht2018locally}\cite{rastervsvector}. As a result, vector graphics generally tend to be easier to manipulate\cite{barendrecht2018locally}.

\textit{Semi-structured images} can benefit from vectorization. As defined by Hoshyari, S., et.al., these images "consists of distinctly coloured regions with piecewise continuous boundaries and visually pronounced corners". Frequently, many digital artworks, specifically computer icons, comic book imagery, and simple graphic illustrations, are considered to be semi-structured images. Many legacy semi-structured images, typically those considered to be icons and clip arts, are stored in raster formats\cite{hoshyari2018perceptiondriven}. Additionally, in manual vectorization of images, artists often have to hand-trace each stroke and region using specialized software, such as Inkscape or Adobe Illustrator, which is a labourious task\cite{matheson2018smoothing}. Automatic vectorization tools, such as Adobe Live Trace and Potrace, often demand more time rectifying the numerous tracing errors the tool produces by hand\cite{matheson2018smoothing}, and does not always produce results aligned with human perception\cite{hoshyari2018perceptiondriven}. Most vectorization algorithms are geared towards natural images (i.e. photographs) which frequently produce results not aligned with human expectations on artist-drawn imagery\cite{hoshyari2018perceptiondriven}. These cases call for a robust vectorization algorithm for semi-structured images that produces outputs that primarily align well with human-perceived results  As noted by Hoshyari, S., et. al., human observers have a clearer mental image of the expected vector result from a raster semi-structured data. As such, vectorization results of semi-structured images require to be as close as possible to the mental image produced by humans\cite{hoshyari2018perceptiondriven}.

Due to the fact that semi-structured images consists of distinct regions based on colour, the general flow of vectorizing semi-structured images can simply be boiled down into extracting those distinct regions and their boundaries, performing vectorization on those boundaries and appropriately colouring them, and combining the resulting vectors into a single vector object. The major technical challenge for this is vectorizing the region boundaries via piecewise free-form vector curves\cite{hoshyari2018perceptiondriven}. The most recent work on this topic is that of Hoshyari, S., et. al. Their work revolves primarily on accuracy and two key principles of Gestalt psychology: \textit{simplicity}, and \textit{continuity}. Simplicity simply states that human observers prefer simpler interpretations of geometric interpretations of raster images. Continuity states that human observers have a tendency to group stimuli into one continuous curves and patterns. Jagged raster boundaries are perceived to be piecewise smooth curves in their vector forms. In addition, observers would only mentally segment boundaries at a small set of discontinuous corners. Their work, however, becomes computationally expensive on larger inputs. Despite utilizing a corner classifier developed via machine learning as part of their vectorization, their classifier still produces results that require them to have an additional processing step, called \textit{corner removal}, to produce better results.

I propose a method that is an improvement of the work done by Hoshyari, S. et. al. by improving the corner classifier in an attempt to render the corner removal step unnecessary, and utilizing machine learning to automatically draw the appropriate vector curves and lines, without relying on greedily computing and fitting the appropriate curves on the region boundaries. The latter, of which, is the method utilized in the framework devised by Hoshyari, S., et. al., which, as stated earlier, is computationally expensive at higher image resolutions.