\chapter{Introduction} \label{sec:intro}

In November 8, 2013, Super Typhoon Yolanda (internationally known as Haiyan) made landfall in the Philippines. 14 million people were affected, and a total of \$5.8 billion in damages were done \cite{Reid}. More recently, at the time of writing, in November of 2020, Cagayan Valley, Philippines was wreaked with flooding and landslides which resulting in the loss of lives, and affected local infrastructure \cite{CagayanFloodingNews}. The aforementioned events are just two of the many natural disasters that affected the world. Many more are to arrive in the future.

Despite all the calamities that will be affecting communities around the world, building constructions and development will still take place. As such, developers, architects, and engineers must take into account natural hazards. This consideration for natural hazards is important, especially given today's world's attitude towards sustainability \cite{Padgett2013}. Sustainability deals with ensuring that the demands today are met, while ensuring that future generations can still meet their own demands. It involves meeting the needs of the environment, society, and economy in a balanced manner \cite{Dimian2014}. None of the three must be compromised when satisfying one of them. Natural risk hazard mitigation is a part of sustainability. Damages by natural disasters impact the environment, society, and the economy. Key business and community functions get disrupted, economic losses will be incurred, and resources will be depleted and energy will expended as a result of rebuilding efforts. Also, as the United States's Federal Emergency Management Agency has noted, building in high hazard areas will result in higher costs in the long-term due to disaster recovery when natural disasters strike \cite{UnitedStatesFederalEmergencyManagementAgency2000}. Building structures with natural hazard mitigation in mind will result in cheaper costs (environmentally-wise, economically-wise, and socially-wise) during post-disaster recovery as well as minimizing economic and environmental impacts, which results in improved sustainability \cite{Padgett2013}.

Many aspects contribute to the resilience of a building against the forces of nature. Building design and construction quality generally affect how a building withstands against flooding, storms, earthquakes \cite{Lewis2012}, and similar events. However, a key aspect of natural hazard mitigation is the location of buildings. This holds especially true when developers seek to mitigate against flooding, landslides, tsunamis, and storm surges \cite{WBDGSecure/SafeCommittee}. The problem, however, is determining the locations of buildings in as optimal as possible way. This research work seeks to deal with that problem. There have been many related researches already. All of them, including this problem, is classified under the \textbf{facility layout problem}.

\section{Facility Layout Problem}
The facility layout problem (FLP) is defined to be problem of arranging a set of facilities or machines in a pre-determined area, or a set of possible locations (such as in the work of Farmakis, P., and Chassiakos, A. \cite{Farmakis2018}), in such a way that is in line with some criteria or objectives and under some constraint \cite{Hosseini-Nasab2018}. FLP is also considered to be an NP-Hard problem \cite{Drira2007}. This has resulted in many research works that utilize heuristics (such as simulated annealing and genetic algorithms). Note that there are also works that utilize exact methods, which seek to find the \textit{optimal} solution for a problem. However, the NP-Hard nature of the FLP prevents them from finding the solution within reasonable time \cite{Asl2015}.

\section{Polygonally-Bounded Unequal Area Static Facility Layout Problem}
