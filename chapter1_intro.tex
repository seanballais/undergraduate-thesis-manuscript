\chapter{Introduction} \label{sec:intro}

In November 8, 2013, Super Typhoon Yolanda (internationally known as Haiyan) made landfall in the Philippines. 14 million people were affected, and a total of \$5.8 billion in damages were done \cite{Reid}. More recently, at the time of writing, in November of 2020, Cagayan Valley, Philippines was wreaked with flooding and landslides which resulting in the loss of lives, and affected local infrastructure \cite{CagayanFloodingNews}. The aforementioned events are just two of the many natural disasters that affected the world. Many more are to arrive in the future.

Despite all the calamities that will be affecting communities around the world, building constructions and development will still take place. As such, developers, architects, and engineers must take into account natural hazards. This consideration for natural hazards is important, especially given today's world's attitude towards sustainability \cite{Padgett2013}. Sustainability deals with ensuring that the demands today are met, while ensuring that future generations can still meet their own demands. It involves meeting the needs of the environment, society, and economy in a balanced manner \cite{Dimian2014}. None of the three must be compromised when satisfying one of them. Natural risk hazard mitigation is a part of sustainability. Damages by natural disasters impact the environment, society, and the economy. Key business and community functions get disrupted, economic losses will be incurred, and resources will be depleted and energy will expended as a result of rebuilding efforts. Also, as the United States's Federal Emergency Management Agency has noted, building in high hazard areas will result in higher costs in the long-term due to disaster recovery when natural disasters strike \cite{UnitedStatesFederalEmergencyManagementAgency2000}. Building structures with natural hazard mitigation in mind will result in cheaper costs (environmentally-wise, economically-wise, and socially-wise) during post-disaster recovery as well as minimizing economic and environmental impacts, which results in improved sustainability \cite{Padgett2013}.

Many aspects contribute to the resilience of a building against the forces of nature. Building design and construction quality generally affect how a building withstands against flooding, storms, earthquakes \cite{Lewis2012}, and similar events. However, a key aspect of natural hazard mitigation is the location of buildings. This holds especially true when developers seek to mitigate against flooding, landslides, tsunamis, and storm surges \cite{WBDGSecure/SafeCommittee}. The problem, however, is determining the locations of buildings in as optimal as possible way. This research work seeks to deal with that problem. There have been many related researches already. All of them, including this problem, is classified under the \textbf{facility layout problem}.

\section{Facility Layout Problem}
The problem of arranging a set of facilities and/or machines in a pre-determined area, or a set of possible locations (such as in the work of Farmakis, P., and Chassiakos, A. \cite{Farmakis2018}) is called the facility layout problem (FLP). The facilities and/or machines are arranged in such a way that the resulting layout is in line with some criteria or objectives and under certain constraints. These constraints, which must not be violated, include shape, size, orientation, pick-up/drop-off points \cite{Hosseini-Nasab2018}, and usable area \cite{Fernando2015}. Facilities and/or machines must also not overlap. Solutions that satisfy the aforementioned conditions are called feasible solutions \cite{Meller1996}.

Numerous fields have already been solving specific instances of the facility layout problem. // Continue.

Generally, the facility layout problem is considered to be an \textbf{NP-Hard} problem \cite{Drira2007}. Hosseini-Nasab, H., Fereidouni, S., and Fatemi, S. have noted in their systematic review of FLP that most researches dealing with the facility layout problem model their problems either as a quadratic assignment problem (QAP) or a mixed integer programming problem \cite{Hosseini-Nasab2018}. According to Drira, A., Pierreval, H., and Hajri-Gabouj, S., the former is sometimes used in discrete FLP formulations, while the latter is often used in continuous formulations \cite{Drira2007}. Discrete and continuous FLP formulations will be discussed later. \textbf{Quadratic assignment problems} deal with placing $n$ facilities in $n$ locations in such a way that minimizes the assignment cost. The assignment cost is the sum of all facility pairs's flow rate between each other multiplied by their flow rate \cite{QAPDefinition}. This assignment cost is commonly seen in many FLP researches, as we will discuss later. QAP is also known to be an NP-Hard problem \cite{Garey1979}. It should be noted though that \textit{some} instances of QAP are easy to solve \cite{Feizollahi2015}. The other modeling framework, \textbf{mixed integer programming}, can solve problems with both discrete decisions and continuous variables. An example of such problem is the assignment problem \cite{Richards2005}, which the FLP can be classified under. In this formulation, a set of integer and real-valued integers are being optimized based on an objective function that is being minimized or maximized, while satisfying constraints which are linear equations or inequalities \cite{Wolsey2008}. Mixed integer programming, when in the context of optimization, is also known to be NP-Hard \cite{Richards2005}. These two formulations being known to be generally NP-Hard proves that FLP is indeed generally NP-Hard.

The fact that FLP is an NP-Hard problem has resulted in many research works that utilize heuristics (such as simulated annealing and genetic algorithms). Note that there are also works that utilize exact methods, which seek to find the \textit{optimal} solution for a problem. However, the NP-Hard nature of the FLP prevents them from finding the solution in large problems within reasonable time \cite{Asl2015}.

\subsection{The Basic Mathematical Model}
Each problem instances of the facility layout problem naturally will have their own mathematical models tailor-fit for their problem instance. Nevertheless, based on our observations and from readings, most of those models are derivatives of or use (such as in \cite{Garcia-Hernandez2013}, \cite{Lin2019}, and \cite{Navarro2016}) what will be calling a basic minimization function, which is defined as:

$$
\text{min} F = \sum_{i=1}^{n}\sum_{j=1}^{n}c_{ij}f_{ij}d_{ij}
$$

where $N$ is the number of facilities, $c_{ij}$ is the cost of handling materials between locations $i$ and $j$, $f_{ij}$ is the flow rate between $i$ and $j$, and $d_{ij}$ is the distance between the centroids of $i$ and $j$. The distance function may differ from work to work. For example, Liu, J., et. al. uses the Manhattan distance in their work \cite{Liu2018}, while in the work of Ripon, K. S. N., et. al., Euclidean distance was used \cite{Ripon2013}. In works that derive from this formula, such as in \cite{Farmakis2018}, \cite{Solimanpur2008}, and \cite{Peng2018}, it was observed that $d_{ij}$, or a similar variable or expression, is commonly present in the work's objective function while $c_{ij}$ and $f_{ij}$ \textit{may} be present and/or the work uses more or fewer variables.

Drira, A., Pierreval, H., and Hajri-Gabouj, S. note the same observation but showcase a slightly differing formula in their 2007 survey of facility layout problems. Unlike the basic minimization formula above, their formula has $f_{ij}$ and $c_{ij}$ combined. They also note that the function above is typically used in continuous formulations of the FLP. The discrete formulation uses a similar function, but ensures that a facility is only in one location, a location only contains one facility, and makes sure that only pairs of locations that contain facilities contribute to the fitness value of a solution. Additionally, they mention that the function is also subject to the following constraints: (1) facilities must obviously not overlap with one another, and (2) the total area used by the facilities must be equal to or less than the allotted area \cite{Drira2007}. These constraints have been observed to be generally in many FLP works.

\subsection{Discrete vs Continuous Formulations}

\subsection{Static vs Dynamic Facility Layout Problems}


\section{Polygonally-Bounded Unequal Area Static Facility Layout Problem}
