\chapter{Results and Discussion}
The results of our experiments with our data set will be presented in this chapter. 30 runs of each approach that produced feasible solutions are included in the results. Note that some runs produce an infeasible solution. This due to the non-deterministic nature of metaheuristics, which will cause it to produce infeasible solutions sometimes.

\section{Environment}
All of the approaches were run in the following hardware and software configurations:

\begin{itemize}
	\item Hardware
	\begin{itemize}
		\item \textbf{CPU}: Intel Core i3-5010U @ 2.1GHz
		\item \textbf{GPU}: NVIDIA GeForce 920M
		\item \textbf{RAM}: 4GB
	\end{itemize}
	\item Software
	\begin{itemize}
		\item \textbf{OS}: elementaryOS 5.1.7 Hera
		\item \textbf{Linux Kernel Version}: 5.4.0-80-generic
	\end{itemize}
\end{itemize}

\section{Experiments}
Each approach has their own parameters, and the values we have set for those parameters are shown in Table \ref{approach-parameters}. Both approaches use a population size of 50, and a maximum number of iterations of 400. The following parameter values for the PSO approach were taken from the work of Jolai, F., Tavakkoli-Moghaddam, R., and Taghipour, M. \cite{Jolai2012}.

\begin{table}[h!]
	\centering
	\begin{tabular}{|l|l|l|}
		\hline
		\textbf{Approach}   & \textbf{Parameter} & \textbf{Value} \\ \hline
		GWO                 & c                  & 4              \\ \hline
		\multirow{3}{*}{GA} & Mutation Rate      & 0.05           \\ \cline{2-3} 
		& Tournament Size    & 4              \\ \cline{2-3} 
		& No. of Elites (EN) & 5              \\ \hline
		\multirow{3}{*}{PSO} & w      & 0.05           \\ \cline{2-3} 
		& c1    & 2              \\ \cline{2-3} 
		& c2 	& 2              \\ \hline
	\end{tabular}
	\caption{Parameter values of the GWO, GA, and PSO approaches.}
	\label{approach-parameters}
\end{table}

The results obtained for each approach is shown in Tables \ref{approach-ga-results}, \ref{approach-gwo-results}, and \ref{approach-pso-results}, respectively. As what the tables show, the competing genetic algorithm approach produces a solution that is better than our proposed GWO approach and the PSO approach, with a fitness averages of $273.537430766667$, $50422.3174052667$, and $88087.9226730667$ for the SFLP-II, mSFLP-III, and mKra30a problem configurations respectively. This is compared to our proposed approach's fitness averages of $290.3857809$, $52702.9314929$, $101874.328208933$. Fortunately for our approach, the PSO approach obtained the fitness averages of $324.356679733333$, $64181.9165261333$, and $118785.277772567$, proving that our GWO is not the worst approach. For SFLP-II, the best and worst solutions have fitnesses of $221.042599$ and $324.874681$ for the GA, respectively, compared to our approach's $234.250481$ and $339.099181$. The PSO approach obtained $268.59568$ and $380.688077$s. In this data set, our approach was capable of producing the best solution. For mSFLP-III, the best and worst solutions have a fitness of $46802.666237$ and $53474.353325$ for the GA, respectively, compared to our approach's $48951.787331$ and $53474.353325$ and the PSO approach's $60037.826591$ and $68194.108383$. Lastly, for mKra30a, the best and worst solutions have fitnesses of $76651.01432$ and $98512.468674$, respectively, with our approach obtaining $90455.74585$ and $118315.534424$. PSO produces the poorest best and worst solutions with fitnesses of $106178.045845$ and $130915.770554$. From the results, the competing GA approach is the most stable among the three, basing from the lower standard deviation in all data sets, with $25.3555205871573$, $1630.42444301206$, and $5077.72744984237$ for SFLP-II, mSFLP-III, and mKra30a, respectively. This is compared to our approach's $32.4373567833344$, $2224.64491886288$, and $7190.18569614101$, and the PSO approach's $38.4539160542525$, $1956.45733138936$, and $6138.39410532314$. % TODO: Explain why this is the case.

\begin{figure}[h!]
\centering
\includegraphics[scale=0.65]{./images/chap07-rd/approaches-average-runtime-over-no-of-buildings.png}
\caption{The average runtime (s) of each of the approaches as the number of buildings in a data set increase.}
\label{graph-approaches-runtime-no-buildings}
\end{figure}

The genetic algorithm approach is also faster when SFLP-II is being used with an average run time of $27$s, compared to our approach's $89.5666666666667$s and the PSO approach's $89.5666666666667$s. However, as the number of buildings increase, the average runtime of the GA approach becomes worse compared to the two other approaches. With mSFLP-III, GA takes $145.766666666667$s, while our approach and the PSO approach takes $197.733333333333$s and $341.266666666667$s, respectively. GA is still faster than GWO in this data set, but it is already slower than the PSO approach. Moving towards mKra30a, we can see that GA now takes $341.266666666667$s. This is longer than our approach's $341.266666666667$s, and the PSO approach's $114.366666666667$s. Figure \ref{graph-approaches-runtime-no-buildings} shows this observation. We can attribute this faster increase in average runtime as the number of buildings increase in the GA approach to its local search methods. Since the local search methods perform a relatively exhaustive search in order to find a better solution, the GA will take more time to finish executing. Hence, we observe this phenomenon. This is not the case with GWO and PSO, due to the lack of local search methods. GWO may have taken a longer time due to the amount of operations that are performed in the metaheuristic compared to PSO. Better implementations, especially those that utilize SIMD operations, for both approaches may reduce the gap in terms of average run time between the two. However, basing from the equations in both metaheuristics, it is likely that PSO will remain faster than GWO. Further studies, however, are required to exactly determine how well each approach scales with regards to the number of buildings.

\begin{table}[h!]
\begin{adjustwidth}{-1.15in}{}
\centering
\begin{tabular}{|l|l|l|l|l|l|}
	\hline
	\multicolumn{1}{|c|}{\multirow{2}{*}{\textbf{Problem}}} & \multicolumn{5}{c|}{\textbf{Genetic Algorithm}}                                                                                                                                                                                            \\ \cline{2-6} 
	\multicolumn{1}{|c|}{}                                  & \multicolumn{1}{c|}{\textbf{Best}} & \multicolumn{1}{c|}{\textbf{Worst}} & \multicolumn{1}{c|}{\textbf{Avg.}} & \multicolumn{1}{c|}{\textbf{Std. Dev.}} & \multicolumn{1}{c|}{\textbf{Avg. Runtime (s)}} \\ \hline
	SFLP-II                                                 & 221.042599                                  & 324.874681                                   & 273.537430766667                      &
	25.3555205871573						& 27                                   \\ \hline
	mSFLP-III                                               & 46802.666237                                & 53474.353325                                 & 50422.3174052667	                      & 1630.42444301206                                  & 145.766666666667                           \\ \hline
	mKra30a                                               & 76651.01432                                & 98512.468674                                 &
	88087.9226730667							&
	5077.72744984237	                        &
	435.033333333333								\\ \hline
\end{tabular}
\end{adjustwidth}
\caption{Results obtained from using the competing GA approach.}
\label{approach-ga-results}
\end{table}

\begin{table}[h!]
\begin{adjustwidth}{-1.18in}{}
\centering
\begin{tabular}{|l|l|l|l|l|l|}
	\hline
	\multicolumn{1}{|c|}{\multirow{2}{*}{\textbf{Problem}}} & \multicolumn{5}{c|}{\textbf{GWO}} \\ \cline{2-6} 
	\multicolumn{1}{|c|}{}                                  & \multicolumn{1}{c|}{\textbf{Best}} & \multicolumn{1}{c|}{\textbf{Worst}} & \multicolumn{1}{c|}{\textbf{Avg.}} & \multicolumn{1}{c|}{\textbf{Std. Dev.}} & \multicolumn{1}{c|}{\textbf{Avg. Runtime (s)}} \\ \hline
	SFLP-II                                                 & 234.250481                                  & 339.099181                                   & 290.3857809                      & 32.4373567833344                                 & 89.5666666666667                                  \\ \hline
	mSFLP-III                                               & 48951.787331                                & 57804.257366                                 & 52702.9314929						         & 2224.64491886288                              & 197.733333333333                               \\ \hline
	mKra30a                                               & 90455.74585                                & 118315.534424                                 &
	101874.328208933							&
	7190.18569614101							&
	341.266666666667						\\ \hline
\end{tabular}
\end{adjustwidth}
\caption{Results obtained from our proposed GWO approach.}
\label{approach-gwo-results}
\end{table}

\begin{table}[h!]
\begin{adjustwidth}{-1.18in}{}
\centering
\begin{tabular}{|l|l|l|l|l|l|}
	\hline
	\multicolumn{1}{|c|}{\multirow{2}{*}{\textbf{Problem}}} & \multicolumn{5}{c|}{\textbf{PSO}} \\ \cline{2-6} 
	\multicolumn{1}{|c|}{}                                  & \multicolumn{1}{c|}{\textbf{Best}} & \multicolumn{1}{c|}{\textbf{Worst}} & \multicolumn{1}{c|}{\textbf{Avg.}} & \multicolumn{1}{c|}{\textbf{Std. Dev.}} & \multicolumn{1}{c|}{\textbf{Avg. Runtime (s)}} \\ \hline
	SFLP-II                                                 & 268.59568                                  & 380.688077                                   &
	324.356679733333							&
	38.4539160542525							&
	31.3666666666667							\\ \hline
	mSFLP-III                                               & 60037.826591                                & 68194.108383                                 &
	64181.9165261333					          &
	1956.45733138936						&
	66.9666666666667						\\ \hline
	mKra30a                                               & 106178.045845                                & 130915.770554                                 &
	118785.277772567							&
	6138.39410532314							&
	114.366666666667						\\ \hline
\end{tabular}
\end{adjustwidth}
\caption{Results obtained from our proposed PSO approach.}
\label{approach-pso-results}
\end{table}

% TODO: Update below.

Figures \ref{graph-ga-vs-gwo-sflp2} and \ref{graph-ga-vs-gwo-msflp3} show the fitness graphs of the best solutions using the SFLP-II and mSFLP-III data sets. The non-linearity of the graph of our GWO approach that is obvious in Figure \ref{graph-ga-vs-gwo-sflp2} is due to the nature of our approach. All solutions in a population are replaced. Hence, the best solutions may be replaced by poorer solutions. This characteristic is not as obvious in Figure \ref{graph-ga-vs-gwo-msflp3}. In Figure \ref{graph-ga-vs-gwo-sflp2} with the SFLP-II data set, the GA approach converges was than the GWO approach. This is due to how our approach replaces all solutions in the population for the next iteration. The size of the bounding region in SFLP-II may also be reason as it does not have a huge space for our approach to move buildings in various directions and prevent intersections as much as possible. This is the different from the mSFLP-III where the bounding region is larger, allowing buildings to less likely intersect with one another. Figure \ref{graph-ga-vs-gwo-msflp3} showcases the behaviour of our GWO approach with a larger bounding region.

\begin{figure}[h!]
\centering
\begin{adjustwidth}{-0.9in}{}
\includegraphics[scale=0.75]{./images/chap07-rd/gwo-sflp2-graph.png}
\end{adjustwidth}
\caption{Fitness graph of the best solutions of the competing GA approach, and our GWO approach using the SFLP-II data set.}
\label{graph-ga-vs-gwo-sflp2}
\end{figure}

\begin{figure}[h!]
\centering
\begin{adjustwidth}{-0.1in}{}
\includegraphics[scale=0.65]{./images/chap07-rd/gwo-msflp3-graph.png}
\end{adjustwidth}
\caption{Fitness graph of the best solutions of the competing GA approach, and our GWO approach using the mSFLP-III data set.}
\label{graph-ga-vs-gwo-msflp3}
\end{figure}

% TODO: Add figure of best generated solutions.

Despite the wins showcased by the competing GA approach, our proposed approach is faster when mSFLP-III is used with $185.7$s for our approach, while the GA approach takes $260.56666667$s. We also argue that the competing approach is already a hybridized approach with phases for a local search to be performed. Our approach can be considered to be a relatively pure adaptation of classical GWO, rather than a hybrid approach. Additionally, the results are not far from the results generated by the competing approach. They show that the simplicity of our approach (which only uses three parameters needed, compared to our competing approach's six) does not prevent our proposed algorithm from almost reaching competitiveness. They also indicate that there is promise in further exploring the applicability of the grey wolf optimization algorithm in the facility layout problem.

