\chapter{Statement of the Problem} 
\label{sec:ResearchQ}

% Arranging a set of buildings with varying areas in a region of space for a long period of time according to a certain criteria is an important endeavour in many fields. Effective layouts of buildings result in better productivity and reduced expenses. Unfortunately, this type of problem and facility layout problems, in general, are also known to be NP-Hard. As such, metaheuristics have been widely used for solving FLPs. Many metaheuristics have been developed over the course of years, and may provide new key insights in solving FLPs. As such, ... %

% In many industries and fields, effective layouts of buildings, assets, or facilities may result in better productivity and reduced expenses. These facitilies may have varying area sizes and may be in the same location for a relatively long period of time. Arranging these facilities within a region of space is not a simple problem to solve. In fact, it is known to be an NP-Hard problem. As such, finding solutions

In many industries and fields, arranging buildings, assets, or facilities of varying areas in positions that will remain the same for a long amount of time according a certain criteria may result in better productivity, reduced expenses, and improved operations efficiency. Unfortunately, the best arrangements are extremely difficult and take too long to obtain. As a matter of fact, problems like these are determined to be NP-Hard. Thus, it is important to develop an approach that lets us find arrangements that are good enough within a reasonable amount of time.
