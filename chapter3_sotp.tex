\chapter{Research Question or Problem Statement} 
\label{sec:ResearchQ}

Engineering theses tend to refer to a ``problem" to be solved where other
disciplines talk in terms of a ``question" to be answered. In either case,
this section has three main parts:

\label{sec:Steps}
\begin{enumerate}
\item a concise statement of the question that your thesis tackles
\item justification, by direct reference to Chapter~\ref{sec:intro}, that your question is previously unanswered
\item discussion of why it is worthwhile to answer this question.
\end{enumerate}

Item 2 above is where you analyze the information which you presented in
Chapter~\ref{sec:ResearchQ}. For example, maybe your problem is to ``develop a Zylon algorithm capable of handling very large scale problems in reasonable
time" (you would further describe what you mean by ``large scale" and
``reasonable time" in the problem statement). Now in your analysis of the
state of the art you would show how each class of current approaches fails
(i.e. can handle only small problems, or takes too much time). In the last
part of this section you would explain why having a large-scale fast Zylon
algorithm is useful; e.g., by describing applications where it can be
used.

Since this is one of the sections that the readers are definitely looking
for, highlight it by using the word ``problem" or ``question" in the title:
e.g. ``Research Question" or ``Problem Statement", or maybe something more
specific such as ``The Large-Scale Zylon Algorithm Problem."

