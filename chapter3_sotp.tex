\chapter{Statement of the Problem} 
\label{sec:ResearchQ}

Raster images are composed of a pixel matrix. This makes their data representation simple. However, this reduces the amount of detail and quality they have. This limitation is clearer when scaling raster images. This motivates the use of an alternative form of representing images. One form is vector images, which uses mathematical equations to represent an image. The process of converting a raster image to a vector image is called \textit{vectorization}. Semi-structured imagery, such as those used in graphic designs, is one of the classes of images that would benefit from vectorization. This process will allow such images to be easily scaled without sacrificing quality nor detail.

There have been numerous works that tackles image vectorization for semi-structured images. Many of these works primarily use a curve optimization algorithm such as variants of NEWUOA and conjugate gradient. A machine learning approach has been used in one of the works, but as a preprocessing step only \cite{hoshyari2018perceptiondriven}. Deep learning have been used to solve problems in multiple domains such as natural language processing (NLP), object detection, and playing board games. No other known work has applied deep learning as a core step in image vectorization. This study will deal with such application.

% There have been numerous works that tackles image vectorization for various inputs. Most of these works have been targeted at natural imagery where many utilize image segmentation. These works do not necessarily be appropriate for semi-structured imagery. As such, many works have also sprung up to tackle these types of images. However, many of the works such as those by Hoshyari, S., et. al \cite{hoshyari2018perceptiondriven}, and Kopf, J., and Lischinski, D. \cite{depixelizingpixelart}, only handle clean, quantized non-anti-aliased images. As such, there has been no work that has been targeted at dealing with anti-aliased semi-structured input with speeds nearing those tested by Hoshyari, S., et. al.. Applying the proposed method by Hoshyari, S., et. al., as the authors have stated, have an impact on the resulting vectorization. Additionally, their work becomes computationally expensive on larger inputs. The work by Xie, G., Sun, X., Tong, X., and Nowrouzezahrai, D., may be applied to semi-structured images \cite{hierarchicaldiffusioncurves}. However, their method is computationally expensive and, as extrapolated from their experimental setup, require expensive hardware (the Nvidia Quadro 6000 was the most expensive of the components in their setup).

% There are numerous raster semi-structured images that are anti-aliased, especially those that have a larger resolution. These images are used in posters, logos, and graphic designs. Vectorization of those images will improve their quality and allow them to support higher resolutions without any degradation of the quality.
