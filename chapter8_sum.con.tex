\chapter{Conclusion and Summary}

In this study, we proposed an alternative approach to the static unequal area facility layout problem, which was previously solved using, among other approaches, genetic algorithms and particle swarm optimization. Our approach utilizes the grey wolf optimization to solve the problem. We have introduced modifications to this metaheuristic in order for it to be able to produce feasible solutions. We have conducted experiments varying the value of the $c$ parameter of our proposed GWO approach, and compared this approach against a GA-based hybrid approach and a PSO approach. Results from our experiment indicate that the value of the $c$ parameter impacts the performance of our approach, and that the appropriate value for the parameter is correlated with the size of the bounding region used. Additionally, we have found that the GA-based approach is generally better than our modified GWO approach and the PSO approach. However, they showed that there is promise in GWO as an algorithm for solving FLPs. The GA approach was shown to take longer to finish as the number of buildings increase. The PSO approach is the fastest among the three, but produces the worst solutions on average. Our approach, on the other hand, is the second best in both speed and solution quality. Hence, it provides a balance in speed and balance. Our approach is also simpler, making it easier to understand and experiment with. In the future, our proposed modified GWO may be further improved to produce significantly better results. Additionally, GWO is relatively new to the field, providing researchers with plentiful opportunities to improve the algorithm. Modifying the equations of our modified GWO, such as the decay rate of $\alpha$, is one avenue in which researchers may take to build upon our study. Another avenue is to identify whether the $c$ parameter's value can be mathematically modelled instead of being a parameter.
